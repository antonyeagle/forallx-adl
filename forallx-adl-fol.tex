%!TEX root = forallx-adl.tex
\part{The Language of Quantified Logic}
\label{ch.FOL}
\addtocontents{toc}{\protect\mbox{}\protect\hrulefill\par}
\chapter{Building Blocks of \textnormal{\FOL}}\label{s:FOLBuildingBlocks}
\section{The Need to Decompose Sentences}
Consider the following argument, which is obviously valid in English:
\begin{quote}
\label{willard1}
Willard is a logician. All logicians wear funny hats. So Willard wears a funny hat.
\end{quote}
To symbolise it in \TFL, we might offer a symbolisation key:
\begin{ekey}
\item[L] Willard is a logician.
\item[A] All logicians wear funny hats.
\item[F] Willard wears a funny hat.
\end{ekey}
And the argument itself becomes:
$$L, A \ttherefore F$$
This is \emph{invalid} in \TFL\ – there is a valuation on which the premises are true and the conclusion false. But the original English argument is clearly valid.

The problem is not that we have made a mistake while symbolising the argument. This is the best symbolisation we can give in \TFL. The problem lies with \TFL\ itself. `All logicians wear funny hats' is about both logicians and hat-wearing. By not retaining this in our symbolisation, we lose the connection between Willard's being a logician and Willard's wearing a hat.

Another example. This argument is also intuitively valid:
\begin{earg}
	\item[] John loves James;
	\item[So:] John loves someone.
\end{earg} Again, the best \TFL\ symbolisation is the invalid $P \ttherefore Q$. The validity of this argument depends on the internal structure of the sentences, and specifically the connection between the name `James' and the phrase `someone'.

The basic units of \TFL\ are atomic sentences, and \TFL\ cannot decompose these. None of the sentences in the arguments above have any truth-functional connectives, so must be symbolised as atomic sentences of \TFL. To symbolise arguments like the preceding, we will have to develop a new logical language which will allow us to \emph{split the atom}. We will call this language \FOL, and the study of this language and its features is \emph{quantified logic}. 

The details of \FOL\ will be explained throughout this chapter, but here is the basic idea about how to split the atom(ic sentence). The key insight is that many natural language sentences have \emph{subject-predicate} structure, and some arguments are valid in virtue of this structure. \FOL\ adds to \TFL\ some resources for modelling this structure – or perhaps more accurately, it allows us to model the predicate-name structure of many sentences, along with any truth-functional structure.

\paragraph{Names} First, we have \emph{names}. In \FOL, we indicate these with lower case italic letters. For instance, we might let `$b$' stand for Bertie, or let `$i$' stand for Willard. The names of \FOL\ correspond to proper names in English, like `Willard' or `Elyse', which also stand for the things they name. 

\paragraph{Predicates} Second, we have \emph{predicates}. English predicates are expressions like `\blank\ is a dog' or `\blank\ is a logician'. These are not complete sentences by themselves. In order to make a complete sentence, we need to fill in the gap. We need to say something like `Bertie is a dog' or `Willard is a logician'. In \FOL, we indicate predicates with upper case italic letters. For instance, we might let the \FOL\ predicate `$D$' symbolise the English predicate `\blank\ is a dog'. Then the expression `$Db$' will be a sentence in \FOL, which symbolises the English sentence `Bertie is a dog'. Equally, we might let the \FOL\ predicate `$L$' symbolise the English predicate `\blank\ is a logician'. Then the expression `$Li$' will symbolise the English sentence `Willard is a logician'.

\paragraph{Quantifiers} Third, we have quantifier phrases. These tell us \emph{how much}. In English there are lots of quantifier phrases. But in \FOL\ we will focus on just two:  `all'/`every' and `there is at least one'/`some'. So we might symbolise the English sentence `there is a dog' with the \FOL\ sentence `$\exists x Dx$', which we would naturally read aloud as `there is at least one thing, $x$, such that $x$ is a dog'.

That is the general idea. But \FOL\ is significantly more subtle than \TFL. So we will come at it slowly. 


\section{Singular Terms} % (fold)
\label{sec:singular_terms}

In English, a \define{singular term} is a noun phrase that refers to a \emph{specific} person, place, or thing.  The word `dog' is not a singular term, because there are a great many dogs. The phrase `Bertie' is a singular term, because it refers to a specific terrier. Here are some further examples:
\begin{itemize}
	\item \textbf{Proper Names}, e.g., `\underline{Bertie} enjoys playing fetch';\\
	`\underline{Scott Morrison} breached the human rights of children seeking asylum'; 
	\item \textbf{Definite Descriptions}, e.g., `\underline{The oldest person} is a woman';\\ `Ortcutt is \underline{the shortest spy}';
	\item \textbf{Possessives}, e.g., `\underline{Antony's eldest child} loves coding';
	\item \textbf{Pronouns}, e.g., `\underline{She} \emph{(points)} plays violin';\\
	`Icecream is delicious. Everyone loves \underline{it}';
	\item \textbf{Demonstratives}, e.g., `\underline{That loud dog} is so annoying'.
\end{itemize} The clearest cases of singular terms are proper names, and these occupy a distinct syntactic category in \FOL. Most of the other types of English singular terms are modelled in more or less indirect ways in \FOL.

Definites and possessives are handled principally by paraphrase. We treat possessives as disguised definite descriptions, paraphrasing `Antony's eldest child' as something like `the eldest child of Antony'. In turn, definites are handled as complex constructions in \FOL, as we'll see when we take them up in §\ref{subsec.defdesc}.

Even trickier in some ways are singular term uses of pronouns and demonstratives. Both of these constructions rely on the \define{conversational context} to fix a determinate reference. I might need to gesture, or rely on our previous utterances, to understand who `she' refers to in `she plays violin'. Likewise, `that loud dog' might vary in which dog it refers to from conversation to conversation. There are approaches that attempt to model the role of context in determining the meaning of an expression, but we will not attempt to do this here.  \FOL\ is not designed to model every aspect of English. If we are forced to try to represent some English sentences involving context-sensitive singular terms in \FOL, we will need to resort to paraphrases that are not fully adequate (e.g., treating demonstratives as definite descriptions, despite their differences).

Moreover, pronouns are not singular terms in every use. Compare the uses of the pronoun `she' in `She plays violin' and `Every girl thinks she deserves icecream'. The first refers to a specific individual, perhaps with some contextual cues like gestures to help identify which. The second, however, doesn't refer to a specific girl – rather, it ranges over all girls. Perhaps surprisingly, \FOL\ takes this second kind of use of pronouns to be primary. We will discuss how \FOL\ represents pronouns when we introduce variables in §\ref{quant.pron}.

% section singular_terms (end)

\section{Names}\label{s:names}

\define{Proper names} are a particularly important kind of singular term. These are expressions that label individuals without describing them. The name `Emerson' is a proper name, and the name alone does not tell you anything about Emerson. Of course, some names are traditionally given to boys and other are traditionally given to girls. If `Hilary' is used as a singular term, you might guess that it refers to a woman. You might, though, be guessing wrongly. Indeed, the name does not necessarily mean that what is referred to is even a person: Hilary might be a giraffe, for all you could tell just from the name `Hilary'. In English, the use of certain names triggers our knowledge of these conventions, so that (for example) the use of name `Fido' might well trigger an expectation that the thing named is a dog. However, while it would violate convention to name your child `Fido', once the you manage to assign the name it can perfectly well refer to a human child.  

In \FOL, there are no conventions around its category of \define{names}. These are pure names, whose only role is to designate some specific individual. Names in \FOL\ are represented by lower-case letters `$a$' through to `$r$'. We can add subscripts if we want to use some letter more than once, if we have a complicated discourse with many different names. So here are some names in \FOL:
	\[a,b,c,…, r, a_{1}, f_{32}, j_{390}, m_{12}.\]
These should be thought of along the lines of proper names in English. But with some differences. First, `Antony Eagle' is a proper name, but there are a number of people with this name. (Equally, there are at least two people with the name `P.D.\ Magnus' and several people named `Tim Button'.) We live with this kind of ambiguity in English, allowing context to determine that some particular utterance of the name `Antony Eagle' refers to one of the contributors to this book, and not some other guy. In \FOL, we do not tolerate any such ambiguity. Each name must pick out \emph{exactly} one thing. (However, two different names may pick out the same thing.)  Second, names (and predicates) in \FOL\ are assigned their meaning, or \emph{interpreted}, only temporarily. (This is just like the way that atomic sentences are only assigned a truth value in \TFL\ temporarily, relative to a valuation.)

As with \TFL, we provide symbolisation keys. These indicate, temporarily, what a name shall pick out. So we might offer:
	\begin{ekey}
		\item[e] Elsa
		\item[g] Gregor
		\item[m] Marybeth
	\end{ekey}

Again, what we are saying in using natural language names is that the thing referred to by the \FOL\ name `$e$' is stipulated – for now – to be the same thing referred to by the English proper noun `Elsa'. We are \emph{not} saying that `$e$' and `Elsa' are synonyms. For all we know, perhaps there is additional nuance to the meaning of the name `Elsa' other than what it refers to. If there is, it is not preserved in the \FOL\ name `$e$', which has no nuance in its meaning other than the thing it denotes. The \FOL\ name `$e$' might be stipulated to denote Elsa on one occasion, and Eddie on another. 

You may have been taught in school that a noun is a `naming word'. It is safe to use names in \FOL\ to symbolise proper nouns. But dealing with common nouns  is a more subtle matter. Even though they are often described as `general names', common nouns actually function as names relatively rarely. Some common nouns which do often function as names are \define{natural kind terms} like `gold' and `tiger'. These are common nouns which really do name a kind of thing. Consider this example: \begin{earg}
	\item[] Gold is scarce;
	\item[] Nothing scarce is cheap;
	\item[So:] Gold isn't cheap.
\end{earg} This argument is valid in virtue of its form. The form of the first premise has the phrase `\blank\ is scarce' being predicated of `gold', in which case, `gold' must be functioning as a name in this argument. But notice that `\blank\ is gold' is a perfectly good predicate. So we cannot simply treat all natural kind terms as proper names of general kinds.

\section{Predicates}
The simplest predicates denote \emph{properties} of individuals. They are things you can say about the features or behaviour of an object. Here are some examples of English predicates:
	\begin{itemize}
		\item \blank\ runs
		\item \blank\ is a dog
		\item \blank\ is a member of Monty Python
		\item \blank\ was a student of David Lewis 
		\item  A piano fell on \blank
	\end{itemize}
In general, you can think about predicates as things which combine with singular terms to make sentences. In these cases, they are interpreted with the aid of natural language \define{verb phrases}. The most elementary phrases that correspond to predicates are simple intransitive verb phrases like `runs'. But predicates can symbolise more complex verb phrases. In a subject-predicate sentence, we can treat any syntactic sub-unit of a sentence including everything but the subject as a predicate. So `Bertie is a dog' can be seen as involving the name `Bertie' and the predicate `is a dog'. Note that proper names can occur within a predicate, as in the verb phrase `was a student of David Lewis'. (We will see how to model sentences where multiple names interact with a complex predicate below in §\ref{s:MultipleGenerality}.) You can begin with sentences and make predicates out of them by removing singular terms and leaving `slots' in which singular terms can be placed. Consider the sentence, `Vinnie borrowed the family car from Nunzio'. By removing one singular term, we can obtain any of three different predicates:
	\begin{itemize}
		\item \blank\ borrowed the family car from Nunzio
		\item Vinnie borrowed \blank\ from Nunzio
		\item Vinnie borrowed the family car from \blank
	\end{itemize} (What if we wanted to remove two or more singular terms and leave more than one gap? We shall return to this in §\ref{s:MultipleGenerality}.)
\FOL\ predicates are capital letters $A$ through $Z$, with or without subscripts. We might write a symbolisation key for predicates thus:
	\begin{ekey}
		\item[A] \blank\ is angry
		\item[H] \blank\ is happy
	\end{ekey}


If we combine our two symbolisation keys, we can start to symbolise some English sentences that use these names and predicates in combination. For example, consider the English sentences:
	\begin{earg}
		\item[\ex{terms1}] Elsa is angry.
		\item[\ex{terms2a}] Gregor and Marybeth are angry.
		\item[\ex{terms2}] If Elsa is angry, then so are Gregor and Marybeth.
	\end{earg}
To make a simple subject-predicate sentence, we need to couple the predicate with as many names as there are `gaps'. Since in the above key `$A$' has one gap in the symbolisation key, following it with a single name makes a grammatically formed sentence of \FOL. So sentence \ref{terms1} is straightforward: we symbolise it by `$Ae$'.

Sentence \ref{terms2a}: this is a conjunction of two simpler sentences. The simple sentences can be symbolised just by `$Ag$' and `$Am$'. Then we help ourselves to our resources from \TFL, and symbolise the entire sentence by `$Ag \eand Am$'. This illustrates an important point: \FOL\ has all of the truth-functional connectives of \TFL.

Sentence \ref{terms2}: this is a conditional, whose antecedent is sentence \ref{terms1} and whose consequent is sentence \ref{terms2a}. So we can symbolise this with `$Ae \eif (Ag \eand Am)$'.

\subsection{Predicates without subjects} \label{s:zpp}

Actually we can make an elegant further assumption to incorporate \TFL\ entirely within \FOL. A predicate combines with singular terms to make sentences. But some English predicates, though they require a singular term syntactically, don't seem to involve the referent of those singular terms in their meaning. We see this in expressions like these: \begin{earg}
	\item[\ex{rains}] It is snowing;
	\item[\ex{seemss}] It seems that George is hungry;
	\item[\ex{shell}] She'll be right.
\end{earg} The pronouns in these cases are known as \define{dummy pronouns}. These pronouns have no obvious referent, but that poses no problem for the interpretation of the sentences. Example \ref{rains} means something like the bare verb `snowfalling', if only that were grammatical. In these cases, the predicate following the pronoun does not denote a property of individuals, because there is no way to attach it to a particular individual.  Contrast these examples:  \begin{earg}
    \item[\ex{cpa}] Coober Pedy is a harsh place. \underline{It} has underground houses.
    	\item[\ex{cpb}] Coober Pedy is a harsh place. \underline{It} is hot.
    \end{earg}
In \ref{cpa}, `it' refers to Coober Pedy. Not so in \ref{cpb}: `It' in that example is a \define{dummy pronoun}. one which needs to be present for syntactic reasons (every English sentence requires a grammatical subject), but which doesn't make any contribution to the meaning of the sentence. (Try substituting some proper name for `it' in `it is hot' and see what nonsense you get. ) They are common with metereological reports: `it is raining', `it is dark', etc. 

\FOL does not have the syntactic limitation of English that every sentence must include a singular term. We will allow \FOL\ predicates occuring by themselves to count as grammatical sentences of \FOL. These zero-place predicates, semantically requiring no subject, are to be symbolised by capital letters `$A$' through `$Z$' (perhaps subscripted), but without needing any adjacent name to be grammatical.

Note we have thereby included all atomic sentences of \TFL\ in \FOL\ among these special predicates of \FOL. In \TFL\ we used these sentences to symbolise any sentence of English which did not include sentence connectives. In \FOL\ their use will be more limited. But nevertheless, syntactically, since \FOL\ has all the connectives of \TFL\ and all the atomic sentences, we see that \emph{every sentence of \textnormal{\TFL} is also a sentence of \textnormal{\FOL}}.

Indeed, this seems obligatory: the standard way to symbolise a bare pronoun sentence will make use of a free variable (§\ref{s:fol}). An English sentence like `He is tall' doesn't manage to express a meaningful claim in the absence of some context providing a referent for `he'. So it might be symbolised `$Tx$' – a formula of \FOL\ that is not a sentence – to indicate that we are not expecting this sentence that is being symbolised to be true or false in a given intepretation. To symbolise `it is hot' as `$Hx$' would on these grounds be a mistake: `it is hot' expresses a proposition even without a referent assigned to the pronoun `it', and is true or false in a situation just depending on whether heat is present or not. So it can be adequately symbolised by a zero-place predicate `$H$'.


 

\section{Quantifiers}\label{quant.pron}
We are now ready to introduce quantifiers. In general, a quantifier tells us \emph{how many}. Consider these sentences:
	\begin{earg}
		\item[\ex{q.a}] Everyone is happy.
%		\item[\ex{q.ac}] Everyone is at least as tough as Elsa.
		\item[\ex{q.e}] Someone is angry.
		\item[\ex{q.eg}] Every girl thinks she deserves icecream.
		\item[\ex{q.most}] Most people are happy.
		\item[\ex{q.two}] Exactly two people are angry.
		\item[\ex{q.more}] More than three people are happy.
	\end{earg}
We will focus initially on the coarse-grained quantifiers `every'/`all' and `some'. We will look at numerical quantifiers, as in examples \ref{q.two} and \ref{q.more}, in §\ref{sec.identity}.

It might be tempting to symbolise sentence \ref{q.a} as `$(He \eand (Hg \eand Hm))$'. Yet this would only say that Elsa, Gregor, and Marybeth are happy. We want to say that \emph{everyone} is happy, even those we have not named, even those who are nameless. %In order to do this, we introduce the `$\forall$' symbol. This is called the \define{universal quantifier}.


Note that \ref{q.a} and \ref{q.e} and \ref{q.eg} can be roughly paraphrased like this: \begin{earg}
	\item[\ex{q.athey}] Every person is such that: they are happy.
	\item[\ex{q.ethey}] Some person is such that: they are angry.
	\item[\ex{q.eshe}] Every girl is such that: she thinks she herself deserves icecream.
\end{earg} In each of these, we have a pronoun – singular `they' in \ref{q.athey} and \ref{q.ethey}, `she' in \ref{q.eshe} – which is governed by the preceding phrase. That phrase gives us information about what this pronoun is pointing to – is it pointing severally to everyone, as in \ref{q.athey}? Or to just someone, though it is generally unknown which particular person it is, as in \ref{q.ethey}? In either case, something general is being said, rather than something specific, even in example \ref{q.ethey} which is true just in case there is at least one angry person – it doesn't matter which person it is. 

In this sort of construction, the sentences `they are happy' and `she thinks she herself deserves icecream', which are headed by a bare pronoun, are called \define{open sentences}. An open sentence in English can be used to say something meaningful, if the circumstances permit a unique interpretation of the pronoun – consider `she plays violin' from §\ref{s:names}. But in many cases no such unique interpretation is possible. If I gesture at a large crowd and say simply `he is angry', I may not manage to say anything meaningful if there is no way to establish which person this use of `he' is pointing to. The other part of the sentence, the `every person is such that …' part, is called a \define{quantifier phrase}. The quantifier phrase gives us guidance about how to interpret the otherwise bare pronoun. 

The treatment of quantifier phrases in \FOL\ actually follows the structure of these paraphrases rather well. The \FOL\ analogue of these embedded pronouns is the category of \define{variable}. In \FOL, variables are italic lower case letters `$s$' through `$z$', with or without subscripts. They combine with predicates to form open sentences of the form `$\meta{A}\meta{x}$'. Grammatically variables are thus like singular terms. However, as their name suggests, variables do not denote any fixed individual. They will not be assigned a meaning by a symbolisation key, even temporarily. Rather, their role is to be governed by an accompanying quantifier phrase to say something general about a situation. In \FOL, an open sentence combines with a quantifier to form a sentence. (Notice that I have here returned to the practice of using `$\meta{A}$' as a metavariable, from §\ref{s:UseMention}.)

\paragraph{Universal Quantifier} The first quantifier from \FOL\ we meet is the \define{universal quantifier}, symbolised `$\forall$', and which corresponds to `every'. Unlike English, we always follow a quantifier in \FOL\ by the variable it governs, to avoid the possibility of confusion. Putting this all together, we might symbolise sentence \ref{q.a} as `$\forall x Hx$'.  The variable `$x$' is serving as a kind of placeholder, playing the role that is allotted to the pronoun in the English paraphrase \ref{q.athey}. The expression `$\forall x$' intuitively means that you can pick anyone to be temporarily denoted by `$x$'. The subsequent `$Hx$' indicates, of that thing you picked out, that \emph{it} is happy. (Note that pronoun again.)

I should say that there is no special reason to use `$x$' rather than some other variable. The sentences `$\forall x Hx$', `$\forall y Hy$', `$\forall z Hz$', and `$\forall x_5 Hx_5$' use different variables, but they will all be logically equivalent.

\paragraph{Existential quantifier} To symbolise sentence \ref{q.e}, we introduce a second quantifier: the \define{existential quantifier}, `$\exists$'. Like the universal quantifier, the existential quantifier requires a variable. Sentence \ref{q.e} can be symbolised by `$\exists x Ax$'. Whereas `$\forall x Ax$' is read naturally as `for all $x$, it ($x$) is angry', `$\exists x Ax$' is read naturally as `there is something, $x$, such that it ($x$) is angry'. Once again, the variable is a kind of placeholder; we could just as easily have symbolised sentence \ref{q.e} with `$\exists z Az$', `$\exists w_{256} Aw_{256}$', or whatever.

Some more examples will help. Consider these further sentences:
	\begin{earg}
		\item[\ex{q.ne}] No one is angry.
		\item[\ex{q.en}] There is someone who is not happy.
		\item[\ex{q.na}] Not everyone is happy.
	\end{earg}
Sentence \ref{q.ne} can be paraphrased as, `It is not the case that someone is angry'. We can then symbolise it using negation and an existential quantifier: `$\enot \exists x Ax$'. Yet sentence \ref{q.ne} could also be paraphrased as, `Everyone is not angry'. With this in mind, it can be symbolised using negation and a universal quantifier: `$\forall x \enot Ax$'. Both of these are acceptable symbolisations.  Indeed, it will transpire that, in general, $\forall x \enot\meta{A}$ is logically equivalent to $\enot\exists x\meta{A}$.  Symbolising a sentence one way, rather than the other, might seem more `natural' in some contexts, but it is not much more than a matter of taste.

Sentence \ref{q.en} is most naturally paraphrased as, `There is some x, such that x is not happy'. This then becomes `$\exists x \enot Hx$'. Of course, we could equally have written `$\enot\forall x Hx$', which we would naturally read as `it is not the case that everyone is happy'. And that would be a perfectly adequate symbolisation of sentence \ref{q.na}.

Quantifiers get their name because they tell us how many things have a certain feature. \FOL\ allows only very crude distinctions: we have seen that we can symbolise `no one', `someone', and `everyone'. English has many other quantifier phrases: `most', `a few', `more than half', `at least three', etc. Some can be handled in a roundabout way in \FOL, as we will see: the numerical quantifier `at least three', for example, we will meet again in §\ref{sec.identity}. But others, like `most', are simply unable to be reliably symbolised in \FOL. 

\section{Domains}\label{sec_domains}
Given the symbolisation key we have been using, `$\forall xHx$' symbolises `Everyone is happy'.  Who is included in this \emph{everyone}? When we use sentences like this in English, we usually do not mean everyone now alive on the Earth. We almost certainly do not mean everyone who was ever alive or who will ever live. We usually mean something more modest: everyone now in the building, everyone enrolled in the ballet class, or whatever.

In order to eliminate this ambiguity, we will need to specify a \define{domain}. The domain is just the things that we are talking about. So if we want to talk about people in Chicago, we define the domain to be people in Chicago. We write this at the beginning of the symbolisation key, like this:
	\begin{ekey}
		\item[\domain] people in Chicago
	\end{ekey}
The quantifiers \emph{range over} the domain. Given this domain, `$\forall x$' is to be read roughly as `Every person in Chicago is such that …' and `$\exists x$' is to be read roughly as `Some person in Chicago is such that …'. 

In \FOL, the domain must always include at least one thing. Moreover, in English we can conclude `something is angry' when given `Gregor is angry'. In \FOL, then, we shall want to be able to infer `$\exists x Ax$' from `$Ag$'. So we shall insist that each name must pick out exactly one thing in the domain. If we want to name people in places beside Chicago, then we need to include those people in the domain. 

In permitting multiple domains, \FOL\ follows the lead of natural languages like English. Consider an argument like this: \begin{earg}
	\item[\ex{ex.qdr}] All the beer has been drunk; so we're going to the bottle-o. 
\end{earg} The premise says that all the beer is gone. But the conclusion only makes sense if there is more beer at the bottle shop. So whatever domain of things we are talking about when we state the premise, it cannot include absolutely \emph{everything}. In \FOL, we sidestep the interesting issues involved in deciding just what domain is involved in evaluating sentences like `all the beer has been drunk', and explicitly include the current domain of quantification in our symbolisation key.

Note further that to make sense of the sentence `all the beer has been drunk', the domain will have to contain both past and present things, so we can understand what we are saying about the now-absent beer. A domain contains \emph{what we are talking about}. It might be difficult to understand how we do it, but we do talk about past things, fictional things, abstract things, merely possible things, and other unusual entities. So our domains must be flexible enough to include any of these things we might be talking about. It is a question in philosophical logic as to how we can explain how we manage to include nonexistent things in our domain of discourse, but for the purposes of \FOL\ all we need to know is that this is something we somehow manage to do.

\factoidbox{
		A domain must have \emph{at least} one member. A name must pick out \emph{exactly} one member of the domain. But a member of the domain may be picked out by one name, many names, or none at all. The domain can consist of anything we might be discussing; it is not restricted to things that presently exist.}


\keyideas{
	\item \FOL\ gives us resources for modelling some aspects of sub-sentential structure in natural languages: names, predicates, and some quantifier phrases.
	\item The names of \FOL\ are expressions that are used to refer to objects in some circumscribed domain; they can often have their referents temporarily set by associating them with natural language proper names in a symbolisation key.
	\item The predicates of \FOL\ are expressions that denote properties of objects; they can often have their referents temporarily set by associating them with natural language verb phrases in a symbolisation key.
	\item The quantifiers of \FOL\ have a fixed meaning not set by a symbolisation key. They tell us how many things in the domain meet a certain predicate. In \FOL, we concentrate on two quantifiers: the universal `$\forall$' (`every') and existential `$\exists$' (`some').
	\item Quantifiers are supported by variables, which behave rather like pronouns in English.
}

\practiceproblems

\problempart In each of the following sentences, identify the names and predicates. Comment on any difficulties.\begin{enumerate}
	\item Kurt and Alonzo are logicians;
	\item Silver is useful in film photography;
	\item Mary's ring is silver;
	\item Kurt and Alonzo lift a couch;
	\item The biggest threat to life on earth is carbon dioxide.
\end{enumerate}

\problempart Identify the possible predicates that can be found by replacing singular terms with gaps in these sentences: \begin{enumerate}
	\item He dislikes Joel;
	\item Andrew Leigh was a professor of economics at the ANU;
	\item The professor of genomics was elected president of the Academy.
\end{enumerate}


\problempart
Make use of this symbolisation key to symbolise the following sentences into Quantifer, commenting on any difficulties:
\begin{ekey}
		\item[\domain] cities and towns in South Australia
		\item[a] Adelaide
		\item[m] Mount Gambier
		\item[T] \blank\ is a town
		\item[U] \blank\ is ugly
		\item[L] \blank\ is large
	\end{ekey}

\begin{enumerate}
	\item Adelaide is big and ugly.
	\item Mount Gambier is not large.
	\item If Mount Gambier is large, then Adelaide definitely is!
	\item Mount Gambier is a large village.
	\item Every city and town is ugly.
\end{enumerate}

\problempart Can this argument be adequately symbolised in \FOL? Comment on any difficulties. \begin{earg}
	\item She is tall;
	\item Bob likes her;
	\item[So:] Bob likes someone tall. 
\end{earg}

\chapter{Sentences with One Quantifier}\label{s:MoreMonadic}
We now have the basic pieces of \FOL. Symbolising many sentences of English will only be a matter of knowing the right way to combine predicates, names, quantifiers, and the truth-functional connectives. There is a knack to this, and there is no substitute for practice.


\section{Common Quantifier Phrases}
As in \TFL\ (recall §\ref{s:TFLConnectives}), we will give canonical symbolisations for certain common English quantificational structures.
Consider these sentences:
	\begin{earg}
		\item[\ex{quan1}] Every coin in my pocket is a 20¢ piece.
		\item[\ex{quan2}] Some coin on the table is a dollar.
		\item[\ex{quan3}] Not all the coins on the table are dollars.
		\item[\ex{quan4}] None of the coins in my pocket are dollars.
	\end{earg}
In providing a symbolisation key, we need to specify a domain. Since we are talking about coins in my pocket and on the table, the domain must at least contain all of those coins. Since we are not talking about anything besides coins, we let the domain be all coins. Since we are not talking about any specific coins, we do not need to need to deal with any names. So here is our key:
	\begin{ekey}
		\item[\domain] all coins
		\item[P] \blank\ is in my pocket
		\item[T] \blank\ is on the table
		\item[Q] \blank\ is a 20¢ piece
		\item[D] \blank\ is a dollar
	\end{ekey}
Sentence \ref{quan1} is most naturally symbolised using a universal quantifier. The universal quantifier says something about everything in the domain, not just about the coins in my pocket. So if we want to talk just about coins in my pocket, we will need to \emph{restrict} the quantifier, by imposing a condition on the things we are saying are 20¢ pieces. That is: something in the domain is claimed to be a 20¢ piece \emph{only if} it meets the restricting condition. That leads us to this conditional paraphrase: \begin{earg}
	\item[\ex{coinpa}] For any (coin): $\underbracket{\text{\emph{if} that coin is in my pocket}}_{\text{\emph{restriction}}}$, \emph{then} it is a 20¢ piece.
\end{earg} So we can symbolise it as `$\forall x(Px \eif Qx)$'.

Since sentence \ref{quan1} is about coins that are both in my pocket \emph{and} that are 20¢ pieces, it might be tempting to translate it using a conjunction. However, the sentence `$\forall x(Px \eand Qx)$' would symbolise the sentence `every coin is both a 20¢ piece and in my pocket'. This obviously means something very different than sentence \ref{quan1}. And so we see:
	\factoidbox{
		A sentence can be symbolised as $\forall x (\meta{F}x \eif \meta{G}x)$ if it can be paraphrased in English as `every F is G' or `all Fs are Gs'.}


Example \ref{quan2} uses the quantifier phrase `some'. The same thought could be expressed using different quantifier phrases: \begin{earg}
	\item[\ex{quan2a}] \underline{At least one} coin on the table is a dollar.
	\item[\ex{quan2b}] \underline{There is} a coin on the table that is a dollar.
\end{earg} These phrases all indicate an existential quantifier. In these examples, the class of coins on the table is being related to the class of dollar coins, and it is claimed that at least one member of the former class is also in the latter class – that there is \emph{overlap}. This is represented in \FOL\ following the example of this paraphrase:
\begin{earg}
	\item[\ex{coindol}] There is something (a coin): it is in \emph{both} the class of things on the table, \emph{and} in the class of dollar coins.
\end{earg} That is symbolised using a conjunction, `$\exists x (Tx \eand Dx)$'. 

We know from \TFL\ that the order of conjuncts doesn't matter: `$(P \eand Q)$' is logically equivalent to `$(Q\eand P)$'. Likewise in \FOL, `$\exists x (Tx \eand Dx)$' is logically equivalent to `$\exists x (Dx \eand Tx)$'. This fits well with the English sentences we are symbolising, because overlap is itself a symmetrical relation between classes. We see this also in the fact that we can paraphrase example \ref{quan2} as `Some dollar coin is on the table'.

Notice that we needed to use a conditional with the universal quantifier, but we used a conjunction with the existential quantifier. Suppose we had instead written `$\exists x(Tx \eif Dx)$'. That would mean that there is some object in the domain such that  \emph{if} it is $T$, then it is also $D$. For this to be true, we just need something to not be $T$. So it is \emph{very easy} for `$\exists x(Tx \eif Dx)$' to be true. Given our symbolisation, it will be true if some coin is not on the table. Of course there is a coin that is not on the table: there are coins lots of other places.  That is rather less demanding than the claim that something is both $T$ and $D$.

A conditional will usually be the natural connective to use with a universal quantifier, but a conditional within the scope of an existential quantifier tends to say something very weak indeed. As a general rule of thumb, do not put conditionals in the scope of existential quantifiers unless you are sure that you need one.
	\factoidbox{
		A sentence can be symbolised as $\exists x (\meta{F}x \eand \meta{G}x)$ if it can be paraphrased in English as `some F is G', `at least one F is G', or `There is an F which is G'.
	}		
Sentence \ref{quan3} can be paraphrased as, `It is not the case that every coin on the table is a dollar'. So we can symbolise it by `$\enot \forall x(Tx \eif Dx)$'. You might look at sentence \ref{quan3} and paraphrase it instead as, `Some coin on the table is not a dollar'. You would then symbolise it by `$\exists x(Tx \eand \enot Dx)$'. Although it is probably not immediately obvious yet, these two sentences are logically equivalent. (This is due to the logical equivalence between $\enot\forall x\meta{A}$ and $\exists x\enot\meta{A}$, mentioned in §\ref{s:FOLBuildingBlocks}, along with the logical equivalence between $\enot(\meta{A}\eif\meta{B})$ and $\meta{A}\eand\enot\meta{B}$.)

Sentence \ref{quan4} can be paraphrased as, `It is not the case that there is some dollar in my pocket'. This can be symbolised by `$\enot\exists x(Px \eand Dx)$'. It might also be paraphrased as, `Everything in my pocket is a nondollar', and then could be symbolised by `$\forall x(Px \eif \enot Dx)$'. Again the two symbolisations are logically equivalent. Both are correct symbolisations of sentence \ref{quan4}.


\section{Empty Predicates}
In §\ref{s:FOLBuildingBlocks}, I emphasised that a name must pick out exactly one object in the domain. However, a predicate need not apply to anything in the domain. A predicate that applies to nothing in a domain is called an \define{empty} predicate (relative to that domain). This is worth exploring.

Suppose we want to symbolise these two sentences:
	\begin{earg}
		\item[\ex{monkey1}] Every monkey knows sign language
		\item[\ex{monkey2}] Some monkey knows sign language
	\end{earg}
It is possible to write the symbolisation key for these sentences in this way:
	\begin{ekey}
		\item[\domain] animals
		\item[M] \blank\ is a monkey.
		\item[S] \blank\ knows sign language.
	\end{ekey}
Sentence \ref{monkey1} can now be symbolised by `$\forall x(Mx \eif Sx)$'. Sentence \ref{monkey2} can be symbolised as `$\exists x(Mx \eand Sx)$'.

It is tempting to say that sentence \ref{monkey1} \emph{entails} sentence \ref{monkey2}. That is, we might think that it is impossible for it to be the case that every monkey knows sign language, without it's also being the case that some monkey knows sign language.  But this would be a mistake. It is possible for the sentence `$\forall x(Mx \eif Sx)$' to be true even though the sentence `$\exists x(Mx \eand Sx)$' is false.

How can this be? The answer comes from considering whether these sentences would be true or false \emph{if there are no monkeys}. If there are no monkeys at all (in some domain), then `$\forall x(Mx \eif Sx)$' would be \emph{vacuously} true. Take the domain of reptiles. Look at the domain, and pick any monkey you like – it knows sign language!\footnote{Remember this is not a counterfactual claim (\ref{s:IndicativeSubjunctive}); even if `All monkeys know sign language' is vacuously true in the domain of reptiles, that wouldn't mean that `Any monkey \emph{would} know sign language' is true.} There is certainly no counterexample to the claim available in this domain. And because of the role of the conditional in our symbolisation, it turns out that a universally quantified claim with an unsatisfied restricting condition will also be true. In \FOL, a universally quantified sentence of the form $\forall \meta{x} (\meta{Ax} \eif \meta{Bx})$ is \emph{false} only if we can find something which is \meta{A} without being \meta{B}. If we can't find such a thing, perhaps because we can't find anything which is \meta{A} in the first place, then the sentence will be true (since truth is just lack of falsity, and this sentence isn't false because we can't find a case that falsifies it). This derives ultimately from the feature of \TFL\ we have already acknowledged to be questionably analogous to English, namely, the fact that a conditional is false only if there is a counterexample, a case where the antecedent is true and the consequent false.

Another example will help to bring this home. Suppose we extend the above symbolisation key, by adding:
	\begin{ekey}
		\item[R] \blank\ is a refrigerator
	\end{ekey}
Now consider the sentence `$\forall x(Rx \eif Mx)$'. This symbolises `every refrigerator is a monkey'. And this sentence is true, given our symbolisation key. This is counterintuitive, since we do not want to say that there are a whole bunch of refrigerator monkeys. It is important to remember, though, that `$\forall x(Rx \eif Mx)$' is true iff any member of the domain that is a refrigerator is a monkey. Since the domain is \emph{animals}, there are no refrigerators in the domain. Again, then, the sentence is \emph{vacuously} true. 

If you were actually dealing with the sentence `All refrigerators are monkeys', then you would most likely want to include kitchen appliances in the domain. Then the predicate `$R$' would not be empty and the sentence `$\forall x(Rx \eif Mx)$' would be false. Remember, though, that a predicate is empty only \emph{relative} to a particular domain. 
	\factoidbox{
		When $\meta{F}$ is an empty predicate relative to a given domain, a sentence $\forall x (\meta{F}x \eif …)$ will be vacuously true of that domain.
	} 

\section{Picking a Domain}


The appropriate symbolisation of an English language sentence in \FOL\ will depend on the symbolisation key. Choosing a key can be difficult. Suppose we want to symbolise the English sentence:
	\begin{earg}
		\item[\ex{pickdomainrose}] Every rose has a thorn.
	\end{earg}
We might offer this symbolisation key:
	\begin{ekey}
		\item[R] \blank\ is a rose
		\item[T] \blank\ has a thorn
	\end{ekey}
It is tempting to say that sentence \ref{pickdomainrose} should be symbolised as `$\forall x(Rx \eif Tx)$'. But we have not yet chosen a domain. If the domain contains all roses, this would be a good symbolisation. Yet if the domain is merely \emph{things on my kitchen table}, then `$\forall x(Rx \eif Tx)$' would only come close to covering the fact that every rose \emph{on my kitchen table} has a thorn. If there are no roses on my kitchen table, the sentence would be trivially true. This is not what we want. To symbolise sentence \ref{pickdomainrose} adequately, we need to include all the roses in the domain. But now we have two options. 

First, we can restrict the domain to include all roses but \emph{only} roses. Then sentence \ref{pickdomainrose} can, if we like, be symbolised with `$\forall x Tx$'. This is true iff everything in the domain has a thorn; since the domain is just the roses, this is true iff every rose has a thorn. By restricting the domain, we have been able to symbolise our English sentence with a very short sentence of \FOL. So this approach can save us trouble, if every sentence that we want to deal with is about roses.

Second, we can let the domain contain things besides roses: rhododendrons; rats; rifles; whatevers.  And we will certainly need to include a more expansive domain if we simultaneously want to symbolise sentences like:
	\begin{earg}
		\item[\ex{pickdomaincowboy}] Every cowboy sings a sad, sad song.
	\end{earg}
Our domain must now include both all the roses (so that we can symbolise sentence \ref{pickdomainrose}) and all the cowboys (so that we can symbolise sentence \ref{pickdomaincowboy}). So we might offer the following symbolisation key:\label{poison}
	\begin{ekey}
		\item[\domain] people and plants
		\item[C] \blank\ is a cowboy
		\item[S] \blank\ sings a sad, sad song
		\item[R] \blank\ is a rose
		\item[T] \blank\ has a thorn
	\end{ekey}
Now we will have to symbolise sentence \ref{pickdomainrose} with `$\forall x (Rx \eif Tx)$', since `$\forall x Tx$' would symbolise the sentence `every person or plant has a thorn'. Similarly, we will have to symbolise sentence \ref{pickdomaincowboy} with `$\forall x (Cx \eif Sx)$'. 

In general, the universal quantifier can be used to symbolise the English expression `everyone' if the domain only contains people. If there are people and other things in the domain, then `everyone' must be treated as `every person'.

If you choose a narrow domain, you can make the task of symbolisation easier. If we are attempting to symbolise example \ref{q.eshe}, `every girl thinks that she deserves icecream', we can pick the domain to be girls and then we only need to introduce a predicate `\blank\ thinks that they themselves deserve icecream', symbolised `$D$'. The symbolisation is then the simple `$\forall x Dx$'. But our options are limited if the conversation goes on to talk about things other than girls. On the other hand, if you pick an expansive domain (such as everything whatsoever), you can always just impose an appropriate restriction. In this case, we could introduce the predicate `$G$' to stand for `\blank\ is a girl', and symbolise the sentence as `$\forall x (Gx \eif Dx)$'. 

When choosing an expansive domain, you must take some care with implicitly restricted predicates. It would not be appropriate to paraphrase `all the beer has been drunk' relative to a very expansive domain as `For everything: if it is beer, it has been drunk'. To adequately capture the intent, we shall need to make the implicit contextual restriction of the predicate explicit, in something like this paraphrase `For everything: if it is beer at the party, it has been drunk'.


\section{The Utility of Paraphrase}
When symbolising English sentences in \FOL, it is important to understand the structure of the sentences you want to symbolise. What matters is the final symbolisation in \FOL, and sometimes you will be able to move from an English language sentence directly to a sentence of \FOL. Other times, it helps to paraphrase the sentence one or more times. Each successive paraphrase should move from the original sentence closer to something that you can finally symbolise directly in \FOL.

For the next several examples, we will use this symbolisation key:
	\begin{ekey}
		\item[\domain] women
		\item[B] \blank\ is a bassist.
		\item[R] \blank\ is a rock star.
		\item[k] Kim Deal
	\end{ekey}
Now consider these sentences:
	\begin{earg}
		\item[\ex{pronoun1}] If Kim Deal is a bassist, then she is a rock star.
		\item[\ex{pronoun2}] If any woman is a bassist, then she is a rock star.
	\end{earg}
The same words appear as the consequent in sentences \ref{pronoun1} and \ref{pronoun2} (`$…$ she is a rock star'), but they mean very different things (recall §\ref{quant.pron}). To make this clear, it often helps to paraphrase the original sentences into a more unusual but clearer form.

Sentence \ref{pronoun1} can be paraphrased as, `Consider Kim Deal: if she is a bassist, then she is a rockstar'. The bare pronoun `she' gets to denote Kim Deal because of our initial `Consider Kim Deal' remark. This then says something about one particular person, and can obviously be symbolised as `$Bk \eif Rk$'.

Sentence \ref{pronoun2} gets a very similar paraphrase, with the same embedded conditional: `Consider any woman: if she is a bassist, then she is a rockstar'. The difference in the `Consider …' phrase however forces a very different intepretation for the sentence as a whole. Replacing the English pronouns by variables, the \FOL\ equivalent of a pronoun, we get this awkward quasi-English paraphrase: `For any woman x, if x is a bassist, then x is a rockstar'. Now this can be symbolised as `$\forall x (Bx \eif Rx)$'. This is the same sentence we would have used to symbolise `Every woman who is a bassist is a rock star'. And on reflection, that is surely true iff sentence \ref{pronoun2} is true, as we would hope.

Consider these further sentences, and let us consider the same interpretation as above, though in a domain of all people.
	\begin{earg}
		\item[\ex{anyone1}] If anyone is a bassist, then Kim Deal is a rock star.
		\item[\ex{anyone2}] If anyone is a bassist, then they are a rock star.
	\end{earg}
The same words appear as the antecedent in sentences \ref{anyone1} and \ref{anyone2}  (`If anyone is a bassist$…$'). But it can be tricky to work out how to symbolise these two uses. Again, paraphrase will come to our aid. 

Sentence \ref{anyone1} can be paraphrased, `If there is at least one bassist, then Kim Deal is a rock star'. It is now clear that this is a conditional whose antecedent is a quantified expression; so we can symbolise the entire sentence with a conditional as the main connective: `$\exists x Bx \eif Rk$'. 

Sentence \ref{anyone2} can be paraphrased, `For all people x, if x is a bassist, then x is a rock star'. Or, in more natural English, it can be paraphrased by `All bassists are rock stars'. It is best symbolised as `$\forall x(Bx \eif Rx)$', just like sentence \ref{pronoun2}.

The word `any' is particularly tricky, because it can sometimes mean `every' and sometimes `at least one'! Think about the two occurrences of `any' in this sentence: \begin{earg}
\item[\ex{student.any}] Any student will be happy if they have any money.\\
 For every student: if there exists some money they possess, then they will  be happy.
\end{earg} This can be symbolised `$\forall x (Sx \eif (\exists y (My \wedge Pxy) \eif Hx))$', where the first occurrence of `any' is represented by a universal quantifier, and the second by an existential quantifier.  

The moral is that the English words `any' and `anyone' should typically be symbolised using quantifiers. And if you are having a hard time determining whether to use an existential or a universal quantifier, try paraphrasing the sentence with an English sentence that uses words \emph{besides} `any' or `anyone'.\footnote{The story about `any' and `anyone' is actually rather interesting. It is well-known to linguists that `any' has at least two readings: so-called \define{free choice `any'}, which is more or less like a universal quantifier (`Any friend of Jessica's is a friend of mine!'), and \define{negative polarity (NPI) `any'}, which only occurs in `negative' contexts, like negation (`I don't want any peas!'), where it functions more or less like an existential quantifier (`It is not the case that: there exist peas that I want'). 

		Interestingly, the antecedent of a conditional is a negative environment (being equivalent to $¬\meta{A}\vee\meta{C}$), and so we expect that `any' in the antecedent of a conditional will have an existential interpretation. And it does: `If anyone is home, they will answer the door' means something like: `If someone is home, then that person will answer the door'. It does not mean `If everyone is home, then they will answer the door'. This is what we see in \ref{anyone1}.

		But \ref{anyone2} is not a conditional – its main connective is a quantifier. So here free choice `any' is the natural interpretation, so we use the universal quantifier.

		We see the same thing with the quantifier `someone'. In `if someone is a bassist, Kim Deal is', someone gets symbolised by an existential quantifier in the scope of a conditional. But in `If someone is a bassist, they are a musician' it should be symbolised by a universal taking scope over a conditional.}


\section{Quantifiers and Scope}\label{s.quantifier.scope}
Continuing the example, suppose I want to symbolise these sentences:
	\begin{earg}
		\item[\ex{qscope1}] If everyone is a bassist, then Tim is a bassist
		\item[\ex{qscope2}] Everyone is such that, if they are a bassist, then Tim is a bassist.
	\end{earg}
To symbolise these sentences, I shall have to add a new name to the symbolisation key, namely:
	\begin{ekey}
		\item[b] Tim
	\end{ekey}
Sentence \ref{qscope1} is a conditional, whose antecedent is `everyone is a bassist'. So we will symbolise it with `$(\forall x Bx \eif Bb)$'. This sentence is \emph{necessarily} true: if \emph{everyone} is indeed a bassist, then take any one you like – for example Tim – and he will be a bassist. 

Sentence \ref{qscope2}, by contrast, might best be paraphrased by `every person x is such that, if x is a bassist, then Tim is a bassist'. This is symbolised by `$\forall x (Bx \eif Bb)$'. And this sentence is false. Kim Deal is a bassist. So `$Bk$' is true. But Tim is not a bassist, so `$Bb$' is false. Accordingly, `$Bk \eif Bb$' will be false. So `$\forall x (Bx \eif Bb)$' will be false as well. 

In short, `$(\forall x Bx \eif Bb)$' and `$\forall x (Bx \eif Bb)$' are very different sentences. We can explain the difference in terms of the \emph{scope} of the quantifier.  The scope of quantification is very much like the scope of negation, which we considered when discussing \TFL (§\ref{s.mainconnscope}), and it will help to explain it in this way. We define quantifier scope officially in §\ref{s:FOLSentences}, but we also return to it in a preliminary way in §\ref{quan.scope}.

In the sentence `$(\enot Bk \eif Bb)$', the scope of `$\enot$' is just the antecedent of the conditional. We are saying something like: if `$Bk$' is false, then `$Bb$' is true. Similarly, in the sentence `$(\forall x Bx \eif Bb)$', the scope of `$\forall x$' is just the antecedent of the conditional. We are saying something like: if `$B$' is true of \emph{everything}, then `$Bb$' is also true. 

In the sentence `$\enot(Bk \eif Bb)$', the scope of `$\enot$' is the entire sentence. We are saying something like: `$(Bk \eif Bb)$' is false. Similarly, in the sentence `$\forall x (Bx \eif Bb)$', the scope of `$\forall x$' is the entire sentence. We are saying something like: `$(Bx \eif Bb)$' is true of \emph{everything}.

Scope can make a drastic difference in meaning. Reconsider these examples:
\begin{earg}
    			\item[\ex{ben1}] $(\overbracket{\forall x Bx}^{\text{Scope of `$\forall x$'}}\eif Bb)$\\ If everything is $B$, then $b$ is too. \emph{Trivially true}
    			\item[\ex{ben2}] $\overbracket{\forall x (Bx \eif Bb)}^{\text{Scope of `$\forall x$'}}$\\ All $B$s are such that $b$ is $B$. \emph{False if $b$ isn't $B$ but there are some $B$s}
    		\end{earg}

The moral of the story is simple. When you are using quantifiers and conditionals, be very careful to make sure that you have sorted out the scope correctly. 

\section{Dealing with Complex Adjectives}%syncategorematic adjectives
\label{s:complex_adj}
When we encounter a sentence like 
	\begin{earg}
		\item[\ex{syn1}] Herbie is a white car,
	\end{earg}
we can paraphrase this as `Herbie is white and Herbie is a car'. We can then use a symbolisation key like:
	\begin{ekey}
		\item[W] \blank\ is white
		\item[C] \blank\ is a car
		\item[h] Herbie
	\end{ekey}
This allows us to symbolise sentence \ref{syn1} as `$Wh \eand Ch$'. But now consider:
\begin{earg}
	\item[\ex{priv1}] Julia Gillard is a former prime minister.
	\item[\ex{priv2}] Julia Gillard is prime minister.
\end{earg}
Following the case of Herbie, we might try to use a symbolisation key like:
	\begin{ekey}
		\item[F] \blank\ is former
		\item[P] \blank\ is Prime Minister
		\item[j] Julia Gillard.
	\end{ekey}
Then we would symbolise \ref{priv1} by `$Fj \eand Pj$', and symbolise \ref{priv2} by `$Pj$'. That would however be a mistake, since that symbolisation suggests that the argument from \ref{priv1} to \ref{priv2} is valid, because the symbolisation of the premise does logically entail the symbolisation of the conclusion. 

`White' is a \define{intersective} adjective, which is a fancy way of saying that the white $F$s are among the $F$s \emph{and} among the white things: any white car is a car and white, just like any successful lawyer is a lawyer and successful, and a one tonne rhinoceros is both a rhino and a one tonne thing. But `former' is a \define{privative} adjective, which means that any former $F$ is not now among the $F$s. Other privative adjectives occur in phrases such as `fake diamond', `Deputy Lord Mayor', and `mock trial'. When symbolising these sentences, you cannot treat them as a conjunction. So you will need to symbolise `\blank\ is a fake diamond' and `\blank\ is a diamond' using completely different predicates, to avoid a spurious entailment between them. The moral is: when you see an adjectivally modified predicate like `white car', you need to ask yourself carefully whether the modifier is intersective, and can be symbolised as a conjunctive predicate, or not.\footnote{This caution also applies to adjectives  which are neither intersective nor privative, like `alleged' in `alleged murderer'. These ought not be symbolised by conjunction either.}

Things are a bit more complicated, however. Recall this example from page \pageref{daisy}:
\begin{earg}
	\item[] Daisy is a small cow.
\end{earg} We note that a small cow is definitely a cow, and so it seems we might treat `small' as an intersective adjective. We might formalise this sentence like this: `$Sd \eand Cd$', assuming this symbolisation key:
	\begin{ekey}
		\item[S] \blank\ is small
		\item[C] \blank\ is a cow
		\item[d] Daisy
	\end{ekey}
But note that our symbolisation would suggest that this argument is valid: \begin{earg}
	\item[\ex{syn2}] Daisy is a small cow; so Daisy is small.
\end{earg} The symbolised argument, $Sd \wedge Cd \ttherefore Sd$, is clearly valid.

But the original argument \ref{syn2} is not valid. Even a small cow is still rather large. (Likewise, even a short basketball player is still generally well above average height.) The point is that `\blank\ is a small cow' denotes the property something has when it is small \emph{for a cow}, while `\blank\ is small' denotes the property of being a small \emph{thing}. (In ordinary speech we tend to keep the  `for an $F$' part of these phrases silent, and let our conversational circumstances supply it automatically.) But neither should we treat `small' as a nonintersective adjective. If we do, we will be unable to account for the valid argument `Daisy is a small cow, so Daisy is a cow'. 

The correct symbolisation key will thus be this, keeping the other symbols as they were: 
	\begin{ekey}
		\item[S] \blank\ is small-for-a-cow
	\end{ekey} On this symbolisation key, the valid formal argument $Sd \wedge Cd \ttherefore Cd$ corresponds to this valid argument, as it should:
	\begin{earg}
		\item[] Daisy is a cow that is small-for-a-cow; so Daisy is a cow.
	\end{earg} 
Likewise, this rather unusual English argument turns out to be valid too: \begin{earg}
		\item[] Daisy is a cow that is small-for-a-cow; so Daisy is small-for-a-cow.
	\end{earg} (Note that it can be rather difficult to hear the English sentence `Daisy is small' as saying the same thing as the conclusion of this argument, `Daisy is small for a cow', which explains why `Daisy is a small cow, so Daisy is small' strikes us as invalid.)

If we take these observations to heart, there are many intersective adjectives which can change their meaning depending on what predicate they are paired with. Small-for-an-oil-tanker is a rather different size property than small-for-a-mouse, but in ordinary English we use the phrases `small oil tanker' and `small mouse' without bothering to make these different senses of `small' explicit. 

The way that `small' behaves makes it a member of the class of \define{subsective} adjectives, as in `poor dancer'. These are like intersective adjectives in that every poor dancer is a dancer (and every small cow is a cow). But the way that `poor' behaves in this expression is such that we \emph{cannot} conclude that a poor dancer is poor – they are bad at dancing, not necessarily financially disadvantaged. In these cases, the meaning of the modifying adjective is itself modified by the noun: in `poor dancer', we get a distinctively dancing-related sense of `poor'.   

When symbolising, it is best to make these modified adjectives  very explicit, generally introducing a new predicate to the symbolisation key to represent them. Doing so blocks the fallacious argument from `Daisy is a small cow' to `Daisy is small', where the natural sense of the conclusion is the generic size claim `Daisy is small-for-a-thing'. (Likewise, symbolising `\gap{} is poor dancer' as `\gap{} is poor-for-a-dancer and \gap{} is a dancer' blocks the fallacious argument from `Rupert Murdoch is a poor dancer' to `Rupert Murdoch is poor'.)

The upshot is this: you will need to symbolise `\blank\ is a small cow' and `\blank\ is a small animal' using different predicates of \FOL\ to stand for the different appearances of `small' – to symbolise `small-for-a-cow' and `small for-an-animal'. You can symbolise `Daisy is a small cow' as a conjunction, but it is probably best to treat it as the conjunction `Daisy is a cow and Daisy is small-for-a-cow'.\footnote{This is not redundant, because I think that `small-for-a-cow' denotes a specific size, and many things which are not cows might be that size: a big dog, for example, is a size that is small for a cow.}

The overall message is not particularly specific: treat adjectives with care, and always think about whether a conjunction or some other symbolisation best captures what is going on in the English. There is no substitute for practice in developing a good sense of how symbolise arguments. 

\section{Generics} % (fold)
\label{sec:generics}
One final complication presents itself. In English, there seems to be a difference between these sentences: \begin{earg}
	\item[\ex{gen1}] Ducks lay eggs;
	\item[\ex{gen2}] All ducks lay eggs.
\end{earg}
The sentence in \ref{gen2} is false: drakes and ducklings do not, for example. But nevertheless \ref{gen1} seems to be true, for all that. That sentence lacks an explicit quantifier – it doesn't say `\emph{all} ducks lay eggs'. It is what is known as a \define{generic} claim: it shares a structure with examples like `cows eat grass' or `rocks are hard'. Generic claims concern what is typical or normal: the typical duck lays eggs, the typical rock is hard, the typical cow eats grass. Unlike universally quantified claims, generics are \emph{exception-tolerant}: even if drakes don't lay eggs, still, ducks lay eggs.

We cannot represent this exception-tolerance very easily in \FOL. The initial idea is to use the universal quantifier, but this will give the wrong results in some cases. For it will make this argument come out valid, when it should be ruled invalid: \begin{quote}
Ducks lay eggs. Donald Duck is a duck. So Donald Duck lays eggs.
\end{quote} One alternative idea that we can implement is that the word `ducks' in \ref{gen2} is referring to a natural kind, the species of ducks. So in fact rather than being a quantified sentence, it is in fact just a subject-predicate sentence, saying something more or less like `The duck is an oviparous species'. This certainly works for some cases, such as `Rabbits are abundant', where we have to be understood as saying something about the kind. (How could an individual be abundant?) 

But this cannot handle every aspect of `ducks lay eggs'. People do treat those generics as quantified, because they are often willing to conclude things about individuals given the generic claim. Given the information that ducks lay eggs, and that Wilhelmina is a duck, most people conclude that Wilhelmina lays eggs – thus apparently treating the generic as having the logical role of a universal quantifier.

The proper treatment of generics in English remains a wide-open question.\footnote{See, for example, Sarah-Jane Leslie and Adam Lerner (2016) `Generic Generalizations', in Edward N Zalta, ed., \emph{The Stanford Encyclopedia of Philosophy}, \httpurl{plato.stanford.edu/archives/win2016/entries/generics/}.} We will not delve into it further, but you should be careful when symbolising not to be drawn into the trap of unwarily treating every generic as a universal.



% section generics (end)

\keyideas{
	\item Our quantifiers, together with truth-functional connectives, suffice to symbolise many natural language quantifier phrases including `all', `some', and `none'.
	\item Judicious choice of domain can save you trouble when symbolising, but can also hamper your ability to symbolise all the sentences you'd like to.
	\item Paraphrasing natural language sentences can often reveal their quantifier structure, even if the surface form is misleading.
	\item Complexities in symbolising arise from adjectival modification of predicates, empty predicates (relative to a chosen domain), and generics – take care. 
}
\practiceproblems
\problempart
\label{pr.BarbaraEtc}
Here are the syllogistic figures identified by Aristotle and his successors, along with their medieval names:
\begin{itemize}
	\item \textbf{Barbara.} All G are F. All H are G. So:  All H are F
	\item \textbf{Celarent.} No G are F. All H are G. So: No H are F
	\item \textbf{Ferio.} No G are F. Some H is G. So: Some H is not F
	\item \textbf{Darii.} All G are F. Some H is G. So: Some H is F.
	\item \textbf{Camestres.} All F are G. No H are G. So: No H are F.
	\item \textbf{Cesare.} No F are G. All H are G. So: No H are F.
	\item \textbf{Baroko.} All F are G. Some H is not G. So: Some H is not F.
	\item \textbf{Festino.} No F are G. Some H are G. So: Some H is not F.
	\item \textbf{Datisi.} All G are F. Some G is H. So: Some H is F.
	\item \textbf{Disamis.} Some G is F. All G are H. So: Some H is F.
	\item \textbf{Ferison.} No G are F. Some G is H. So: Some H is not F.
	\item \textbf{Bokardo.} Some G is not F. All G are H. So:  Some H is not F.
	\item \textbf{Camenes.} All F are G. No G are H So: No H is F.
	\item \textbf{Dimaris.} Some F is G. All G are H. So: Some H is F.
	\item \textbf{Fresison.} No F are G. Some G is H. So: Some H is not F.
\end{itemize}
Symbolise each argument in \FOL.



\problempart
\label{pr.FOLvegetarians}
Using the following symbolisation key:
\begin{ekey}
\item[\domain] people
\item[K] \blank\ knows the combination to the safe
\item[S] \blank\ is a spy
\item[V] \blank\ is a vegetarian
%\item[T] \blank\ trusts \gap{2}.
\item[h] Hofthor
\item[i] Ingmar
\end{ekey}
symbolise the following sentences in \FOL:
\begin{earg}
\item Neither Hofthor nor Ingmar is a vegetarian.
\item No spy knows the combination to the safe.
\item No one knows the combination to the safe unless Ingmar does.
\item Hofthor is a spy, but no vegetarian is a spy.
\end{earg}
\problempart\label{pr.FOLalligators}
Using this symbolisation key:
\begin{ekey}
\item[\domain] all animals
\item[A] \blank\ is an alligator.
\item[M] \blank\ is a monkey.
\item[R] \blank\ is a reptile.
\item[Z] \blank\ lives at the zoo.
\item[a] Amos
\item[b] Bouncer
\item[c] Cleo
\end{ekey}
symbolise each of the following sentences in \FOL:
\begin{earg}
\item Amos, Bouncer, and Cleo all live at the zoo. 
\item Bouncer is a reptile, but not an alligator. 
%\item If Cleo loves Bouncer, then Bouncer is a monkey. 
%\item If both Bouncer and Cleo are alligators, then Amos loves them both.
\item Some reptile lives at the zoo. 
\item Every alligator is a reptile. 
\item Any animal that lives at the zoo is either a monkey or an alligator. 
\item There are reptiles which are not alligators.
%\item Cleo loves a reptile.
%\item Bouncer loves all the monkeys that live at the zoo.
%\item All the monkeys that Amos loves love him back.
\item If any animal is a reptile, then Amos is.
\item If any animal is an alligator, then it is a reptile.
%\item Every monkey that Cleo loves is also loved by Amos.
%\item There is a monkey that loves Bouncer, but sadly Bouncer does not reciprocate this love.
\end{earg}


\problempart
\label{pr.FOLarguments}
For each argument, write a symbolisation key and symbolise the argument in \FOL. In each case, try to decide if the argument you have symbolized is valid.
\begin{earg}
\item Willard is a logician. All logicians wear funny hats. So Willard wears a funny hat.
\item Nothing on my desk escapes my attention. There is a computer on my desk. As such, there is a computer that does not escape my attention.
\item All my dreams are black and white. Old TV shows are in black and white. Therefore, some of my dreams are old TV shows.
\item Neither Holmes nor Watson has been to Australia. A person could see a kangaroo only if they had been to Australia or to a zoo. Although Watson has not seen a kangaroo, Holmes has. Therefore, Holmes has been to a zoo.
\item No one expects the Spanish Inquisition. No one knows the troubles I've seen. Therefore, anyone who expects the Spanish Inquisition knows the troubles I've seen.
\item All babies are illogical. Nobody who is illogical can manage a crocodile. Berthold is a baby. Therefore, Berthold is unable to manage a crocodile.
\end{earg}



\chapter{Multiple Generality}\label{s:MultipleGenerality}
So far, we have only considered sentences that require simple predicates with just one `gap', and at most one quantifier. Much of the fragment of \FOL\ that focuses on such sentences was already discovered and codified into \emph{syllogistic} logic by Aristotle more than 2000 years ago. The full power of \FOL\ really comes out when we start to use predicates with many `gaps' and multiple quantifiers. Despite first appearances, the discovery of how to handle such sentences was a very significant one.  For this insight, we largely have the German mathematician and philosopher Gottlob Frege (1879) to thank.\footnote{Frege (1879) \emph{Begriffsschrift, eine der arithmetischen nachgebildete Formelsprache des reinen Denkens}, Halle a. S.: Louis Nebert. Translated as `Concept Script, a formal language of pure thought modelled upon that of arithmetic', by S. Bauer-Mengelberg in J. van Heijenoort, ed. (1967) \emph{From Frege to Gödel: A Source Book in Mathematical Logic, 1879–1931}, Cambridge, MA: Harvard University Press. The same logic was independently discovered by the American philosopher and mathematician Charles Sanders Peirce: see Peirce (1885) `On the Algebra of Logic. A Contribution to the Philosophy of Notation' \emph{American Journal of Mathematics} \textbf{7}, pp.\ 197–202. \textbf{Warning}: don't consult either of these original works, which will likely just confuse you. The present text is the result of more than a century of refinement in how to present the basic systems Frege and Peirce introduced, and is a lot more user friendly.}



\section{Many-place Predicates}
All of the predicates that we have considered so far concern properties that can be attributed to objects might have by themselves, as it were. `Herbie is white', `Kim Deal is a bassist', etc., all talk about the features of a single individual. The associated predicates, such as `\blank\ is white', have one gap in them, and to make a sentence, we simply need to slot in one name. They are \define{one-place} predicates. 

But other predicates concern the \emph{relationship} between two things. Here are some examples of many-place predicates in English:
	\begin{quote}
		\blank\ loves \blank\\
		\blank\ is to the left of \blank\\
		\blank\ is in debt to \blank\\
		\blank\ is supervised by \blank
	\end{quote}
These are \define{two-place} predicates. They need to be filled in with two terms (names or pronouns, most commonly) in order to make a sentence. Conversely, if we start with an English sentence containing many singular terms, we can remove two singular terms, to obtain different two-place predicates. Consider the sentence `Vinnie borrowed the family car from Nunzio'. By deleting two singular terms, we can obtain any of three different two-place predicates:
	\begin{quote}
		Vinnie borrowed \blank\ from \blank;\\
		\blank\ borrowed the family car from \blank;\\
		\blank\ borrowed \blank\ from Nunzio.
	\end{quote}
And by removing all three singular terms, we  obtain a \define{three-place} predicate:
	\begin{quote}
		\blank\ borrowed \blank\ from \blank.
	\end{quote}
Indeed, there is no in principle upper limit on the number of gaps or places that our predicates may contain.

Now there is a little problem with the above. I have used the same symbol, `\blank', to indicate a gap formed by deleting a term from a sentence. However (as Frege emphasised), these are \emph{different} gaps. To obtain a sentence, we can fill them in with the same term, but we can equally fill them in with different terms, and in various different orders. The following are all perfectly good sentences, obtained by filling in the gaps in `\blank\ loves \blank', but they mean quite different things:
	\begin{quote}
		Karl loves Karl;\\
		Karl loves Imre;\\
		Imre loves Karl;\\
		Imre loves Imre.
	\end{quote}
The point is that we need some way of keeping track of the gaps in predicates, so that we can keep track of how we are filling them in. 

Another way to put the point: when it comes to two-(or more)-place predicates, sometimes the order matters. `Shaq is taller than Jordan' doesn't mean the same thing as `Jordan is taller than Shaq'. It matters whose name fills the first gap in the predicate, and whose name fills the second. 

To keep track of the gaps, we shall label them. The labelling conventions I  adopt are best explained by example. Suppose I want to symbolise the following sentences:
	\begin{earg}
%		\item[\ex{terms3}] Imre is at least as tall Karl.
%		\item[\ex{terms4}] Imre is shorter than Karl.
		\item[\ex{terms3}] Karl loves Imre.
		\item[\ex{terms4}] Imre loves himself.
		\item[\ex{terms5}] Karl loves Imre, but not vice versa.
		\item[\ex{terms6}] Karl is loved by Imre.
	\end{earg}
I will start with the following symbolisation key:
	\begin{ekey}
		\item[\domain] people
		\item[i] Imre
		\item[k] Karl
		\item[L] \gap{1}\ loves \gap{2}
	\end{ekey}
%Sentence \ref{terms3} can now be symbolised by `$Tmd$'. Note the order of the names! 
%Sentence \ref{terms4} might seem as if it requires a new predicate. But there is obviously a connection connection between `shorter' and `taller.' We can paraphrase sentence \ref{terms4} using predicates already in our key: `It is not the case that Imre is as tall or taller than Karl'. We can now symbolise it as `$\enot Tmd$'.
\begin{itemize}
	\item Sentence \ref{terms3} will now be symbolised by `$Lki$'. 
	\item Sentence \ref{terms4} can be paraphrased as `Imre loves Imre'. It can now be symbolised by `$Lii$'. 
\item Sentence \ref{terms5} is a conjunction. We might paraphrase it as `Karl loves Imre, and Imre does not love Karl'. It can now be symbolised by `$Lki \eand \enot Lik$'. 
\item Sentence \ref{terms6} might be paraphrased by `Imre loves Karl'. It can then be symbolised by `$Lik$'. Of course, this erases the difference in tone between the active and passive voice; such nuances are lost in \FOL. 
\end{itemize}
This last example highlights something important. Suppose we add to our symbolisation key the following:
	\begin{ekey}
		\item[M] \gap{2} loves \gap{1}
	\end{ekey}
Here, we have used the same English word (`loves') as we used in our symbolisation key for `$L$'. However, we have swapped the order of the \emph{gaps} around (just look closely at those little subscripts!) So `$Mki$' and `$Lik$' now \emph{both} symbolise `Imre loves Karl'. `$Mik$' and `$Lki$' now \emph{both} symbolise `Karl loves Imre'. Since love can be unrequited, these are very different claims. The moral is simple. When we are dealing with predicates with more than one place, we need to pay careful attention to the order of the places. 

With these examples in hand, I can now give the official account of how we understand \FOL\ symbolisations. Suppose we have an \FOL\ expression $\meta{A}t_{1}…t_{k}$, where each $t_{i}$ is a name or a variable, symbolising a singular term, and where $\meta{A}$ symbolises a $k$-place predicate. The $i$-th term is to be interpreted as filling the gap labelled `$i$'. So consider the following symbolisation key: \begin{ekey}
	\item[\domain] places
		\item[a] Adelaide
		\item[b] Alice Springs
		\item[c] Coober Pedy
		\item[B] \gap{2}\ is between \gap{1}\ and \gap{3}
		\item[K] \gap{1}\ is between \gap{2}\ and \gap{3}
\end{ekey}. Then if we want to symbolise `Coober Pedy is between Adelaide and Alice Springs', we can do so using either `$Bacb$' or `$Kcab$'. The difference is in how the symbolisation key instructs us to fill the gaps we have established in the predicate as we take steps to represent it symbolically. There is no `right' answer here: either can be good. The representation using `$B$' graphically represents which item is between the other two in the syntax itself, while the representation using `$K$' is more faithful to the original English.


Suppose we add to our symbolisation key the following:
	\begin{ekey}
		\item[S] \gap{1} thinks only of \gap{1}
		\item[T] \gap{1} thinks only of \gap{2}
		\item[a] Alice
	\end{ekey}
As in the case of `$L$' and `$M$' above, the difference between these examples is only in how the gaps in the construction `… thinks only of …' are labelled. In `$T$', we have labelled the two gaps differently. They do not \emph{need} to be filled with different names or variables, but there is always the potential to put different names in those different gaps. In the case of `$S$', the gaps have the same label. In some sense, there is \emph{only one} gap in this sentence, which is why the symbolisation key associates it with a one-place predicate – it means something like `$x$ thinks only of themself'. The second predicate is more flexible. Take something we can say with the predicate `$S$', such as `$Sa$', `Alice thinks only of herself'. We can express pretty much the same thought using the two-place predicate `$T$': `$Taa$'. 


We have introduced a potential ambiguity in our treatment of predicates. (See also §\ref{s:termsandf}.) There is nothing overt in our language that distinguishes the one-place predicate `$A$' (such that `$Ab$' is grammatical) from the two-place predicate `$A$' (such that `$Ab$' is ungrammatical, but `$Abk$' is grammatical). We are, in effect, just letting context disambiguate how many argument places there are in a given predicate, by assuming that in any expression of \FOL\ we write down, the number of names or variables following a predicate indicates how many places it has. We could introduce a system to disambiguate: perhaps adding a superscripted `1' to all one-place predicates, a superscripted `2' to all two-place predicates, etc. Then `$A^{1}b$' is grammatical while `$A^{1}bk$' is not; conversely, `$A^{2}b$' is ungrammatical and `$A^{2}bk$' is grammatical. This system of superscripts would be effective but cumbersome. We will thus keep to our existing practice, letting context disambiguate. What you should \emph{not} do, however, is make use of the same capital letter to symbolise two different predicates in the same symbolisation key. If you do that, context will not disambiguate, and you will have failed to give an interpretation of the language at all. 


\section{Scope and Nested Quantifiers}\label{quan.scope}


Once we have two (or more) gaps in a predicate, we can fill them with different things. We've so far seen cases where multiple names are slotted into a many-place predicate. But we can also insert other terms, like variables. So to continue our example using the predicate `$T$', `\gap{1} thinks only of \gap{2}', we can put a variable in the first gap, and a name in the second, if we wish: `$Txa$'. This isn't a sentence, because no quantifer tells us how to understand that variable. (The sentence might be representing `they think only of Alice', but without context there is no determinate referent for the pronoun `they'.) Introduce a quantifier, and we have an interpretable sentence: 
\begin{earg}
	\item[\ex{selfi1}] $\forall x Txa$;\\ Everyone thinks only of Alice.
\end{earg}

The fact that we can fill the two gaps of two-place predicates with different things, or even with the same thing, gives us a reason to favour the two-place predicate symbolisation of `Alice thinks only of themself' as `$Taa$'. That allows us to symbolise certain arguments that cannot be adequately symbolised using a one-place predicate. For example: `Alice thinks only of herself; so there is someone who is the only person Alice thinks of'. The symbolisation of this argument might be: `$Taa \ttherefore \exists x Tax$'. This might have some prospect of being valid, whereas `$Sa \ttherefore \exists x Tax$' will not be valid.

The real power of many-place predicates comes when we consider examples in which \emph{both} gaps in the predicate are filled by variables governed by different quantifiers. In cases where the quantifier expressions interact, we can express things we cannot say even when we allow logically complex combinations of one-quantifier sentences. With this power comes potential confusion too. So let's proceed carefully. 

Consider the sentence `everyone loves someone'. This illustrates our goal, as two quantifier expressions occur in this sentence: `everyone' and `someone'. But it also illustrates the potential pitfalls, as there is a possible ambiguity in this sentence. It might mean either of the following:
	\begin{earg}
		\item[\ex{lovecycle}] For every person, there is some person that they love
		\item[\ex{loveconverge}] There is some particular person whom every person loves
	\end{earg} It is fairly straightforward to see that these don't mean the same thing. The first would be true as long as everybody has somebody they love. One sort of case in which \ref{lovecycle} is true is the cyclic central love triangle in \emph{Twelfth Night}, where Viola loves Duke Orsino, the Duke loves Olivia, and Olivia loves Viola (who, disguised as a young man, is the Duke's go-between with Olivia). 

In the \emph{Twelfth Night} situation, \ref{loveconverge} is not true. It could only be true if everybody loves the same person, e.g., if the Duke, Viola, \emph{and Olivia herself} all love Olivia. 


How can we symbolise these two different disambiguations of our original sentence? (Remember: one of the strengths of symbolic logic is that it is supposed to be able to clearly represent that which would be ambiguous in natural language.)

Let's paraphrase a little more formally as we step towards a fully symbolic representation. As our sentence has two quantifiers, I will use numbers to link pronouns in our paraphrase with the quantifier expressions which govern them. Using this device, our sentences can be paraphrased as follows: \begin{earg}
	\item[\ex{lovecycle1}] Everyone\textsubscript{1} is such that there is someone\textsubscript{2} such that: they\textsubscript{1} love them\textsubscript{2}. 
	\item[\ex{loveconverge1}] There is someone\textsubscript{2} such that everyone\textsubscript{1} is such that: they\textsubscript{1} love them\textsubscript{2}.
\end{earg} (Take a moment to convince yourself that these paraphrases succeed.)

You can see immediately that the difference in these paraphrases lies in the order of the quantifier expressions, and the remainder of the paraphrase, `they\textsubscript{1} love them\textsubscript{2}', is the same in each sentence. Using variables to symbolise pronouns, and choosing the variable `$x$' for `they\textsubscript{1}' and `$y$' for `them\textsubscript{2}', we can symbolise this `$Lxy$', where `$L$' stands for the two-place predicate `\gap{1}\ loves \gap{2}'. 

The quantifier order in the paraphrases governs how they interact. As we saw in §\ref{s.quantifier.scope}, the scope of a quantifier is roughly the \FOL\ expression in which that quantifier is the main connective. (Later on we will be a little more precise about the way that quantifier scope functions in \FOL: see §\ref{s:FOLSentences}.) So in `$\forall x \exists y Lxy$', the scope of `$\forall x$' is the whole sentence, while the scope of `$\exists y$' is just `$\exists y Lxy$'. The following guides us in interpreting these `nested' quantifiers, in which one falls in the scope of another: \factoidbox{When one quantifier occurs in the scope of another, the narrower scope quantifier should be understood \emph{with respect to} the value assigned to a variable by the wider scope quantifier.} 

Let's apply this to our example. In \ref{lovecycle1} `everyone' comes first, and `someone' comes next. The intended interpretation is that this is true iff for any person $x$ that you pick, with respect to that choice you can then find someone $y$ who $x$ loves. If you had chosen someone else as the value of $x$, then parasitic on that different choice you may end up needing to find a different value for $y$. Compare the reversed quantifier scope in \ref{loveconverge1}. That is true iff there is someone $y$ such that, with respect to that particular choice for $y$, any person $x$ you pick, $x$ loves $y$. With respect to a different initial choice for $y$, there may be values for $x$ that do not satisfy $x$ loves $y$, but as the initial choice is governed by an existential quantifier, that won't undermine the truth of the sentence. This gives us our two different symbolisations:
\begin{itemize}
	\item Sentence \ref{lovecycle} can be symbolised by `$\forall x \exists y Lxy$'. Return to our example love triangle between Duke Orsino, Viola, and Olivia. For any of the three people you might choose, you can find another person in the domain who they love. So sentence \ref{lovecycle} is true. 
\item Sentence \ref{loveconverge} is symbolised by `$\exists y \forall x Lxy$'. Sentence \ref{loveconverge} is \emph{not} true in the \emph{Twelfth Night} situation. For each of the people in the domain, you can find someone who \emph{doesn't} love them, and hence no one is universally beloved. If, instead, each person loved Olivia, then we could find someone (Olivia), such that everyone else we examined turns out to love them. In that case, \ref{loveconverge} would be true.
\end{itemize}


This example, besides giving some indication of how to read sentences with multiple quantifiers, illustrates that quantifier scope matters a great deal. Indeed, the mistake that arises when one illegitimately switches them around even has a special name: a \emph{quantifier shift fallacy}. Here is a real life example from Aristotle:\footnote{Note that it is hotly contested whether Aristotle actually commits a fallacy here, given the compressed nature of his prose. See, \emph{inter alia}, J L Ackrill (1999), `Aristotle on \emph{eudaimonia}', pp.\ 57–77 in N Sherman, ed., \emph{Aristotle’s Ethics: Critical Essays}, Rowman \& Littlefield.} \begin{quote}
	Suppose, then, that [A] the things achievable by action have some end that we wish for because of itself, and because of which we wish for the other things, and that we do not choose everything because of something else – for if we do, it will go on without limit, so that desire will prove to be empty and futile[; c]learly, [B] this end will be the good, that is to say, the best good. (Aristotle,  \emph{Nichomachean Ethics} 1094\textsuperscript{a}18-22)
\end{quote} Setting aside Aristotle's subsidiary argument about desire, this argument seems to involve the following pattern of inference: \begin{earg}
		\item[] Every action aims at some end which is desired because of itself. \hfill ($\forall \exists$)
		\item[So:] There is end desired because of itself which is the aim of every action, the best good. \hfill ($\exists \forall$)
	\end{earg}
This argument form is obviously invalid. It's just as bad as:\footnote{Thanks to Rob Trueman for the example.}
	\begin{earg}
		\item[] Every dog has its day. \hfill ($\forall \exists$)
		\item[So:] There is a day for all the dogs. \hfill ($\exists \forall$)
	\end{earg}
The moral is: take great care with the scope of quantification. 

%The fallacies, though, arise only when we swap around universal with existential quantifiers.  does not much matter within a single block of quantifiers. Using the same scheme, compare `$\exists x \exists y Lxy$' and `$\exists y \exists x Lxy$'. These would naturally symbolise the English sentences `there is someone who loves someone' and `there is someone whom is loved by someone', respectively. But, though these differ in nuance, they are true in exactly the same situations. (Similar comments apply to the universal quantifier.)


\section{Stepping Stones to Symbolisation}
Once we have the possibility of multiple quantifiers and many-place predicates, representation in \FOL\ can quickly start to become a bit tricky. When you are trying to symbolise a complex sentence, I recommend laying down several stepping stones. As usual, this idea is best illustrated by example. Consider this representation key:
\begin{ekey}
\item[\domain] people and dogs
\item[D] \gap{1} is a dog
\item[F] \gap{1} is a friend of \gap{2}
\item[O] \gap{1} owns \gap{2}
\item[g] Geraldo
\end{ekey}
And now let's try to symbolise these sentences:
\begin{earg}
\item[\ex{dog2}] Geraldo is a dog owner.
\item[\ex{dog3}] Someone is a dog owner.
\item[\ex{dog4}] All of Geraldo's friends are dog owners.
\item[\ex{dog5}] Every dog owner is the friend of a dog owner.
\item[\ex{dog6}] Every dog owner's friend owns a dog of a friend.
\end{earg}
Sentence \ref{dog2} can be paraphrased as, `There is a dog that Geraldo owns'. This can be symbolised by `$\exists x(Dx \eand Ogx)$'.

Sentence \ref{dog3} can be paraphrased as, `There is some y such that y is a dog owner'. Dealing with part of this, we might write `$\exists y(y\text{ is a dog owner})$'. Now the fragment we have left as `$y$ is a dog owner' is much like sentence \ref{dog2}, except that it is not specifically about Geraldo. (We chose the variable `$y$' with this in mind, to avoid a clash with the variable `$x$' in our symbolisation of \ref{dog2} – see below.) So we can symbolise sentence \ref{dog3} by:
$$\exists y \exists x(Dx \eand Oyx).$$ 


I need to pause to clarify something here. In working out how to symbolise the last sentence, we wrote down `$\exists y(y\text{ is a dog owner})$'. To be very clear: this is \emph{neither} a \FOL\ sentence \emph{nor} an English sentence: it uses bits of \FOL\ (`$\exists$', `$y$') and bits of English (`dog owner'). It is really is \emph{just a stepping-stone} on the way to symbolising the entire English sentence with a \FOL\ sentence, a bit of rough-working-out.

Sentence \ref{dog4} can be paraphrased as, `Everyone who is a friend of Geraldo is a dog owner'. Using our stepping-stone tactic, we might write 
$$\forall x \bigl(Fxg \eif x \text{ is a dog owner}\bigr)$$
Now the fragment that we have left to deal with, `$x$ is a dog owner', is structurally just like sentence \ref{dog2}. But it would be a mistake for us simply to put `$x$' in place of `$g$' from our symbolisation of \ref{dog2}, yielding
$$\forall x \bigl(Fxg \eif \exists x(Dx \eand Oxx) \bigr).$$
Here we have a \define{clash of variables}. The scope of the universal quantifier, `$\forall x$', is the entire conditional. But `$Dx$' also falls within the scope of the existential quantifier `$\exists x$'. Which quantifier has priority and governs the interpretation of the variable? In \FOL, if a variable $\meta{x}$ occurs in an \FOL\ sentence, it is always governed by the quantifier which has the narrowest scope which includes that occurrence of \meta{x}. So in the sentence above, the quantifier `$\exists x$' governs every occurrence of `$x$' in `$(Dx \eand Oxx)$'. Given this, the symbolisation does not mean what we intended. It says, roughly, `everyone who is a friend of Geraldo is such that there is a self-owning dog'. This is not at all the meaning of the English sentence we are aiming to symbolise.


To provide an adequate symbolisation, then, we must avoid clashing variables. We can do this easily enough. There was no requirement to use `$x$' as the variable in our symbolisation of \ref{dog2}, so we can easily choose some different variable for our existential quantifier. That will give us something like this, which adequately symbolises sentence \ref{dog4}:
$$\forall x\bigl(Fxg \eif\exists z(Dz \eand Oxz) \bigr).$$


Sentence \ref{dog5} can be paraphrased as `For any x that is a dog owner, there is a dog owner who is a friend of x'. Using our stepping-stone tactic, this becomes 
$$\forall x\bigl(\mbox{$x$ is a dog owner}\eif\exists y(\mbox{$y$ is a dog owner}\eand Fyx) \bigr)$$
Completing the symbolisation, we end up with
$$\forall x\bigl(\exists z(Dz \eand Oxz)\eif\exists y\bigl(\exists z(Dz \eand Oyz)\eand Fyx\bigr) \bigr)$$
Note that we have used the same variable, `$z$', in both the antecedent and the consequent of the conditional, but that these are governed by two different quantifiers. This is ok: there is no potential confusion here, because it is obvious which quantifier governs each variable. We might graphically represent the scope of the quantifiers thus:
$$\overbracket{\forall x\bigl(\,\overbracket{\exists z(Dz \eand Oxz)}^{\text{scope of 1st `}\exists z\text{'}}\eif \overbracket{\exists y(\,\overbracket{\exists z(Dz \eand Oyz)}^{\text{scope of 2nd `}\exists z\text{'}}\mathop{\eand} Fyx) \bigr)}^{\text{scope of `}\exists y\text{'}}}^{\text{scope of `}\forall x\text{'}}$$
Even in this case, however, you might want to choose different variables for every quantifier just as a practical matter, preventing any possibility of confusion for your readers.

Sentence \ref{dog6} is the trickiest yet. First we paraphrase it as `For any x that is a friend of a dog owner, x owns a dog which is also owned by a friend of x'. Using our stepping-stone tactic, this becomes:
$$\forall x\bigl(x\text{ is a friend of a dog owner}\eif x\text{ owns a dog which is owned by a friend of }x\bigr).$$
Breaking this down a bit more:
$$\forall x\bigl(\exists y(Fxy \eand y\text{ is a dog owner})\eif \exists y(Dy \eand Oxy \eand y\text{ is owned by a friend of }x) \bigr).$$
And a bit more: 
$$\forall x\bigl(\exists y(Fxy \eand \exists z(Dz \eand Oyz)) \eif \exists y(Dy \eand Oxy \eand \exists z(Fzx \eand Ozy)) \bigr).$$
And we are done!

\section{Sentence Structure and Levels of Analysis} % (fold)
\label{sec:structure_and_levels_of_analysis}

% section structure_and_levels_of_analysis (end)

As I emphasised in §\ref{ss.idargstr}, a single English sentence has many structures, depending on how fine-grained one is in the analysis of that sentence. We could symbolise `Antony owns a car' in a number of different ways given the resources we have so far, in increasingly finer detail: \begin{enumerate}
	\item We could symbolise it as an atomic sentence of \TFL\ like `$A$', ignoring all internal structure of the sentence (because it has no internal truth-functional structure). This is also a zero-place predicate, so is also a sentence of \FOL. 
	\item We could symbolise it as another atomic sentence of \FOL, but one which does recognise the internal subject-predicate struture of the English sentence. For example, we could symbolise it as `$Wa$', where `$W$' symbolises `\gap{1}\ owns a car' and `$a$' symbolises `Antony'.
	\item Or we could symbolise it as a complex quantified sentence `$\exists y (Cy \eand Oay)$', where `$a$' is as before, but `$C$' means `\gap{1}\ is a car' and `$O$' is `\gap{1}\ owns \gap{2}'.
	\item We could imaginably symbolise it at even finer levels of structure, breaking down the predicate `is a car' into the verb phrase `is' and the indefinite noun phrase `a car' (which is itself complex). This would go beyond the representational resources even of \FOL.
\end{enumerate} Note that any structure identified is preserved: the name `$a$' continues to appear in more fine-grained symbolisations once it has appeared. A one-place predicate, having appeared in the second symbolisation, still appears indirectly in the third symbolisation. For the open sentence `$\exists y (Cy \eand O\meta{x}y)$' can be understood a representing a complex one-place predicate:  `\meta{x}' is not associated with any quantifier, and can be replaced by a name to form a grammatical sentence.

How to symbolise the structure of a sentence very much depends on what purpose you have in symbolising. What matters is that you manage to represent enough structure to determine whether the target argument you are symbolising is valid. Valid arguments can be symbolised as invalid arguments, if you don't attend to the relevant structure, or don't have resources in your language to represent that structure. But we just observed that any more fine-grained analysis of the structure of a sentence retains the coarser structure (as it just adds more detailed substructure). So if you can show an argument is valid (conclusive in virtue of its structure) at some level of analysis, it will remain valid according to any more fine-grained understanding of the structure of that sentence. So you should aim to symbolise \emph{just enough} structure in an argument to be able to demonstrate its validity – if indeed it is valid. 

\keyideas{
	\item The true power of \FOL\ comes from its ability to handle multiple generality.
	\item But with great power comes additional complexity: we need to keep track of different places in our predicates, and different quantifiers that may govern those places. Even small alterations in scope or order can drastically change the meaning of the symbolisation we produce.
	\item Use of paraphrases and hybrid English-\FOL\ sentences can be useful in figuring out how to symbolise a complex claim featuring multiple quantifiers and complex many-place predicates.
}


\practiceproblems
\problempart
Using this symbolisation key:
\begin{ekey}
\item[\domain] all animals
\item[A] \gap{1} is an alligator
\item[M] \gap{1} is a monkey
\item[R] \gap{1} is a reptile
\item[Z] \gap{1} lives at the zoo
\item[L] \gap{1} loves \gap{2}
\item[a] Amos
\item[b] Bouncer
\item[c] Cleo
\end{ekey}
symbolise each of the following sentences in \FOL:
\begin{earg}
%\item Amos, Bouncer, and Cleo all live at the zoo. 
%\item Bouncer is a reptile, but not an alligator. 
\item If Cleo loves Bouncer, then Bouncer is a monkey. 
\item If both Bouncer and Cleo are alligators, then Amos loves them both.
%\item Some reptile lives at the zoo. 
%\item Every alligator is a reptile. 
%\item Any animal that lives at the zoo is either a monkey or an alligator. 
%\item There are reptiles which are not alligators.
\item Cleo loves a reptile.
\item Bouncer loves all the monkeys that live at the zoo.
\item All the monkeys that Amos loves love him back.
%\item If any animal is an reptile, then Amos is.
%\item If any animal is an alligator, then it is a reptile.
\item Every monkey that Cleo loves is also loved by Amos.
\item There is a monkey that loves Bouncer, but sadly Bouncer does not reciprocate this love.
\end{earg}

\problempart 
Using the following symbolisation key:
\begin{ekey}
\item[\domain] all animals
\item[D] \gap{1} is a dog
\item[S] \gap{1} likes samurai movies
\item[L] \gap{1} is larger than \gap{2}
\item[b] Bertie
\item[e] Emerson
\item[f] Fergus
\end{ekey}
symbolise the following sentences in \FOL:
\begin{earg}
\item Bertie is a dog who likes samurai movies.
\item Bertie, Emerson, and Fergus are all dogs.
\item Emerson is larger than Bertie, and Fergus is larger than Emerson.
\item All dogs like samurai movies.
\item Only dogs like samurai movies.
\item There is a dog that is larger than Emerson.
\item If there is a dog larger than Fergus, then there is a dog larger than Emerson.
\item No animal that likes samurai movies is larger than Emerson.
\item No dog is larger than Fergus.
\item Any animal that dislikes samurai movies is larger than Bertie.
\item There is an animal that is between Bertie and Emerson in size.
\item There is no dog that is between Bertie and Emerson in size.
\item No dog is larger than itself.
\item Every dog is larger than some dog.
\item There is an animal that is smaller than every dog.
\item If there is an animal that is larger than any dog, then that animal does not like samurai movies.
\end{earg}

\problempart
Using this symbolisation key, 
\begin{ekey}
	\item[\domain] all animals
	\item[L] \gap{1} is larger than \gap{2}.
	\item[F] \gap{1} is friendlier than \gap{2}.
	\item[D] \gap{1} is a dog.
	\item[b] Bertie.
	\item[e] Emerson.
	\item[f] Fergus.
\end{ekey}
render the following into natural English, commenting on any difficulties:
\begin{earg}
	\item $(Leb \wedge Lfe)$;
	\item $\exists x (Dx \eand Lxe)$;
	\item $\bigl(\exists x (Dx \eand Lxf) \eif \exists y (Dy \eand Fye)\bigr)$;
	\item $\forall x (Dx \eif \neg Lxf)$;
	\item $\exists y ((Lyb \eand Ley) \eor (Lye \eand Lby))$;
	\item $\forall x (Dx \eif \exists y (Dy \eand Fxy))$;
	\item $\forall x \forall y (Lxy \eif \exists z (Dz \eand (Fyz \eand Fzx)))$.
\end{earg}

\problempart
Using the following symbolisation key:
\begin{ekey}
\item[\domain] people and dishes at a potluck
\item[R] \gap{1} has run out.
\item[T] \gap{1} is on the table.
\item[F] \gap{1} is food.
\item[P] \gap{1} is a person.
\item[G] \gap{1} is guacamole.
\item[L] \gap{1} likes \gap{2}.
\item[e] Eli
\item[f] Francesca
\end{ekey}
render the following into natural English, commenting on any difficulties:
\begin{earg}
\item $\forall x (Fx \eif Tx)$; %All the food is on the table.
\item $(\exists x Gx \eif \forall x (Gx \eif Tx))$; %If the guacamole has not run out, then it is on the table.
\item $\forall x \forall y ((Px \eand Gy)\eif Lxy)$; %Everyone likes the guacamole.
\item $\exists x \exists y (((Gx\eand Py) \eand Lyx) \eif Lex)$; %If anyone likes the guacamole, then Eli does.
\item $\forall x(Lfx \eif Rx) $; %Francesca only likes the dishes that have run out.
\item $\forall x (Px \to \enot (Lfx \eor Lxf))$; %Francesca likes no one, and no one likes Francesca.
\item $\forall x (\exists y (Gy \eand Lxy) \to Lex)$; %Eli likes anyone who likes the guacamole.
\item $\forall x \forall y ((Px \eand Py) \eif ((Lex \eand Lyx) \eif) Ley)$; %Eli likes anyone who likes the people that he likes.
\item $(\exists x (Px \eand Tx) \eif \forall y (Fy \to Ry))$. %If there is a person on the table already, then all of the food must have run out.
\end{earg}


\problempart
\label{pr.FOLballet}
Using the following symbolisation key:
\begin{ekey}
\item[\domain] people
\item[D] \gap{1} dances ballet.
\item[F] \gap{1} is female.
\item[M] \gap{1} is male.
\item[C] \gap{1} is a child of \gap{2}.
\item[S] \gap{1} is a sibling of \gap{2}.
\item[e] Elmer
\item[j] Jane
\item[p] Patrick
\end{ekey}
symbolise the following arguments in \FOL:
\begin{earg}
\item All of Patrick's children are ballet dancers.
\item Jane is Patrick's daughter.
\item Patrick has a daughter.
\item Jane is an only child.
\item All of Patrick's sons dance ballet.
\item Patrick has no sons.
\item Jane is Elmer's niece.
\item Patrick is Elmer's brother.
\item Patrick's brothers have no children.
\item Jane is an aunt.
\item Everyone who dances ballet has a brother who also dances ballet.
\item Every woman who dances ballet is the child of someone who dances ballet.
\end{earg}




\problempart
Consider the following symbolisation key:
\begin{ekey}
\item[\domain] Fred, Amy, Kuiping
\item[P] \gap{1} pats \gap{2} on the head.
\item[T] \gap{1} is taller than \gap{2}.
\end{ekey}
If we hold fixed this assignment of meanings to the predicates, why is it possible that `$\exists x \forall y Pxy$' is true, but not possible that  `$\exists x \forall y Txy$' is true?


\chapter{Identity}\label{sec.identity}

\section{A Tricky Argument}

Consider this sentence:
\begin{earg}
\item[\ex{else1}] Pavel owes money to everyone.
\end{earg}
Let the domain be people; this will allow us to translate `everyone' as a universal quantifier. Offering the symbolisation key:
	\begin{ekey}
		\item[O] \gap{1} owes money to \gap{2}
		\item[p] Pavel
	\end{ekey}
we can symbolise sentence \ref{else1} by `$\forall x Opx$'. But this has a (perhaps) odd consequence. It requires that Pavel owes money to every member of the domain (whatever the domain may be). The domain certainly includes Pavel. So this entails that Pavel owes money to himself. 

Perhaps we meant to say:
	\begin{earg}
		\item[\ex{else1b}] Pavel owes money to everyone \emph{else}
		\item[\ex{else1c}] Pavel owes money to everyone \emph{other than} Pavel
		\item[\ex{else1d}] Pavel owes money to everyone \emph{except} Pavel himself
	\end{earg}
%But we do not know how to deal with the italicised words yet. The solution is to add another symbol to \FOL. 
We want to add something to the symbolisation of \ref{else1} to handle these italicised words. Some interesting issues arise as we do so.


\section{First Attempt to Handle the Argument}

The sentences in \ref{else1b}–\ref{else1d} can all be paraphrased as follows:
\begin{earg}
	\item[\ex{else1e}] Everyone who isn't Pavel is such that: Pavel owes money to them.
\end{earg} This is a sentence of the form `every \meta{F} [person who is not Pavel] is \meta{G} [owed money by Pavel]'. Accordingly it can be symbolised by something with this structure: $\forall x (\meta{F}x \eif \meta{G}x)$. Here is an attempt to fill in the schematic letters:
	\begin{ekey}
		\item[O] \gap{1} owes money to \gap{2}
		\item[I] \gap{1} is \gap{2}
		\item[p] Pavel
	\end{ekey}
With this symbolisation  key, here is a proposed symbolisation: $\forall y (\enot Ipy \eif Opy)$. 

This symbolisation works well. But, it turns out, it doesn't quite do what we wanted it to. Suppose `$h$' names Hikaru, and consider this argument, with the symbolisation next to the English sentences:
\begin{earg}
	\item[\ex{hikaru1}] Pavel owes money to everyone else: $\forall y (\enot Ipy \eif Opy)$.
	\item[\ex{hikaru2}] Hikaru isn't Pavel: $\enot Ihp$.
	\item[So:] Pavel owes money to Hikaru: $Oph$.
\end{earg}This argument is valid in English. But its symbolisation is not valid. If we pick Hikaru to be the value of `$y$', we get from \ref{hikaru1} the conditional $\enot Iph \eif Oph$. But \ref{hikaru2} doesn't give us the antecedent of this conditional: $\enot Iph$ is potentially quite different from $\enot Ihp$. So the argument isn't formally valid.

The argument isn't formally valid, because the sentence `$\enot Ihp$' doesn't formally entail `$\enot Ihp$' as a matter of logical structure alone. If the original argument is valid, then we need a symbolisation that as \emph{a matter of logic} allows the distinctness (non-identity) of Hikaru and Pavel to entail the distinctness of Pavel and Hikaru.  

\section{Adding Identity}

Logicians resolve this issue by adding identity as a new \emph{logical} predicate – one of the structural words with a fixed interpretation.\footnote{We don't absolutely have to do this: there are logical languages in which identity is not a logical predicate, and is symbolised by just choosing a two-place predicate like $Ixy$. But in our logical language \FOL, we are choosing to treat identity as a structural word.} We add a new symbol `$=$', to clearly differentiate it from our existing predicates. 

The symbol `$=$' is a two-place predicate. Since it is to have a special meaning, we shall write it a bit differently: we put it between two terms, rather than out front. And it \emph{does} have a very particular fixed meaning. Like the quantifiers and connectives, the identity predicate does not need a symbolisation key to fix how it is to be used. Rather, it always gets the same interpretation: `$=$' always means `\gap{1} is identical to \gap{2}'. So identity is a special predicate because it has its meaning as part of logic.

That one thing is logically identical to another does not mean \emph{merely} that the objects in question are indistinguishable, or that all of the same things are true of them. When two things are alike in every respect, we may say they are \define{qualitatively identical}. This is the sense of identity involved in `identical twins', who are two distinct individuals who share their properties. In \FOL, the identity predicate represents not this relation of similarity, but a relation of absolute or \define{numerical identity}: there is only one, rather than two. This is the sense in which Lewis Carroll (the author of \emph{Alice in Wonderland}) is identical to Charles Lutwidge Dodgson (the Oxford mathematician): `they' are the very same person, with two different names.

This might seem odd. Identity is a relation, but it doesn't relate different things to each other: it relates everything to itself, and to nothing else. We need a predicate for that relation because the names and (especially) variables of \FOL\ aren't guaranteed to have different referents, and sometimes we want to explicitly require that two terms don't denote the same thing. For example, suppose we want to symbolise `Barry is the tallest person'. You might try `Barry is taller than everyone'. However, that would lead to the absurdity that Barry is taller than himself, since he is surely among `everyone'. So what we really need is `Barry is taller than everyone \emph{else}, i.e., everyone who's not (identical to) Barry', which is most naturally formulated using the identity predicate.


\section{Symbolising Identity Sentences}

Now suppose we want to symbolise this sentence:
\begin{earg}
\item[\ex{else2}] Pavel is Mister Checkov.
\end{earg}
Let us add to our symbolisation key:
	\begin{ekey}
		\item[c] Mister Checkov
	\end{ekey}
Now sentence \ref{else2} can be symbolised as `$p=c$'. This means that $p$ is $c$, and it follows that the thing named by `$p$' is the thing named by `$c$'.\footnote{One must be careful: the sentence `$p=c$' is, on this symbolisation, about Pavel and Mister Checkov; it is not about `Pavel' and `Mister Checkov', which are obviously distinct expressions of English.}

Let's return to our example `Barry is taller than everyone else'. We want to start with a paraphrase, like this: \emph{choose anyone from the domain; if they are not Barry, than Barry is taller than them}. Where $b$ symbolises `Barry' and $T$ symbolises `\gap{1} is taller than \gap{2}', we might symbolise this as: `$\forall x (\enot x = b \eif Tbx)$' (on the domain of people). 

Using that same kind of structure, we can also now deal with sentences \ref{else1b}–\ref{else1d}. All of these sentences can be  paraphrased as `Everyone who isn't Pavel is such that: Pavel owes money to them'. Paraphrasing some more, we get: `For all x, if x is not Pavel, then x is owed money by Pavel'. Now that we are armed with our new identity symbol, we can symbolise this as `$\forall x (\enot x = p \eif Opx)$'.

This last sentence contains the formula `$\enot x = p$'. And that might look a bit strange, because the symbol that comes immediately after the `$\enot$' is a variable, rather than a predicate. But this is no problem. We are simply negating the entire formula, `$x = p$'. But if this is confusing, you may use  the \define{non-identity predicate} `$≠$'. This is an abbreviation, characterised as folllows: \factoidbox{Any occurrence of the expression `$\meta{x}≠\meta{y}$' in a sentence abbreviates the expression `$\enot \meta{x} = \meta{y}$', and either can substitute for the other in any \FOL\ sentence.} I will use both expressions in what follows, but strictly speaking `$\enot a = b$' is the official version, and we allow a conventional abbreviation `$a≠b$'.

In addition to sentences that use the word `else', `other than' and `except', identity will be helpful when symbolising some sentences that contain the words `besides' and `only.' Consider these examples:

\begin{earg}
\item[\ex{else3}] No one besides Pavel owes money to Hikaru.
\item[\ex{else4}] Only Pavel owes Hikaru money.
\end{earg}
Sentence \ref{else3} can be paraphrased as, `No one who is not Pavel owes money to Hikaru'. This can be symbolised by `$\enot\exists x(\enot x = p \eand Oxh)$'. Equally, sentence \ref{else3} can be paraphrased as `for all x, if x owes money to Hikaru, then x is Pavel'. Then it can be symbolised as `$\forall x (Oxh \eif x = p)$'. Sentence \ref{else4} can be treated similarly.\footnote{But there is one subtlety here. Do either sentence \ref{else3} or \ref{else4} entail that Pavel himself owes money to Hikaru?}

\section{Principles of Identity} \label{lli}

Return to our argument from premises \ref{hikaru1} and \ref{hikaru2}. We now symbolise it: $\forall y (p≠y \eif Opy); h≠p \ttherefore Oph$. This argument is valid. Given the fixed interpretation we have assigned to `$=$', it is not possible that a first thing be not identical to a second, while the second is identical to the first. So we can conclude, as a matter of logical alone, that $h≠p$ is equivalent to $p≠h$. 

This argument rests on one of the logical principles governing identity: that if $a$ is $b$, then $b$ is $a$. This property is known as symmetry, and is one of a cluster of basic properties governing the structure of identity:
\factoidbox{\begin{itemize}
	\item  Identity is \define{reflexive}: everything is identical to itself. So for any meaningful name `\meta{a}', `$\meta{a}=\meta{a}$' is true.
		\item Identity is \define{symmetric}: if $\meta{a} = \meta{b}$,  then also $\meta{b} = \meta{a}$.
		\item Identity is \define{transitive}: if $\meta{a} = \meta{b}$,  and $\meta{b} = \meta{c}$, then also $\meta{a} = \meta{c}$.  
	\end{itemize}}
These principles of identity can be expressed as sentences of \FOL. In the definitions given above, the names involved are arbitrary. So we can in fact paraphrase the reflexivity of identity as saying that for any thing, it is self-identical. Symbolised in \FOL, this is `$\forall x x=x$'. Likewise for the others: \begin{itemize}
	\item Symmetry: $\forall x \forall y (x=y \eif y=x)$;
	\item Transitivity: $\forall x \forall y \forall z ((x=y \eand y=z) \eif x=z)$.
\end{itemize} 

These principles can apply to other two-place predicates too. For example, the two-place English predicate `\gap{1} is taller than \gap{2}' is also transitive, since if Albert is taller than Barbara, and Barbara is taller than Chloe, then Albert must be taller than Chloe too. But it is not reflexive or symmetric: Albert is not taller than himself, and if Albert is taller than Barbara, it cannot be also that Barbara is taller than Albert. We will return to this topic in §\ref{binary}.

A final principle about identity is \define{Leibniz' Law}, named after the philosopher and mathematician Gottfried Leibniz: \factoidbox{If $\meta{x}=\meta{y}$ then for any property at all,  $\meta{x}$ has it iff $\meta{y}$ has it too. That is: every instance of this schematic sentence of \FOL, for any predicate $\meta{F}$ whatsoever, is true: $$\forall x \forall y \bigl(x=y \eif (\meta{F}x \eiff \meta{F}y) \bigr).$$} 
Leibniz' Law certainly entails that identical things are indistinguishable, sharing every property in common. But as we have already noted, identity isn't merely indistinguishability. Two things might be indistinguishable, but if there are two, they are not strictly identical in the logical sense we are concerned with. Yet in many cases, even very similar things do turn out to have some distinguishing property. There is a significant philosophical controversy over whether there can be cases of mere numerical difference, i.e., of nonidentity without any qualitative dissimilarity. 




\section{There are at Least …} \label{s:atleast}

So far an identity predicate might seem useful in a few cases, like pseudonyms, but a bit niche. In fact, adding it to our language gives us the ability to say lots of things we cannot hope to say without it. In particular, the identity predicate gives us the ability to \emph{count} – to quantify our quantifiers, and say how many things there are of a particular kind. Indeed, it is tempting to argue that the concept of counting depends on the prior concept of numerical distinctness.

For example, consider these sentences:
\begin{earg}
\item[\ex{atleast1}] There is at least one apple
\item[\ex{atleast2}] There are at least two apples
\item[\ex{atleast3}] There are at least three apples
\end{earg}
We shall use the symbolisation key:
	\begin{ekey}
		\item[A] \gap{1} is an apple
	\end{ekey}
Sentence \ref{atleast1} does not require identity. It can be adequately symbolised by `$\exists x Ax$': There is some apple; perhaps many, but at least one.

It might be tempting to also translate sentence \ref{atleast2} without identity. Yet consider the sentence `$\exists x \exists y(Ax \eand Ay)$'. Roughly, this says that there is some apple $x$ in the domain and some apple $y$ in the domain. Since nothing precludes these from being one and the same apple, this would be true even if there were only one apple.\footnote{Note that both $\exists x Ax$ and $\exists y Ay$ are true in a domain with only one apple: the use of different variables doesn't require that different apples are the values of those variables.} In order to make sure that we are dealing with \emph{different} apples, we need an identity predicate. Sentence \ref{atleast2} needs to say that the two apples that exist are not identical, so it can be symbolised by `$\exists x \exists y(Ax \eand Ay \eand \enot x = y)$'.

Sentence \ref{atleast3} requires talking about three different apples. Now we need three existential quantifiers, and we need to make sure that each will pick out something different: `$\exists x \exists y\exists z(Ax \eand Ay \eand Az \eand  x ≠ y \eand  y ≠ z \eand x ≠ z)$'.

\section{There are at Most …}
Now consider these sentences:
\begin{earg}
	\item[\ex{atmost1}] There is at most one apple.
	\item[\ex{atmost2}] There are at most two apples.
\end{earg}
Sentence \ref{atmost1} can be paraphrased as, `It is not the case that there are at least \emph{two} apples'. This is just the negation of sentence \ref{atleast2}: 
$$\enot \exists x \exists y(Ax \eand Ay \eand \enot x = y).$$
But sentence \ref{atmost1} can also be approached in another way. It means that if you pick out an object and it's an apple, and then you pick out an object and it's also an apple, you must have picked out the same object both times. With this in mind, it can be symbolised by
$$\forall x\forall y\bigl( (Ax \eand Ay) \eif x=y\bigr).$$
The two sentences will turn out to be logically equivalent.

In a similar way, sentence \ref{atmost2} can be approached in two equivalent ways. It can be paraphrased as, `It is not the case that there are \emph{three} or more distinct apples', so we can offer:
$$\enot \exists x \exists y\exists z(Ax \eand Ay \eand Az \eand  x ≠ y \eand  y ≠ z \eand  x ≠ z).$$
Or, we can read it as saying that if you pick out an apple, and an apple, and an apple, then you will have picked out (at least) one of these objects more than once. Thus:
$$\forall x\forall y\forall z\bigl( (Ax \eand Ay \eand Az) \eif (x=y \eor x=z \eor y=z) \bigr).$$


\section{There are Exactly …}
Now we can symbolise `there are at least $n$' and we can symbolise `there are at most $n$'. Using them together, we can symbolise `there are exactly $n$':
\begin{earg}
\item[\ex{exactly1}] There is exactly one apple.
\item[\ex{exactly2}] There are exactly two apples.
\item[\ex{exactly3}] There are exactly three apples.
\end{earg}
Sentence \ref{exactly1} can be paraphrased as, `There is \emph{at least} one apple and there is \emph{at most} one apple'. This is just the conjunction of sentence \ref{atleast1} and sentence \ref{atmost1}. So we can offer:
$$\exists x Ax \eand \forall x\forall y\bigl( (Ax \eand Ay) \eif x=y\bigr).$$
But it is perhaps more straightforward to paraphrase sentence \ref{exactly1} as, `There is a thing x which is an apple, and everything which is an apple is just x itself'. Thought of in this way, we offer: 
$$\exists x\bigl(Ax \eand \forall y(Ay \eif x= y) \bigr).$$
Similarly, sentence \ref{exactly2} may be paraphrased as, `There are \emph{at least} two apples, and there are \emph{at most} two apples'. Thus we could offer 
$$\exists x \exists y(Ax \eand Ay \eand \enot x = y) \eand \forall x\forall y\forall z\bigl( (Ax \eand Ay \eand Az) \eif (x=y \eor x=z \eor y=z) \bigr).$$
More efficiently, though, we can paraphrase it as `There are at least two different apples, and every apple is one of those two apples'. Then we offer:
$$\exists x\exists y\bigl(Ax \eand Ay \eand \enot x = y \eand \forall z(Az \eif ( x= z \eor y = z) \bigr).$$
Finally, consider these sentence:
\begin{earg}
\item[\ex{exactly2things}] There are exactly two things.
\item[\ex{exactly2objects}] There are exactly two objects.
\end{earg}
It might be tempting to add a predicate to our symbolisation key, to symbolise the English predicate `\blank\ is a thing' or `\blank\ is an object'. But this is unnecessary. Words like `thing' and `object' do not sort wheat from chaff: they apply trivially to everything, which is to say, they apply trivially to every thing. So we can symbolise either sentence with either of the following:
	\begin{center}
		$\exists x \exists y \enot x = y \eand \enot \exists x \exists y \exists z (\enot x = y \eand \enot y = z \eand \enot x = z)$; or\\
		$\exists x \exists y \bigl(\enot x = y \eand \forall z(x=z \eor y = z) \bigr)$.
	\end{center}

\keyideas{
	\item Identity (`$=$') is the one logical predicate in \FOL, with a fixed interpretation.
	\item Identity satisfies a number of logical principles, the most important of which is Leibniz' Law, that when $x$ is $y$, then $x$ is \meta{F} iff $y$ is \meta{F}, for any predicate \meta{F}.
	\item Identity is crucial for symbolising numerical quantification: `there are at least $n$ \meta{F}s', `there are at most $n$ \meta{F}s' and `there are exactly $n$ \meta{F}s', because these all involve – tacitly – the notion of distinctness between things.
}

\practiceproblems

%\solutions
%\problempart
%\label{pr.FOLcandies}
%Using the following symbolisation key:
%\begin{ekey}
%\item[\domain] candies
%\item[C] \gap{1} has chocolate in it.
%\item[M] \gap{1} has marzipan in it.
%\item[S] \gap{1} has sugar in it.
%\item[T] Boris has tried \gap{1}.
%\item[B] \gap{1} is better than \gap{2}.
%\end{ekey}
%symbolise the following English sentences in \FOL:
%\begin{earg}
%\item Boris has never tried any candy.
%\item Marzipan is always made with sugar.
%\item Some candy is sugar-free.
%\item The very best candy is chocolate.
%\item No candy is better than itself.
%\item Boris has never tried sugar-free chocolate.
%\item Boris has tried marzipan and chocolate, but never together.
%%\item Boris has tried nothing that is better than sugar-free marzipan.
%\item Any candy with chocolate is better than any candy without it.
%\item Any candy with chocolate and marzipan is better than any candy that lacks both.
%\end{earg}



\problempart Explain why:
\begin{itemize}
\item   `$\exists x \forall y(Ay \eiff x= y)$' is a good symbolisation of `there is exactly one apple'.
\item `$\exists x \exists y \bigl(\enot x = y \eand \forall z(Az \eiff (x= z \eor y = z) \bigr)$' is a good symbolisation of `there are exactly two apples'.
\end{itemize}

\problempart Using the following symbolisation key:
\begin{ekey}
\item[\domain] all animals
\item[D] \gap{1} is a dog
\item[L] \gap{1} is larger than \gap{2}
\item[F] \gap{1} is fierce
\item[b] Bertie
\item[e] Emerson
\item[f] Fergus
\end{ekey}
symbolise the following sentences in \FOL:
\begin{earg}
\item Bertie is larger than all the other dogs.
\item Bertie, Emerson, and Fergus are all different dogs.
\item Emerson is smaller than at least two dogs.
\item The largest dog is not fierce.
\item One fierce dog is the same size as another fierce dog.
\end{earg}

\problempart
\label{pr.FOLcardsa}
Using the following symbolisation key:
\begin{ekey}
\item[\domain] cards in a standard deck
\item[B] \gap{1} is black.
\item[C] \gap{1} is a club.
\item[D] \gap{1} is a deuce.
\item[J] \gap{1} is a jack.
\item[M] \gap{1} is a man with an axe.
\item[O] \gap{1} is one-eyed.
\item[W] \gap{1} is wild.
\end{ekey}
symbolise each sentence in \FOL:
\begin{earg}
\item All clubs are black cards.
\item There are no wild cards.
\item There are at least two clubs.
\item There is more than one one-eyed jack.
\item There are at most two one-eyed jacks.
\item There are two black jacks.
\item There are four deuces.
\end{earg}

\problempart Using the following symbolisation key:
\begin{ekey}
\item[\domain] animals in the world
\item[B] \gap{1} is in Farmer Brown's field.
\item[H] \gap{1} is a horse.
\item[P] \gap{1} is a Pegasus.
\item[W] \gap{1} has wings.
\end{ekey}
symbolise the following sentences in \FOL:
\begin{earg}
\item There are at least three horses in the world.
\item There are at least three animals in the world.
\item There is more than one horse in Farmer Brown's field.
\item There are three horses in Farmer Brown's field.
\item There is a single winged creature in Farmer Brown's field; any other creatures in the field must be wingless.
\end{earg}

\problempart
Identity is a reflexive, symmetric, and transitive predicate. Can you give examples of English predicates which are \begin{earg}
	\item Reflexive and symmetric but not transitive;
	\item Reflexive and transitive but not symmetric;
	\item Symmetric and transitive but not reflexive?
\end{earg}


\chapter{Definite Descriptions}\label{subsec.defdesc}

In \FOL, names function rather like names in English. They are simply labels for the things they name, and may be attached arbitrarily, without any indication of the characteristics of what they name.\footnote{This is not strictly true: consider the name `Fido' which is conventionally the name of a dog. But even here the name doesn't carry any information in itself about what it names – the fact that we use that as a name only for dogs allows someone who knows that to reasonably infer that Fido is a dog. But `Fido is a dog' isn't a trivial truth, as it would be if somehow `Fido' carried with it the information that it applies only to dogs.} 

But complex noun phrases can also be used to denote particular things in English (recall §\ref{sec:singular_terms}), and they do so not merely by acting as arbitrary labels, but often by \emph{describing} the thing they refer to. Consider sentences like:
	\begin{earg}
		\item[\ex{traitor1}] Nick is \underline{the traitor}.
		\item[\ex{traitor2}] \underline{The traitor} went to Cambridge.
		\item[\ex{traitor3}] \underline{The traitor} is \underline{the deputy}.
		\item[\ex{traitor4}] \underline{The traitor} is \underline{the shortest person who went to Cambridge}.
	\end{earg}
These underlined noun phrases headed by `the' – `the traitor', `the deputy', `the shortest person who went to Cambridge' – are known as \define{definite descriptions}. They are meant to pick out a \emph{unique} object, by using a description which applies to that object and to no other (at least, to no other salient object). The class of possessive singular terms, such as `Antony's eldest child' or `Facebook's founder', might be subsumed into the class of definite descriptions. They can be paraphrased using definite descriptions: `the eldest child of Antony' or `the founder of Facebook'. 

Definite descriptions must be contrasted with \define{indefinite descriptions}, such as `\underline{A traitor} went to Cambridge', where no unique traitor is implied. Definite descriptions must also be contrasted with what we might call \emph{descriptive names}, such as `the Pacific Ocean'. While the Pacific Ocean is an ocean, it isn't reliably peaceful, and even when it is, it surely isn't the unique ocean that merits that description. These descriptive name uses might also be involved in cases of \define{generic} `the', such as in `\underline{The whale} is a mammal'. Here there is no implication that some specific whale is under discussion, but rather that the species is mammalian. (So maybe `the whale' is a complex name for the species.) In the generic use, `the whale is a mammal' can be paraphrased `whales are mammals'. But a genuine definite description, such as `\underline{the Prime Minister} is a Liberal' cannot be paraphrased as `Prime Ministers are Liberals'. The question we face is: can we adequately symbolise definite descriptions in \FOL?\footnote{There is another question that I don't address: can we come up with a good theory of the meaning of `the' in English that unifies how it behaves in `the whale is a mammal', `the Pacific Ocean is stormy', and `Ortcutt is the shortest spy'? That question is very hard. Our task is to offer a symbolisation, and as we've seen, a symbolisation needn't be a translation, but only needs to capture the relevant implications to be successful.}

\section{Treating Definite Descriptions as Terms}
One option would be to introduce new names whenever we come across a definite description. This is not a great idea. We know that \emph{the} traitor – whoever it is – is indeed \emph{a} traitor. We want to preserve that information in our symbolisation. So the symbolisation of `The traitor is a traitor' (or `the traitor is traitorous') should be a logical truth. But if we symbolise `the traitor' by a name $a$, the symbolisation will be $Ta$, which is not a logical truth.

A second option would be to introduce a \emph{new} definite description operator, such as `$\maththe$'. The idea would be to symbolise `the F' as `$\maththe xFx$'. This is taken to mean something like this `the unique thing such that it is F', which obviously involves a definite description in its semantics.  Expressions of the form $\maththe \meta{x} \meta{A}\meta{x}$ would then behave, grammatically speaking, like names – they combine with predicates to form sentences. Suppose we follow this path. Start with the following symbolisation key:
	\begin{ekey}
		\item[\domain] people
		\item[T] \gap{1} is a traitor
		\item[D] \gap{1} is a deputy
		\item[C] \gap{1} went to Cambridge
		\item[S] \gap{1} is shorter than \gap{2}
		\item[n] Nick
	\end{ekey}
We could symbolise sentence \ref{traitor1} with `$\maththe x Tx = n$' (`the thing which is a traitor is identical to Nick'), sentence \ref{traitor2} with `$C\maththe xTx$', sentence \ref{traitor3} with `$\maththe x Tx = \maththe x Dx$', and sentence \ref{traitor4} with `$\maththe x Tx = \maththe x (Cx \wedge \forall y((Cy \wedge x≠y) \eif Sxy))$'. This last example may be a bit tricky to parse. In semi-formal English, it says (supposing a domain of persons): `the unique person such that they are a traitor is identical with the unique person such that they went to Cambridge and they are shorter than anyone else who went to Cambridge'. 

However, even adding this new symbol to our language doesn't quite help with our initial complaint, since it is not self-evident that the symbolisation of `The traitor is a traitor' as `$T\maththe x Tx$' yields a logical truth. More seriously, the idea that all definite descriptions are to be treated as terms makes it more difficult to give a unified treatment of descriptions in predicate position. It would be desirable to give a unified treatment of `Ortcutt is a short spy' and `Ortcutt is the short spy'; but while the former might be symbolised as `$(So \eand Po)$', using the predicative `is', the latter would need to treated as `$o = \maththe x (Sx \eand Px)$', using the `is' of identity.

More practically, it would be nice if we didn't have to add a new symbol to \FOL. And indeed, we might be able to handle descriptions using what we already have.

\section{Russell's Paraphrase}
Bertrand Russell offered an influential account of definite descriptions that might serve our purposes. Very briefly put, he observed that, when we say `the \meta{F}', where that phrase is a definite description, our aim is to pick out the \emph{one and only} thing that is \meta{F} (in the appropriate contextually selected domain). Our discussion of counting in §\ref{sec.identity} gives us an idea about how Russell proposed to handle sentences expressing that there is exactly one \meta{F}.

Thus Russell offered a systematic paraphrase of sentences involving definite descriptions along these lines:\footnote{Bertrand Russell (1905) `On Denoting', \emph{Mind} \textbf{14}, pp.\ 479–93; see also Russell (1919) \emph{Introduction to Mathematical Philosophy}, London: Allen and Unwin, ch.\ 16.}
	\begin{align*}
		\text{the \meta{F} is \meta{G}  iff: }&\text{there is at least one \meta{F}, \emph{and}}\\
	&\text{there is at most one \meta{F}, \emph{and}}\\	
	&\text{every \meta{F} is \meta{G}}
\end{align*}
Note a very important feature of this paraphrase: \emph{`the' does not appear on the right-side of the equivalence.} This approach would allow us to paraphrase every sentence of the same form as the left hand side into a sentence of the same form as the right hand side, and thus `paraphrase away' the definite description. 

It is crucial to notice that we can handle each of the conjuncts on the right hand side of the equivalence in \FOL, using our techniques for dealing with numerical quantification.  We can deal with the three conjuncts on the right-hand side of Russell's paraphrase as follows:
	$$\exists x \meta{F}x \eand \forall x \forall y ((\meta{F}x \eand \meta{F}y) \eif x = y) \eand \forall x (\meta{F}x \eif \meta{G}x)$$
In fact, we could express the same point rather more crisply, by recognising that the first two conjuncts just amount to the claim that there is \emph{exactly} one \meta{F}, and that the last conjunct tells us that that object is \meta{G}. So, equivalently, we could offer this symbolisation of `The \meta{F} is \meta{G}':
	$$\exists x \bigl(\meta{F}x \eand \forall y (\meta{F}y \eif x = y) \eand \meta{G}x\bigr)$$
Using these sorts of techniques, we can now symbolise sentences \ref{traitor1}–\ref{traitor3} without using any new-fangled fancy operator, such as `$\maththe$'. 
\begin{itemize}
	\item Sentence \ref{traitor1} is exactly like the examples we have just considered. So we would symbolise it by `$\exists x (Tx \eand \forall y(Ty \eif x = y) \eand x = n)$'. 
\item Sentence \ref{traitor2} poses no problems either: `$\exists x (Tx \eand \forall y(Ty \eif x = y) \eand Cx)$'.
\item Sentence \ref{traitor3} is a little trickier, because it links two definite descriptions. But, deploying  Russell's paraphrase, it can be paraphrased by `something is such that: there is exactly one traitor and there is exactly one deputy and it is each of them'. So we can symbolise it by: $$\exists x\biggl(\bigl(Tx \eand \forall y(Ty \eif x = y) \bigr) \eand \bigl(Dx \eand \forall z(Dz \eif x = z) \bigr)\biggr).$$ Note that I have made sure that both uniqueness conditions are in the scope of the initial existential quantifier.
\end{itemize} Thus, we can adequately symbolise sentences involving definite descriptions in \FOL. Incidentally, Russell also offers an account of indefinite descriptions of the same general form, differing only in that he regards indefinite descriptions as lacking any connotation of uniqueness. (One of the exercises below, p.\ \pageref{indefs}, deals with that account.)

Let us dispel a worry. It seems that I can say `the table is brown' without implying that there is one and only one table in the universe. But doesn't Russell's paraphrase literally entail that there is only one table? Indeed it does – it entails that there is only one table \emph{in the domain under discussion}. While sometimes we explicitly restrict the domain, usually we leave it to our background conversational presuppositions to do so (recall §\ref{sec_domains}). If I can successfully say `the table is brown' in a conversation with you, for example, some background restriction on the domain must be in place that we both tacitly accept. For example, it might be that the prior discussion has focussed on your dining room, and so the implicit domain is \emph{things in that room}. In that case, `the table is brown' is true just in case there is exactly one table in that domain, and it is brown.



\section{The Structure of Definite Descriptions}

Russell offers his theory of definite descriptions as part of a campaign to effect `a reduction of all propositions in which denoting phrases occur to forms in which no such phrases occur' (`On Denoting', p.\ 482). (Russell thought there were paradoxes attendant to descriptions in English that could be avoided if we showed how they could be systematically eliminated.) We do not wish to attempt anything so radical as this kind of \emph{reductive analysis} – we only wanted to show that there is a way of symbolising definite descriptions in \FOL. That is, we only need to assume that Russell's proposal allows us to \emph{model} definite descriptions in \FOL, a language that lacks any native resources for expressing them. Officially, then, we will take no stand on whether Russell's analysis is correct. Officially, we only suggest that Russell's analysis is the best approach to symbolising English sentences involving definite descriptions in \FOL. 

Yet Russell's account has some nice features that predict and explain some otherwise puzzling features of English `the', and many logicians have followed Russell in thinking that the Russellian account might provide an adequate semantics for English definite descriptions. So in this section and in §\ref{strawson}, I cannot resist discussing some of the evidence for Russell's account of the English `the', and some of the major puzzles for that account. \emph{These two sections should be regarded as optional.}

\paragraph{Empty Definite Descriptions}
One of the nice features of Russell's paraphrase is that it allows us to handle \emph{empty} definite descriptions neatly. France has no king at present. Now, if we were to introduce a name, `$k$', to name the present King of France, then everything would go wrong: remember from §\ref{s:FOLBuildingBlocks} that a name must always pick out some object in the domain, and whatever actual domain we choose, it will contain no present King of France. So we are at a loss to understand `the King of France is bald': does it even say anything, since its subject has no referent?

Russell's paraphrase neatly avoids this problem. Russell tells us to treat definite descriptions using predicates and quantifiers, instead of names. Since predicates can be empty (see §\ref{s:MoreMonadic}), this means that no difficulty now arises when the definite description is empty. The sentence `the present King of France is bald' is paraphrased as `there exists exactly one present King of France and every present King of France is bald', and so turns out to be easily understood, and in fact to be false.

\paragraph{Scope and Descriptions}
Indeed, Russell's paraphrase helpfully highlights two ways one can go wrong with definite descriptions. To adapt an example from Stephen Neale,\footnote{Neale (1990) \emph{Descriptions}, Cambridge: MIT Press.} suppose I, Antony Eagle, claim:
	\begin{earg}
		\item[\ex{kingdate}] I am grandfather to the present king of France.
	\end{earg}
Using the following symbolisation key:
	\begin{ekey}
		\item[a] Antony
		\item[K] \gap{1} is a present king of France
		\item[G] \gap{1} is a grandfather of \gap{2}
	\end{ekey}
Sentence \ref{kingdate} would be symbolised by `$\exists x (\forall y(Ky \eiff  x = y) \eand Gax)$'. Now, suppose you don't think this sentence \ref{kingdate} is true. You might express your rejection by saying:
\begin{earg}
	\item[\ex{rejectk}] Antony isn't the grandfather of the present king of France.
 \end{earg} But your denial is \emph{ambiguous}. There are two available readings of \ref{rejectk}, corresponding to these two different sentences:
	\begin{earg}
		\item[\ex{outernegation}] There is no one who is both the present King of France and such that Antony is his grandfather.
		\item[\ex{innernegation}] There is a unique present King of France, but Antony is not his grandfather.
	\end{earg}
Sentence \ref{outernegation} might be paraphrased by `It is not the case that: Antony is a grandfather of the present King of France'. It will then be symbolised by `$\enot \exists x\bigl(Kx \eand \forall y(Ky \eif  x = y) \eand Gax \bigr)$'. We might call this \define{wide scope} negation, since the negation takes scope over the entire sentence. Note that this sentence is predicted to be true, because the embedded sentence contains an empty definite description.

Sentence \ref{innernegation} can be symbolised by `$\exists x (Kx \eand \forall y(Ky \eif x = y) \eand \enot Gax)$. We might call this \define {narrow scope} negation, since the negation occurs within the scope of the definite description. Note that its truth would require that there be a present King of France, albeit one who is a grandchild of Antony; so this sentence, unlike \ref{outernegation}, is predicted to be false.

These two disambiguations of your rejection \ref{rejectk} have different truth values, so don't mean the same thing. So there are two different reasons you could have for your rejection of my claim. Are you accepting that the definite description refers and denying what I said of the present king of France? Or are you denying that the definite description refers, rejecting a more basic assumption of what I said? 

The basic point is that the Russellian paraphrase provides two places in the symbolisation of \ref{rejectk} for the negation of `isn't' to fit: either taking scope over the whole sentence, or taking scope just over the predicate `$Gax$'. We see evidence that there are these two options in the ambiguity of \ref{rejectk}, and so we should opt for a semantics, like Russell's, which has the resources to handle these ambiguities – in this case, by positing a quantifier scope ambiguity like those we discussed in §\ref{quan.scope}.

The term-forming operator approach to definite descriptions cannot handle this contrast. There is just one symbolisation of the negation of \ref{kingdate} available in this framework: `$¬Ga\maththe x Kx$'. The original sentence is false, so this negation must be true. Since sentence \ref{innernegation} is false, this sentence does not express the inner negation of sentence \ref{kingdate}. But there is no way to put the negation elsewhere that will express the same claim as \ref{innernegation}. (`$Ga\maththe x ¬Kx$' clearly doesn't do the job – it says there is just one unique thing which isn't the present king of France, and Antony is grandfather to it!) So the sentence `Antony isn't grandfather to the present king of France' has only one correct symbolisation, and hence only one reading, if the operator approach to definite descriptions is correct. Since it is ambiguous, with multiple readings, the `$\maththe$'-operator approach cannot be correct – it doesn't provide a complex enough grammar for definite description sentences to allow for these kinds of scope ambiguities.


\section{The Adequacy of Russell's Paraphrase} \label{strawson}
The evidence from scope ambiguity we just considered is a substantial point in favour of Russell's paraphrase, at least compared to the operator approach. But how successful is Russell's paraphrase in general?  There is a substantial philosophical literature around this issue, but I shall content myself with two observations.

\paragraph{Presupposition and Negation}
One worry focusses on Russell's treatment of empty definite descriptions. If there are no \meta{F}s, then on Russell's paraphrase, both `the \meta{F} is \meta{G}' and its narrow scope negation, `the \meta{F} is not-\meta{G}', are false. Strawson suggested that such sentences should not be regarded as false, exactly.\footnote{P F Strawson (1950) `On Referring', \emph{Mind} \textbf{59} pp.\ 320–34.} Rather, they both seem to assume that `the \meta{F}' refers, and since this assumption is incorrect, the sentences misfire in a way that should, Strawson thinks, make us regard it as \emph{neither} true \emph{nor} false, but nevertheless still meaningful.

A \define{semantic presupposition} of a declarative sentence is something that must be \emph{taken for granted} by anyone asserting the sentence, triggered or forced by the words involved.\footnote{See David I Beaver, Bart Geurts, and Kristie Denlinger (2021) `Presupposition', in Edward N Zalta, ed., \emph{The Stanford Encyclopedia of Philosophy} \httpurl{plato.stanford.edu/archives/spr2021/entries/presupposition/}, esp. §2 and §6.} A pretty reliable test for whether $\meta{P}$ is a semantic presupposition of a sentence $\meta{A}$ is whether $\meta{P}$ is a consequence of both $\meta{A}$ and $¬\meta{A}$. Strawson elevates this test to a \emph{definition} of semantic presupposition: presuppositions are entailments that persist when a sentence is embedded under negation.  So `John has stopped drinking' and its negation `John hasn't stopped drinking' both entail in English that John used to drink, and hence `John used to drink' is a semantic presupposition of `John has stopped drinking'. Here the presupposition is triggered by the aspectual verb `stopped'. 

Strawson says that \define{presupposition failure} occurs when the presupposition of a sentence is false. If John never used to drink, both `John has stopped drinking' and `John hasn't stopped drinking' misfire. Strawson, following Frege, suggests that in cases of presupposition failure, a sentence is neither true nor false. 

In the case of definite descriptions, the Frege-Strawson view would say that `the present King of France is bald' presupposes that there is a present King of France. Since that presupposition fails, the sentence is neither true nor false. This is contrary to Russell's position that the sentence is false. 

With the notion of presupposition failure in hand, the Frege-Strawson theory seems to be able to address the scope evidence for Russell's account. For there is now a distinction between the denial involved in \ref{outernegation} and that involved in \ref{innernegation}, even though the logical form of the denied sentence is `$Ga\maththe x Kx$'. The logical negation of that sentence is `$¬Ga\maththe x Kx$', which shares the presupposition that there is a present King of France. But there is also another way of rejecting a sentence, one which targets not what was said by the sentence, but its presuppositions. This is sometimes called \define{metalinguistic negation}.\footnote{Larry Horn (1989) \emph{A Natural History of Negation}, University of Chicago Press.} One way of identifying its presence is the use of focal stress, emphasising the word to be targeted, and accompanied by an gloss explaining the presupposition to be rejected, as in:
\begin{earg}
\item[\ex{likelove}] I don't \emph{like} cricket, I love it!
\item[\ex{neverstarted}] John hasn't \emph{stopped} drinking; he never even started!
\item[\ex{metaling}] Sarah is \emph{not} the source of the leak; there was no leak!
\end{earg} The effects which Russell sees as the result of scope ambiguity are re-analysed as involving the contrast between ordinary negation in \ref{innernegation}, and metalinguistic negation in \ref{outernegation}. Crucially, on the Frege-Strawson view, the successful deployment of metalinguistic negation in cases like \ref{metaling} renders the sentence `Sarah is the source of the leak' neither true nor false.

The phenomenon of metalinguistic negation is real. But if we agree with Frege and Strawson here on how to model it semantically,  we shall need to revise our logic. For, in our logic, there are only two truth values (True and False), and every meaningful sentence is assigned exactly one of these truth values. Why? Suppose there is presupposition failure of `John has stopped drinking', because John never drank. Then `John has stopped drinking' can't be true since it entails something false, its presupposition `John drank'. And `John hasn't stopped drinking' can't be true, since it also entails that same falsehood. So neither can be true. Since one is the negation of the other, if either is false, the other is true. So neither can be false, either. So if there are nontrivial semantic presuppositions – semantic presuppositions that might be false – then we shall have to admit that some meaningful sentences with a false presupposition are neither true nor false. It remains an open question, admittedly, whether presuppositions in the ordinary and intuitive sense are really semantic presuppositions in \emph{this} sense. 

But there is room to disagree with Strawson. Strawson is appealing to some linguistic intuitions, but it is not clear that they are very robust. For example: isn't it just \emph{false}, not `gappy', that Antony is grandfather to the present King of France? (This is Neale's line.)

\paragraph{Misdescription}
Keith Donnellan raised a second sort of worry, which (very roughly) can be brought out by thinking about a case of mistaken identity.\footnote{Keith Donnellan (1966) `Reference and Definite Descriptions', \emph{Philosophical Review} \textbf{77}, pp.\ 281–304.} Two men stand in the corner: a very tall man drinking what looks like a gin martini; and a very short man drinking what looks like a pint of water. Seeing them, Malika says:
	\begin{earg}
		\item[\ex{gindrinker}] The gin-drinker is very tall!
	\end{earg}
Russell's paraphrase will have us render Malika's sentence as:
	\begin{earg}
		\item[\ref{gindrinker}$'$.] There is exactly one gin-drinker [in the corner], and whomever is a gin-drinker [in the corner] is very tall.
	\end{earg}
But now suppose that the very tall man is actually drinking \emph{water} from a martini glass; whereas the very short man is drinking a pint of (neat) gin. By Russell's paraphrase, Malika has said something false. But don't we want to say that Malika has said something \emph{true}? 

Again, one might wonder how clear our intuitions are on this case. We can all agree that Malika intended to pick out a particular man, and say something true of him (that he was tall). On Russell's paraphrase, she actually picked out a different man (the short one), and consequently said something false of him. But  maybe advocates of Russell's paraphrase only need to explain \emph{why} Malika's intentions were frustrated, and so why she said something false. This is easy enough to do:  Malika said something false because she had false beliefs about the men's drinks; if Malika's beliefs about the drinks had been true,  then she would have said something true.\footnote{Interested parties should read Saul Kripke (1977) `Speaker Reference and Semantic Reference', 1977 in French \emph{et al.}, eds., \emph{Contemporary Perspectives in the Philosophy of Language}, Minneapolis: University of Minnesota Press, pp.\ 6-27.}

To say much more here would lead us into deep philosophical waters. That would be no bad thing, but for now it would distract us from the immediate purpose of learning formal logic. So, for now, we shall stick with Russell's paraphrase of definite descriptions, when it comes to putting things into \FOL. It is certainly the best that we can offer, without significantly revising our logic. And it is quite defensible as an paraphrase. 

\keyideas{
	\item Definite descriptions like `the inventor of the zipper' are singular terms in English, but behave rather unlike names – for example, while the inventor of the zipper must be an inventor, Julius needn't be – even if Julius is the name of the inventor of the zipper.
	\item Russell's insight was if definite descriptions denote by uniquely describing, then we can use the ability of \FOL\ to symbolise sentences like `there is exactly one \meta{F} and it is \meta{G}' to represent definite descriptions.
	\item  There is an ongoing debate in linguistics about whether Russell's account captures the meaning of natural language definite descriptions, but there is no question that his approach is the only viable way to represent English descriptive noun phrases in \FOL. 
}


\practiceproblems

\problempart
Using the following symbolisation key:
\begin{ekey}
\item[\domain] people
\item[K] \gap{1} knows the combination to the safe.
\item[S] \gap{1} is a spy.
\item[V] \gap{1} is a vegetarian.
\item[T] \gap{1} trusts \gap{2}.
\item[h] Hofthor
\item[i] Ingmar
\end{ekey}
symbolise the following sentences in \FOL:
\begin{earg}
\item Hofthor trusts a vegetarian.
\item Everyone who trusts Ingmar trusts a vegetarian.
\item Everyone who trusts Ingmar trusts someone who trusts a vegetarian.
\item Only Ingmar knows the combination to the safe.
\item Ingmar trusts Hofthor, but no one else.
\item The person who knows the combination to the safe is a vegetarian.
\item The person who knows the combination to the safe is not a spy.
\end{earg}


\problempart
Using the following symbolisation key:
\begin{ekey}
\item[\domain] animals
\item[C] \gap{1} is a cat.
\item[G] \gap{1} is grumpier than \gap{2}.
\item[f] Felix 
\item[g] Sylvester
\end{ekey}
symbolise each of the following in \FOL:
\begin{earg}
\item  Some animals are grumpier than cats.
\item Some cat is grumpier than every other cat.
\item Sylvester is the grumpiest cat.
\item Of all the cats, Felix is the least grumpy.
\end{earg}

\problempart
\label{pr.FOLcards}
Using the following symbolisation key:
\begin{ekey}
\item[\domain] cards in a standard deck
\item[B] \gap{1} is black.
\item[C] \gap{1} is a club.
\item[D] \gap{1} is a deuce.
\item[J] \gap{1} is a jack.
\item[M] \gap{1} is a man with an axe.
\item[O] \gap{1} is one-eyed.
\item[W] \gap{1} is wild.
\end{ekey}
symbolise each sentence in \FOL:
\begin{earg}
\item The deuce of clubs is a black card.
\item One-eyed jacks and the man with the axe are wild.
\item If the deuce of clubs is wild, then there is exactly one wild card.
\item The man with the axe is not a jack.
\item The deuce of clubs is not the man with the axe.
\end{earg}


\problempart Using the following symbolisation key:
\begin{ekey}
\item[\domain] animals in the world
\item[B] \gap{1} is in Farmer Brown's field.
\item[H] \gap{1} is a horse.
\item[P] \gap{1} is a Pegasus.
\item[W] \gap{1} has wings.
\end{ekey}
symbolise the following sentences in \FOL:
\begin{earg}
\item The Pegasus is a winged horse.
\item The animal in Farmer Brown's field is not a horse.
\item The horse in Farmer Brown's field does not have wings.
\end{earg}

\problempart
In this section, I symbolised `Nick is the traitor' by `$\exists x (Tx \eand \forall y(Ty \eif x = y) \eand x = n)$'. Two equally good symbolisations would be:
	\begin{itemize}
		\item $Tn \eand \forall y(Ty \eif n = y)$
		\item $\forall y(Ty \eiff y = n)$
	\end{itemize}
Explain why these would be equally good symbolisations.

\problempart
Candace returns to her parents' home to find the family dog with a bandaged nose. Her mother says `the dog got into a fight with another dog', and this seems a perfectly appropriate thing to say in the circumstances. 

Does this example pose a problem for Russell's approach to definite descriptions?




\problempart
Some people have argued that the following two sentences are ambiguous. Are they? If they are, explain how this fact might be used to provide support to Russell's paraphrases of definite description sentences. \begin{earg}
	\item The prime minister has always been Australian.
	\item The number of planets is necessarily eight.
\end{earg}

\problempart
Russell's paraphrase of an indefinite description \label{indefs} sentence like `José met a man' is: \emph{there is at least one thing x such that x is male and human and and José met x}. 

Note that the word `is' has apparently two readings: sometimes, as in `Fido is heavy', it indicates \emph{predication}; sometimes, as in `Fido is Rover', it indicates identity (in the case, the same dog is known by two names). Something interesting arises if Russell's account of indefinite descriptions is right, since in an example like `Fido is a dog of unusual size', we might interpret the `is' in either way, roughly:
\begin{earg}
 \item Fido is identical to a dog of unusual size;
 \item Fido has the property of being a dog of unusual size.
 \end{earg} Suppose that `$U$' symbolises the property of being a dog of unusual size, then our two readings can be symbolised `$\exists x (Ux \wedge f=x)$' and `$Uf$'. 

 Is there any significant difference in meaning between these two symbolisations?  






\chapter{Sentences of \textnormal{\FOL}}\label{s:FOLSentences}
We know how to represent English sentences in \FOL. The time has finally come to properly define the notion of a \emph{sentence} of \FOL.

\section{Expressions}
There are six kinds of symbols in \FOL:

\begin{center}
\begin{tabular}{l l} \toprule 
Predicate symbols & $A,B,C,…,Z$\\
with subscripts, as needed & $A_{1}, B_{1},Z_{1},A_{2},A_{25},J_{375},…$\\
and the identity symbol &  $=$.
\\
Names & $a,b,c,…, r$\\
with subscripts, as needed & $a_{1}, b_{224}, h_7, m_{32},…$\\
\\
Variables & $s, t, u, v, w, x,y,z$\\
with subscripts, as needed & $x_{1}, y_{1}, z_{1}, x_{2},…$\\
\\
Connectives, of two types & \\
Truth-functional Connectives & $\enot,\eand,\eor,\eif,\eiff$\\
Quantifiers & $\forall, \exists$\\
\\
Parentheses &( , )\\
\bottomrule \end{tabular}
\end{center}
We define an \define{expression} of \FOL\ as any string of symbols of \FOL. Take any of the symbols of \FOL\ and write them down, in any order, and you have an expression.

\section{Terms and Formulae}\label{s:termsandf}
In §\ref{s:TFLSentences}, we went straight from the statement of the vocabulary of \TFL\ to the definition of a sentence of \TFL. In \FOL, we shall have to go via an intermediary stage: via the notion of a \emph{formula}. The intuitive idea is that a formula is any sentence, or anything which can be turned into a sentence by adding quantifiers out front. But this will take some unpacking.

We start by defining the notion of a term.
	\factoidbox{
		A \define{term} is any name or any variable.}
So, here are some terms:
	$$a, b, x, x_{1} x_{2}, y, y_{254}, z$$
We next need to define an \define{atom}.
	\factoidbox{
		\begin{enumerate}
		\item If $\meta{R}$ is any predicate other than the identity predicate `$=$', and we have zero or more terms $\meta{t}_{1}, \meta{t}_{2}, …, \meta{t_n}$ (not necessarily distinct from one another), then the expression $\meta{R t}_{1} \meta{t}_{2} … \meta{t}_n$ is an atom. 
		\item If $\meta{t}_{1}$ and $\meta{t}_{2}$ are terms, then the expression $\meta{t}_{1} = \meta{t}_{2}$ is an atom.
		\item Nothing else is an atom.
		\end{enumerate}
	}
The use of script fonts here follows the conventions laid down in §\ref{s:UseMention}. So, `$\meta{R}$' is not itself a predicate of \FOL. Rather, it is a symbol of our metalanguage (augmented English) that we use to talk about any predicate of \FOL. Similarly, `$\meta{t}_{1}$' is not a term of \FOL, but a symbol of the metalanguage that we can use to talk about any term of \FOL. So here are some atoms:
	\begin{center}
		$x = a$\\
		$a = b$\\
		$Fx$\\
		$Fa$\\
		$Gxay$\\
		$Gaaa$\\
		$Sx_{1} x_{2} a b y x_{1}$\\
		$Sby_{254} z a a z$\\
	\end{center}
Remember that we allow zero-place predicates too, to ensure that sentence letters of \TFL\ are grammatical expressions of \FOL\ too. According to the definition, any predicate symbol followed by no terms at all is also an atom of \FOL. So `$Q$' by itself is an acceptable atom.

Earlier, we distinguished many-place from one-place predicates. We made no distinction however in our list of acceptable symbols between predicate symbols with different numbers of places. This means that `$F$', `$Fa$', `$Fax$', and `$Faxb$' are all atoms. We will not introduce any device for explicitly indicating what number of places a predicate has. Rather, we will assume that in every atom of the form $\meta{A}\meta{t}_{1}…\meta{t}_{n}$, $\meta{A}$ denotes an $n$-place predicate. This means there is a potential for confusion in practice, if someone chooses to symbolise an argument using both the one-place predicate `$A$' and the two-place predicate `$A$'. Rather than forbid this entirely, we recommend choosing distinct symbols for different place predicates in any symbolisation you construct.

Once we know what atoms are, we can offer recursion clauses to define a \define{formula}. The first few clauses are exactly the same as for the definition of sentences in \TFL\ in §\ref{s.sentencesTFL}.
	\factoidbox{
	\begin{enumerate}
		\item Every atom is a formula. 
		\item If \meta{A} is a formula, then $\enot\meta{A}$ is a formula.
		\item If \meta{A} and \meta{B} are formulae, then $(\meta{A}\eand\meta{B})$ is a formula.
		\item If \meta{A} and \meta{B} are formulae, then $(\meta{A}\eor\meta{B})$ is a formula.
		\item If \meta{A} and \meta{B} are formulae, then $(\meta{A}\eif\meta{B})$ is a formula.
		\item If \meta{A} and \meta{B} are formulae, then $(\meta{A}\eiff\meta{B})$ is a formula.
		\item If \meta{A} is a formula and \meta{x} is a variable, then $\forall\meta{x}\meta{A}$ is a formula.
		\item If \meta{A} is a formula and \meta{x} is a variable, then $\exists\meta{x}\meta{A}$ is a formula.
		\item Nothing else is a formula.
	\end {enumerate}
	} % simplified here to permit e.g. $\forall x (Fx \eif \exists x Gx)$ as well-formed – the original definition says that Fx \eif \exists x Gx is a formula but its universal closure isn't! 
Here are some formulae:
	\begin{center}
		$Fx$\\
		$Gayz$\\
		$Syzyayx$\\
		$(Gayz \eif Syzyayx)$\\
		$\forall z (Gayz \eif Syzyayx)$\\
		$Fx \eiff \forall z (Gayz \eif Syzyayx)$\\
		$\exists y (Fx \eiff \forall z (Gayz \eif Syzyayx))$\\
		$\forall x \exists y (Fx \eiff \forall z (Gayz \eif Syzyayx))$\\
		$\forall x (Fx \eif \exists x Gx)$\\
		$\forall y (Fa \eif Fa) $\end{center}


We can now give a formal definition of scope, which incorporates the definition of the scope of a quantifier. Here we follow the case of \TFL; in \FOL, quantifiers are also considered to be connectives:
	\factoidbox{
		The \define{main connective} in a formula is the connective that was introduced last, when that formula was constructed using the recursion rules.
		
		\

		The \define{scope} of a connective in a formula is the subformula for which that connective is the main connective.
	}
So we can graphically illustrate the scope of the quantifiers in the last two examples thus:
	$$\overbracket{\forall x \overbracket{\exists y (Fx \eiff \overbracket{\forall z (Gayz \eif Syzyayx)}^{\text{scope of `}\forall z\text{'}}}^{\text{scope of `}\exists y\text{'}})}^{\text{scope of `}\forall x\text{'}} \qquad\qquad 
\overbracket{\forall x (Fx \eif \overbracket{\exists x Gx}^{\text{scope of `$\exists x$'}})}^{\text{scope of `$\forall x$'}}	$$

Note that it follows from our recursive definition that `$\forall x Fa$' is a formula. While puzzling on its face, with that quantifier governing no variable in its scope, it nevertheless is a formula of the language. It will turn out that, because the quantifier binds no variable in its scope, it is redundant; the formula `$\forall xFa$' is logically equivalent to `$Fa$'. Thus while `$\forall x Fa$' may be a bit puzzling, it is harmless to include it among our formulae. 

\section{Sentences in \FOL} \label{s:fol}
Recall that we are largely concerned in logic with assertoric sentences: sentences that can be either true or false. Many formulae are not sentences. Consider the following symbolisation key:
	\begin{ekey}
		\item[\domain] people
		\item[L] \gap{1} loves \gap{2}
		\item[b] Boris
	\end{ekey}
Consider the atom `$Lzz$'. All atoms are formulae, so `$Lzz$' is a formula. But can it be true or false? You might think that it will be true just in case the person named by `$z$' loves herself, in the same way that `$Lbb$' is true just in case Boris (the person named by `$b$') loves himself. \emph{But `$z$' is a variable, and does not name anyone or any thing.} It is true that we can sometimes manage to make a claim by saying `it loves it', the best Engish rendering of `$Lzz$'. But we can only do so by making use of contextual cues to supply referents for the pronouns `it' – contextual cues that artifical languages like \FOL\ lack. (If you and I are both looking at a bee sucking nectar on a flower, we might say `it loves it' to express the claim that the bee [\emph{it}]  loves the nectar [\emph{it}]. But we don't have such a rich environment to appeal to when trying to interpret formulae of \FOL.)

Of course, if we put an existential quantifier out front, obtaining `$\exists zLzz$', then this would be true iff someone loves themself (i.e., someone [$z$] is such that they [$z$] love themself [$z$]). Equally, if we wrote `$\forall z Lzz$', this would be true iff everyone loves themselves. The point is that, in the absence of an explicit introduction or contextual cues, we need a quantifier to tell us how to deal with a variable. 

Let's make this idea precise.
	\factoidbox{
		A \define{bound variable} is an occurrence of a variable \meta{x} that is within the scope of either $\forall\meta{x}$ or $\exists\meta{x}$. 
		
		A \define{free variable} is any variable that is not bound.
	}
For example, consider the formula
	$$\forall x(Ex \eor Dy) \eif \exists z(Ex \eif Lzx)$$
The scope of the universal quantifier `$\forall x$' is `$\forall x (Ex \eor Dy)$', so the first `$x$' is bound by the universal quantifier. However, the second and third occurrence of `$x$' are free. Equally, the `$y$' is free. The scope of the existential quantifier `$\exists z$' is `$\exists z(Ex \eif Lzx)$', so `$z$' is bound. 

In our last example from the previous section, `$\forall x (Fx \eif \exists xGx)$', the variable `$x$' in `$Gx$' is bound by the quantifier `$\exists x$', and so $x$ is free in `$Fx \eif \exists x Gx$' only when it appears in `$Fx$'. So while the scope of `$\forall x$' is the whole sentence, it nevertheless doesn't bind every variable in its scope – only those such that, were it absent, would be free. (So we might say an occurrence of a variable $\meta{x}$ is \define{bound by} an occurrence of quantifier $\forall \meta{x}/\exists \meta{x}$ just in case it would have been free had that quantifier been omitted.) 


Finally we can say the following.	
	\factoidbox{	
		A \define{sentence of \FOL} is a formula that contains no free variables.
	}
Since an atom formed by a zero-place predicate contains no terms at all, and hence cannot contain a variable, every such expression is a sentence – they are just the atomic sentences of \TFL. Any other formula which contains no variables, but only names, is also a sentence, as well as all those formulae which contain only bound variables. 

Our definition of a formula allows for examples like `$\exists x \forall x Fx$'. This is a sentence, since the variable in `$Fx$' is in the scope of a quantifier attached to `$x$'. But which one? It could make a difference whether the sentence is to be understood as saying everything is $F$, or something it. To resolve this issue, let us stipulate that a variable is bound by the quantifier which is the main connective of the smallest subformula in which the variable is bound. So in `$\exists x \forall x Fx$', the variable is bound by the universal quantifier, because it was already bound in the subformula `$\forall x Fx$'.


\section{Parenthetical Conventions}

We will adopt the same notational conventions governing parentheses that we did for \TFL\ (see §\ref{s:TFLSentences} and §\ref{s:MoreParentheticalConventions}.) 

First, we may omit the outermost parentheses of a formula, and we sometimes use variably sized parentheses to aid with readability.  

Second, we may omit parentheses between each pair of conjuncts when writing long series of conjunctions. 

Third, we may omit parentheses between each pair of disjuncts when writing long series of disjunctions.

\keyideas{
	\item There is a precise recursive definition of the notion of a sentence in \FOL, describing how they are built out of basic expressions.
	\item The distinction between a formula and a sentence is important; in a sentence, no variable occurs without an associated quantifier binding it. Sentences, unlike mere formulae, contain all the information needed to understand their variables, once interpreted.
}

\practiceproblems
\problempart
\label{pr.freeFOL}
Identify which variables are bound and which are free. Are any of these expressions formulas of \FOL? Are any of them sentences? Explain your answers.
\begin{earg}
\item $\forall x (Ax \eor (Cx \eif Bx))$
\item $\exists x (Lxy \eand \forall y Lyx)$
\item $(\forall x Ax \eand Bx)$
\item $(\forall x (Ax \eand Bx) \eand \forall y(Cx \eand Dy))$
\item $(\forall x (Ax \eand Bx) \eand \forall xy(Cx \eand Dy))$
\item $(\forall x\exists y\bigl(Rxy \eif (Jz \eand Kx) \bigr) \eor Ryx)$
\item $(\forall x_{1}(Mx_{2} \eiff Lx_{2}x_{1}) \eand \exists x_{2} Lx_{3}x_{2})$
\end{earg}

\problempart
Identify which of the following are (a) expressions of \FOL; (b) formulae of \FOL; and (c) sentences of \FOL.
\begin{earg}
	\item $\exists x (Fx \eif \forall y (Fx \eif \exists xGx))$; % sentence
	\item $\exists \neg x (Fx \wedge Gx)$; % expression
	\item $\exists x (Gx \wedge Gxx)$; % sentence
	\item $(Fx \eif \forall x (Gx \wedge Fx))$; % formula
	\item $(\forall x (Gx \wedge Fx) \eif Fx)$; % formula
	\item $(P \wedge (\forall x (Fx \eor P)))$; % expression
	\item $(P \wedge \forall x(Fx \eor P))$; % sentence
	\item $\forall x (Px \eand \exists y (x ≠ y))$. % expression
\end{earg}


