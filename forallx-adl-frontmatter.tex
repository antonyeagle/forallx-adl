%!TEX root = forallx-adl.tex
\begin{titlingpage}

~\\[4cm]
\begin{center}\forallxadl \end{center}\end{titlingpage}

\newpage \thispagestyle{empty} ~\\
~\\[2cm]
\noindent \makebox[6cm]{\begin{minipage}{7cm}
	 \begin{flushright}
	{\color{orange}\resizebox{!}{3em}{\forallx}}\par{\Huge\textsc{Adelaide\phantom{x}}}
\end{flushright}\end{minipage}}

\vfill


\noindent {Antony Eagle}\\
\emph{University of Adelaide}\\[1.5cm]

{Tim Button}\\
\emph{University College London}\\[1.5cm]

{P.D.\ Magnus}\\
\emph{University at Albany, State University of New York}
\newpage \thispagestyle{empty} ~\\
\copyright\ \ifthenelse{\year=2005}{\number\year}{2005–\number\year} by Antony Eagle, Tim Button and P.D.\ Magnus. Some rights reserved.

This version of \forallxadl\ is current as of \today.

\medskip\thispagestyle{empty}%

This book is a derivative work created by Antony Eagle, based upon Tim Button's 2016 Cambridge version of P.D.\ Magnus's \forallx\ (version 1.29). There are pervasive substantive changes in content, theoretical approach, coverage, and appearance. (For one thing, it's more than twice as long.)

You can find the most up to date version of \forallxadl\ at \httpurl{github.com/antonyeagle/forallx-adl}. 

The current version of \forallx\ Cambridge is available at \httpurl{github.com/OpenLogicProject/forallx-cam}, which has now also diverged noticeably from the 2016 version. Magnus' original is available at \httpurl{fecundity.com/logic}. 

This book, like its predecessors, is released under a Creative Commons license (Attribution 4.0).\medskip 

\begin{center}
	\includegraphics{cc-by.png} 
\end{center}

\begin{quote}
	{\small This work is licensed under the Creative Commons Attribution 4.0 International License. To view a copy of this license, visit \httpurl{creativecommons.org/licenses/by/4.0/} or send a letter to Creative Commons, PO Box 1866, Mountain View, CA 94042, USA.} 
\end{quote} \medskip


\vfill

Typesetting was carried out entirely in \XeLaTeX\ using the \texttt{Memoir} class. The body text is set in Constantia; math is set in Cambria Math; sans serif in Avenir Next; and monospaced text in Menlo. The style for typesetting proofs is based on \texttt{fitch.sty} (v0.4) by Peter Selinger, University of Ottawa.

\begin{quote}
	\emph{Kaurna miyurna, Kaurna yarta, ngai tampinthi.}\\
	Made on Kaurna land. Its sovereignty was never ceded.
\end{quote}

\newpage
\tableofcontents*\newpage
\listoffigures*

\chapter*{How to Use This Book}

This book has been designed for use in conjunction with the University of Adelaide courses \textsc{phil} 1110 \emph{Introduction to Logic} and \textsc{phil 1111ol} \emph{Introductory Logic}. But it is suitable for self-study as well. I have included a number of features to assist your learning. \begin{itemize}
	\item \forallx\ is divided into seven chapters, each further divided into sections and subsections. The sections are continuously numbered. \begin{itemize}
		\item Chapter \ref{ch.intro} gives an overview of how I understand the project of formal logic;
		\item Chapters \ref{ch.TFL}–\ref{ch.TruthTables} cover sentential or truth-functional logic;
		\item Chapters \ref{ch.FOL}–\ref{ch.semantics} cover quantified or predicate logic;
		\item and Chapters \ref{ch.NDTFL}–\ref{ch.NDFOL} cover the formal proof systems for our logical languages.
	\end{itemize}
	\item The book contains many cross-references to other sections. So a reference to `§\ref{s.sentencesTFL}' indicates that you should consult section 6, subsection 2 – you will find this on page \pageref{s.sentencesTFL}. Cross-references and entries in the table of contents are hyperlinked.
	\item  Figure~\ref{fig:depend} shows how the sections depend on one another. For example, the arrows coming from §\ref{s:Interpretations} in the diagram show that understanding that section requires familiarity with §\ref{c:entvalid} and §\ref{s:FOLSentences}, and also any sections on which they depend.
	\item Each chapter in the book concludes with a box labelled `Key Ideas in §$n$'. These are not a summary of the chapter, but contain some indication of what I regard as the main ideas that you should be taking away from your reading of the chapter. 
	\item Logical ideas and notation are pretty ubiquitous in philosophy, and there are a lot of different systems. We cannot cover all the alternatives, but some indication of other terminology and notation is contained in Appendix \ref{app.notation}.
	\item A quick reference to many of the aspects of the logical systems I introduce can be found in Appendix \ref{ch.qr}.
	\item When I first use a new piece of technical terminology, it is introduced by writing it in small caps, \textsc{like this}. The index of defined terms is Appendix \ref{app.index}.
	\item The book is the product of a number of authors. `I' doesn't always mean me, but `you' mostly means you, the reader, and `we' mostly means you and me.
\end{itemize}
I appreciate any comments or corrections: \href{mailto:antony.eagle@adelaide.edu.au}{\nolinkurl{antony.eagle@adelaide.edu.au}}, or you can propose changes at the book's repository: \httpurl{github.com/antonyeagle/forallx-adl}.

 \thispagestyle{empty}
\begin{figure}
{\footnotesize
\begin{tikzpicture}[sibling distance=8em]
 \matrix (depend)
 [
      matrix of nodes,
      column sep      = 3.5em,
      row sep         = 3.75ex,
      nodes={shape=ellipse,
    draw, anchor=center, inner sep=0.75ex,
    top color=white, bottom color=gray!20}
    ]
{
§\ref{s:Arguments} & & & & \\
§\ref{s:Valid} & §\ref{s:FOLBuildingBlocks} & & & \\ 
§\ref{s:BasicNotions} & §\ref{s:MoreMonadic} & §\ref{s:FOLSentences} & & \\
§\ref{s:firststeps} & §\ref{s:MultipleGenerality} & §\ref{s:Interpretations} & §\ref{s:NDVeryIdea}  & \\
§\ref{s:TFLConnectives} & §\ref{sec.identity} & §\ref{s:TruthFOL} & §\ref{c:ass} & \\
§\ref{s:TFLSentences} &  & §\ref{FOL.semantics} & §\ref{s:BasicTFLns} & \\
§\ref{s:UseMention} &  §\ref{subsec.defdesc}   & §\ref{sec.UsingModels} & §\ref{s:BasicTFLs} & §\ref{s:philosophy} \\
§\ref{s:TruthFunctionality} &  &  §\ref{s:All.Interp}   & §\ref{s:ProofTheoreticConcepts} & \\
§\ref{s:CompleteTruthTables} & 	 & §\ref{s:Derived} & §\ref{c:proof.strat} & \\
§\ref{s:Semantic.concepts} & 	 & §\ref{s:alternates} & §\ref{s:BasicFOL} & §\ref{s:CQ} \\
§\ref{c:entvalid} & §\ref{s:shortcuts} &    & §\ref{ch.identity} & \\
§\ref{s:expressiveness} & §\ref{s:PartialTruthTable} &   §\ref{s:nextsteps} & §\ref{sec:soundcomp} &  \\
    };
    \path[<-,thick]
\foreach \x/\y in {1/2, 2/3, 3/4, 4/5, 5/6, 6/7, 7/8, 8/9, 9/10, 10/11, 11/12} { (depend-\x-1) edge  (depend-\y-1) }
\foreach \x/\y in {2/3, 3/4, 4/5, 5/7} {(depend-\x-2) edge  (depend-\y-2) }
\foreach \x/\y in {3/4, 4/5, 5/6, 6/7, 7/8} {(depend-\x-3) edge  (depend-\y-3) }
\foreach \x/\y in {4/5, 5/6, 6/7, 7/8, 8/9, 9/10, 10/11, 11/12} { (depend-\x-4) edge  (depend-\y-4) }
(depend-11-1) edge (depend-11-2)
(depend-11-2) edge (depend-12-2)
(depend-12-4) edge [in=340,out=225] (depend-12-3)
(depend-7-1) edge [looseness=0.8] (depend-2-2)
(depend-5-2) edge [looseness=0.5] (depend-3-3)
(depend-7-4) edge (depend-7-5)
(depend-9-4) edge (depend-9-3)
(depend-9-3) edge (depend-10-3)
(depend-10-4) edge (depend-10-5)
(depend-8-3) edge [out=235,in=180] (depend-12-4) %looseness=1
(depend-8-1) edge [looseness=1.2]  (depend-4-4)
(depend-11-1) edge [out=46,in=244] (depend-4-3)
(depend-12-1) edge [in=235,out=340] (depend-12-3)
(depend-7-5) edge [in=305,out=315,looseness=1.2] (depend-12-3)
(depend-10-3) edge (depend-12-3)
(depend-7-2) edge [in=120,out=270] (depend-12-3);
	\end{tikzpicture}
	\caption{How the sections depend on one another. \label{fig:depend}}}
\end{figure}
 \thispagestyle{empty}
