%!TEX root = forallxadl.tex
\thispagestyle{empty}

~\\[4cm]
\begin{center}\forallx\ \textsc{adelaide}\end{center}
\newpage \thispagestyle{empty} ~\\
\newpage \thispagestyle{empty}

\noindent {\HUGE\forallx}



\noindent{\large\texttt{Adelaide}}

\vfill


\noindent {P.D.\ Magnus}\\
\emph{University at Albany, State University of New York}\\[2cm]

{Tim Button}\\
\emph{University of Cambridge}\\[2cm]

{Antony Eagle}\\
\emph{University of Adelaide}



\newpage
\thispagestyle{empty}%

{\copyright\ \ifthenelse{\year=2005}{\number\year}{2005–\number\year} by P.D.\ Magnus, Tim Button, and Antony Eagle. Some rights reserved.

This version of \forallx-Adelaide is current as of \today.

\medskip

This book is a derivative work created by Antony Eagle, based upon Tim Button's 2016 Cambridge version of P.D.\ Magnus's \forallx\ (version 1.29). There are pervasive changes, big and small, substantive and cosmetic. You can find the most up to date version of this work at \href{https://github.com/antonyeagle/forallx-adl}{\nolinkurl{github.com/antonyeagle/forallx-adl}}. 

The most up-to-date version of \forallx\ Cambridge is available at \href{https://github.com/OpenLogicProject/forallx-cam}{\nolinkurl{github.com/OpenLogicProject/forallx-cam}} – this has diverged noticeably from the 2016 version by now. Magnus' original is available at \href{http://fecundity.com/logic}{\nolinkurl{fecundity.com/logic}}. 

This book, like its predecessors, is released under a Creative Commons license (Attribution 4.0).\medskip 

\begin{center}
	\includegraphics{cc-by.png} 
\end{center}

\begin{quote}
	{\small This work is licensed under the Creative Commons Attribution 4.0 International License. To view a copy of this license, visit \href{https://creativecommons.org/licenses/by/4.0/}{\nolinkurl{creativecommons.org/licenses/by/4.0/}} or send a letter to Creative Commons, PO Box 1866, Mountain View, CA 94042, USA.} 
\end{quote} \medskip
\vfill

Typesetting was carried out entirely in \XeLaTeX. The body text is set in Constantia; math is set in Cambria Math; sans serif in Calibri; and monospaced text in Consolas. The style for typesetting proofs is based on \texttt{fitch.sty} (v0.4) by Peter Selinger, University of Ottawa.

\newpage
\tableofcontents*

\chapter*{How to use this book}

This book has been designed for use in conjunction with the University of Adelaide course \textsc{phil} 1110 \emph{Introduction to Logic}. But it is suitable for self-study as well. I have included a number of features to assist your learning. \begin{itemize}
	\item \forallx\ is divided into seven chapters, each further divided into sections and subsections. The sections are continuously numbered. \begin{itemize}
		\item Chapter \ref{ch.intro} gives an overview of how I understand the project of formal logic;
		\item Chapters \ref{ch.TFL}–\ref{ch.TruthTables} cover sentential or truth-functional logic;
		\item Chapters \ref{ch.FOL}–\ref{ch.semantics} cover quantified or predicate logic;
		\item and Chapters \ref{ch.NDTFL}–\ref{ch.NDFOL} cover the formal proof systems for our logical languages.
	\end{itemize}
	\item The book contains many cross-references to other sections. So a reference to `§\ref{s.sentencesTFL}' indicates that you should consult section 6, subsection 2 – you will find this on page \pageref{s.sentencesTFL}. Cross-references are hyperlinked, as are entries in the table of contents.
	\item The diagram in Figure~\ref{fig:depend} shows how the sections depend on one another. A few sections don't have any later sections that depend on them; if you are studying this material on your own, you may wish to skim these sections on a first reading.
	\item Logical ideas and notation are pretty ubiquitous in philosophy, and there are a lot of different systems. We cannot cover all the alternatives, but some indication of other terminology and notation is contained in Appendix \ref{app.notation}.
	\item A quick reference to many of the aspects of the logical systems I introduce can be found in Appendix \ref{ch.qr}.
	\item When I first use a new piece of technical terminology, it is introduced by writing it in small caps, \textsc{like this}. You can find an index of defined terms in Appendix \ref{app.index}.
	\item Each chapter in the book concludes with a box labelled `Key Ideas in §$n$'. These are not a summary of the chapter, but contain some indication of what I regard as the main ideas that you should be taking away from your reading of the chapter. 
	\item The book is the product of a number of authors. `I' doesn't always mean me, but `you' mostly means you, the reader, and `we' mostly means you and me.
\end{itemize}
I appreciate any comments or corrections: \href{mailto:antony.eagle@adelaide.edu.au}{\nolinkurl{antony.eagle@adelaide.edu.au}}.

\begin{figure} \begin{center}
\begin{tikzpicture}[sibling distance=10em]
 \matrix (depend)
 [
      matrix of nodes,
      column sep      = 3em,
      row sep         = 5ex,
      nodes={shape=ellipse,
    draw, anchor=center, inner sep=1ex,
    top color=white, bottom color=gray!20}
    ]
{
§\ref{s:Arguments} & & & & \\
§\ref{s:Valid} & §15 & & & \\ 
§\ref{s:BasicNotions} & §16 & §\ref{s:Interpretations} & & \\
§\ref{s:firststeps} & §17 & §\ref{s:TruthFOL} & §\ref{s:NDVeryIdea}  & \\
§\ref{s:TFLConnectives} & §18 & §\ref{FOL.semantics} & §\ref{c:ass} & \\
§\ref{s:TFLSentences} & §19 & §\ref{sec.UsingModels} & §\ref{s:BasicTFLns} & \\
§\ref{s:UseMention} &     & §\ref{s:All.Interp} & §\ref{s:BasicTFLs} & §\ref{s:philosophy} \\
§8 & §25 &     & §\ref{s:ProofTheoreticConcepts} & \\
§9 & 	 & §\ref{s:Derived} & §\ref{c:proof.strat} & \\
§10 & 	 & §\ref{s:alternates} & §35 & §36 \\
§11 & §12 &    & §37 & \\
§14 & §13 &    & §38 & \\
    };
	\end{tikzpicture}
\end{center}
	\caption{Dependency of the sections on one another. \label{fig:depend}}
\end{figure}
