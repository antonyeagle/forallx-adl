%!TEX root = forallxadl.tex
\thispagestyle{empty}

~\\[4cm]
\begin{center}\forallx\ \textsc{adelaide}\end{center}
\newpage \thispagestyle{empty} ~\\
\newpage \thispagestyle{empty}

\noindent {\HUGE\forallx}



\noindent{\large\texttt{Adelaide}}

\vfill


\noindent {P.D.\ Magnus}\\
\emph{University at Albany, State University of New York}\\[2cm]

{Tim Button}\\
\emph{University of Cambridge}\\[2cm]

{Antony Eagle}\\
\emph{University of Adelaide}



\newpage
\thispagestyle{empty}%

{\copyright\ \ifthenelse{\year=2005}{\number\year}{2005–\number\year} by P.D.\ Magnus, Tim Button, and Antony Eagle. Some rights reserved.

This version of \forallx-Adelaide is current as of \today.

\medskip

This book is a derivative work created by Antony Eagle, based upon Tim Button's 2016 Cambridge version of P.D.\ Magnus's \forallx\ (version 1.29). There are pervasive changes, big and small, substantive and cosmetic. You can find the most up to date version of this work at \href{https://github.com/antonyeagle/forallx-adl}{\nolinkurl{github.com/antonyeagle/forallx-adl}}. 

The most up-to-date version of \forallx\ Cambridge is available at \href{https://github.com/OpenLogicProject/forallx-cam}{\nolinkurl{github.com/OpenLogicProject/forallx-cam}} – this has diverged noticeably from the 2016 version by now. Magnus' original is available at \href{http://fecundity.com/logic}{\nolinkurl{fecundity.com/logic}}. 

This book, like its predecessors, is released under a Creative Commons license (Attribution 4.0).\medskip 

\begin{center}
	\includegraphics{cc-by.png} 
\end{center}

\begin{quote}
	{\small This work is licensed under the Creative Commons Attribution 4.0 International License. To view a copy of this license, visit \href{https://creativecommons.org/licenses/by/4.0/}{\nolinkurl{creativecommons.org/licenses/by/4.0/}} or send a letter to Creative Commons, PO Box 1866, Mountain View, CA 94042, USA.} 
\end{quote} \medskip
\vfill

Typesetting was carried out entirely in \XeLaTeX. The body text is set in Constantia; math is set in Cambria Math; sans serif in Calibri; and monospaced text in Consolas. The style for typesetting proofs is based on \texttt{fitch.sty} (v0.4) by Peter Selinger, University of Ottawa.