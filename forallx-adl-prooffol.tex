%!TEX root = forallx-adl.tex
\part{Natural Deduction for \FOL}
\label{ch.NDFOL}
\addtocontents{toc}{\protect\mbox{}\protect\hrulefill\par}

\chapter{Basic Rules for \textnormal{\FOL}}\label{s:BasicFOL}

\FOL\ makes use of all of the connectives of \TFL. So proofs in \FOL\ will use all of the basic and derived rules from chapter \ref{ch.NDTFL}. We shall also use the proof-theoretic notions (particularly, the symbol `$\proves$') introduced in that chapter. 

Some arguments in \FOL\ don't need any new rules. Consider this: $$\enot(\forall x Px \eor \exists y Py) \ttherefore \enot \forall x Px.$$ \begin{proof}
	\hypo{a}{\enot(\forall x Px \eor \exists y Py)}
	\open
	\hypo{b}{\forall x Px}
	\have{e}{(\forall x Px \eor \exists y Py)}\oi{b}
	\have{ar}{\enot(\forall x Px \eor \exists y Py)}\by{R}{a}
	\close
	\have{c}{\enot \forall x Px}\ni{b-e,b-ar}
\end{proof}

However, not every \FOL\ sentence has a \TFL\ connective as its main connective. So we will also need some new basic rules to govern the quantifiers, and to govern the identity sign, to deal with those sentences where the main connective is a quantifier and where the sentence is an identity predication.


\section{Universal Elimination}\label{unielim}

Holding fixed the claim that everything is F, you can conclude that any particular thing is F. \emph{You name it; it's F.} The same is true for many-place predicates: if every human is shorter than 3km tall, then Amy is shorter than 3km tall, and Bob is, and Jonquil is,  and everyone else you can name.

Accordingly, the following reasoning should be fine for the corresponding symbolisations in \FOL:
\begin{proof}
	\hypo{a}{\forall xRxxd}
	\have{c}{Raad} \Ae{a}
\end{proof}
We obtained line 2 by dropping the universal quantifier and replacing every instance of `$x$' with `$a$'. Equally, the following should be allowed:
\begin{proof}
	\hypo{a}{\forall xRxxd}
	\have{c}{Rddd} \Ae{a}
\end{proof}
We obtained line 2 here by dropping the universal quantifier and replacing every instance of `$x$' with `$d$'. We could have done the same with any other name we wanted. 

This motivates the \define{universal elimination} rule ($\forall$E), using the notion for uniform substitution we introduced in §\ref{fol.truth.quant}:
\factoidbox{
\begin{proof}
	\have[m]{a}{\forall \meta{x}\meta{A}}
	\have[\ ]{}{\vdots}
	\have[\ ]{c}{\meta{A}\subs{\meta{c}}{\meta{x}}} \Ae{a}
\end{proof}
Where \meta{c} can be any name.}
The intent of the rule is that you can obtain any \emph{substitution instance} of a universally quantified formula: replace every occurrence of the free variable \meta{x} in \meta{A} with any chosen name. (If there are any – the rule is also good when \meta{A} has no free variable, because then the quantifier $\forall\meta{x}$ is redundant.) Remember here that the expression `\meta{c}' is a metalanguage variable over names: you are not required to replace the variable \meta{x} by the \FOL\ name `$c$', but you can select any name you like!

I should emphasise that (as with every elimination rule) you can only apply the $\forall$E rule when the universal quantifier is the main connective. Thus the following is outright banned:
\begin{proof}
	\hypo{a}{(\forall x Bx \eif Bk)}
	\have{c}{(Bb \eif Bk)}\by{naughtily attempting to invoke $\forall$E}{a}
\end{proof}
This is illegitimate, since `$\forall x$' is not the main connective in line 1. (If you need a reminder as to why this sort of inference should be banned, reread §\ref{s:MoreMonadic}.)

Here is an example of the rule in action. Suppose we wanted to show that $\forall x \forall y (Rxx \eif Rxy), Raa  \ttherefore Rab$ is provable. The proof might go like this: \begin{proof}
	\hypo{a}{\forall x \forall y (Rxx \eif Rxy)}
	\hypo{b}{Raa}
	\have{c}{\forall y (Raa \eif Ray)}\Ae{a}
	\have{d}{Raa \eif Rab}\Ae{c}
	\have{e}{Rab}\ce{d,b}
\end{proof}



\section{Existential Introduction}\label{exint}
Given the assumption that some specific named thing is an F, you can conclude that something is an F: `Sylvester reads, so someone reads' seems like a conclusive argument. So we ought to allow the inference from a claim about some particular thing being F, to a general claim that something or other is F:
\begin{proof}
	\hypo{a}{Raad}
	\have{b}{\exists x Raax} \Ei{a}
\end{proof}
Here, we have replaced the name `$d$' with a variable `$x$', and then existentially quantified over it. Equally, we would have allowed:
\begin{proof}
	\hypo{a}{Raad}
	\have{c}{\exists x Rxxd} \Ei{a}
\end{proof}
Here we have replaced both instances of the name `$a$' with a variable, and then existentially generalised. 

There are some pitfalls with this description of what we have done. The following argument is invalid: `Someone loves Alice; so someone is such that someone loves themselves'. So we ought not to be able to conclude `$\exists x \exists x Rxx$' from `$\exists x Rxa$'. Accordingly, our rule cannot be \emph{replace a name by a variable, and stick a corresponding quantifier out the front} – since that would would permit the proof of the invalid argument.

We take our cue from the $\forall$E rule. This rule says: take a sentence $\forall\meta{xA}$, then we can remove the quantifier and substitute an arbitrary name for some free variable in the formula \meta{A} (assuming there is one). The $\exists$I rule is in some sense a mirror image of this rule: it allows us to move from a sentence with an arbitrary name – that might be thought of as the result of substituting a name for a free variable in some formula \meta{A} – to a quantified sentence $\exists \meta{xA}$. So here is how we formulate our rule of \define{existential introduction}:
\factoidbox{
	\begin{proof}
		\have[m]{c}{\meta{A}\subs{\meta{c}}{\meta{x}}} 
		\have[\ ]{}{\vdots}
			\have[\ ]{a}{\exists \meta{x}\meta{A}} \Ei{c}
	\end{proof}
} So really we should think that the proof just above should be thought of as concluding $\exists x Rxxd$ from `$Rxxd$'$\subs{a}{x}$.

If we have this rule, we cannot provide a proof of the invalid argument. For `$\exists x Rxa$' is not a substitution instance of `$\exists x \exists x Rxx$' – both instances of `$x$' in `$Rxx$' are bound by the second existential quantifier, so neither is free to be substituted. So the premise is not of the right form for the rule of $\exists$I to apply. 

On the other hand, this proof is correct:
\begin{proof}
	\hypo{a}{Raa}
	\have{b}{\exists x Rax}\Ei{a}
\end{proof} Why? Because the assumption `$Raa$' is in fact not only a substitution instance of $\exists x Rxx$, but also a substitution instance of `$\exists x Rax$', since `$Rax$'$\subs{a}{x}$ is just `$Raa$' too. So we can vindicate the intuitively correct argument `Narcissus loves himself, so there is someone who loves Narcissus'. 

As we just saw, applying this rule requires some skill in being able to recognise substitution instances. Thus the following is allowed:
\begin{proof}
	\hypo{a}{Raad}
	\have{d}{\exists x Rxad} \Ei{a}
	\have{e}{\exists y \exists x Rxyd} \Ei{d}
\end{proof} This is okay, because `$Raad$' can arise from substitition of `$a$' for `$x$' in `$Rxad$', and `$\exists x Rxad$' can arise from substitition of `$a$' for `$y$' in `$\exists x Rxyd$'.
But this is banned:
\begin{proof}
	\hypo{a}{Raad}
	\have{d}{\exists x Rxad} \Ei{a}
	\have{e}{\exists x \exists x Rxxd}\by{naughtily attempting to invoke $\exists$I}{d}
\end{proof} This is because `$\exists x Rxad$' is not a substitution instance of `$\exists x \exists x Rxxd$', since (again) both occurrences of `$x$' in `$Rxxd$' are already bound and so not available for free substitution.

Here is an example which shows our two proof rules in action, a proof that this argument is correct: $$\forall x \forall y (Rxy \eand Ryx) \ttherefore \exists x Rxx,$$ \begin{proof}
	\hypo{a}{\forall x \forall y (Rxy \eand Ryx)}
	\have{b}{\forall y (Ray \eand Rya)}\Ae{a}
	\have{c}{(Raa \eand Raa)}\Ae{b}
	\have{d}{Raa}\ae{c}
	\have{e}{\exists x Rxx}\Ei{d}
\end{proof}

For another example, consider this proof of `$\exists x (Px \eor \enot Px)$' from no assumptions:\phantomsection\label{exexmid}
\begin{proof}
	\open\hypo{a}{\enot(Pd \eor \enot Pd)}
	\open\hypo{b}{\enot Pd}
	\have{c}{(Pd \eor \enot Pd)}\oi{b}
	\have{ar}{\enot(Pd \eor \enot Pd)}\by{R}{a}
	\close
	\have{c1}{Pd}\ne{b-c,b-ar}
	\have{e}{(Pd \eor \enot Pd)}\oi{c1}
	\have{arr}{\enot(Pd \eor \enot Pd)}\by{R}{a}
	\close
	\have{f}{(Pd \eor \enot Pd)}\ne{a-e,a-arr}
	\have{con}{\exists x (Px \eor \enot Px)}\Ei{f}
\end{proof}


\section{Empty Domains}
The following proof combines our two new rules for quantifiers:
	\begin{proof}
		\hypo{a}{\forall x Fx}
		\have{in}{Fa}\Ae{a}
		\have{e}{\exists x Fx}\Ei{in}
	\end{proof}
Could this be a bad proof? If anything exists at all, then certainly we can infer that something is F, from the fact that everything is F. But what if \emph{nothing} exists at all? Then it is surely vacuously true that everything is F; however, it does not following that something is F, for there is nothing to \emph{be} F. So if we claim that, as a matter of logic alone, `$\exists x Fx$' follows from `$\forall x Fx$', then we are claiming that, as a matter of \emph{logic alone}, there is something rather than nothing. This might strike us as a bit odd.

Actually, we are already committed to this oddity. In §\ref{s:FOLBuildingBlocks}, we stipulated that domains in \FOL\ must have at least one member. We then defined a logical truth (of \FOL) as a sentence which is true in every interpretation. Since `$\exists x\ x=x$' will be true in every interpretation, this \emph{also} had the effect of stipulating that it is a matter of logic that there is something rather than nothing.

Since it is far from clear that logic should tell us that there must be something rather than nothing, we might well be cheating a bit here. 

If we refuse to cheat, though, then we pay a high cost. Here are three things that we want to hold on to:
	\begin{itemize}
		\item $\forall x Fx \proves Fa$: after all, that was $\forall$E.
		\item $Fa \proves \exists x Fx$: after all, that was $\exists$I.
		\item the ability to copy-and-paste proofs together: after all, reasoning works by putting lots of little steps together into rather big chains.
	\end{itemize}
If we get what we want on all three counts, then we have to countenance that $\forall xFx \proves \exists x Fx$. So, if we get what we want on all three counts, the proof system alone tells us that there is something rather than nothing. And if we refuse to accept that, then we have to surrender one of the three things that we want to hold on to!

Before we start thinking about which to surrender,\footnote{Many will opt for restricting $\exists$I. If we permit an empty domain, we will also need `empty names' – names without a referent. When the name \meta{c} is empty, it seems problematic to conclude from `$\meta{c}$ is F' that there is something which is F. (Does `Santa Claus drives a flying sleigh' entail `Someone drives a flying sleigh'?) But empty names are not cost-free; understanding how a name that doesn't name anything can have any meaning at all has vexed many philosophers and linguists.} we might want to ask how \emph{much} of a cheat this is. Granted, it may make it harder to engage in theological debates about why there is something rather than nothing. But the rest of the time, we will get along just fine. So maybe we should just regard our proof system (and \FOL, more generally) as having a very slightly limited purview. If we ever want to allow for the possibility of \emph{nothing}, then we shall have to cast around for a more complicated proof system. But for as long as we are content to ignore that possibility, our proof system is perfectly in order. (As, similarly, is the stipulation that every domain must contain at least one object.)


\section{Universal Introduction}\label{uniint}
Suppose you had shown of each particular thing that it is F (and that there are no other things to consider). Then you would be justified in claiming that everything is F. This would motivate the following proof rule. If you had established each and every single substitution instance of `$\forall x Fx$', then you can infer `$\forall x Fx$'. 

Unfortunately, that rule would be utterly unusable. To establish each and every single substitution instance would require proving `$Fa$', `$Fb$', $…$, `$Fj_2$', $…$, `$Fr_{79002}$', $…$, and so on. Indeed, since there are infinitely many names in \FOL, this process would never come to an end. So we could never apply that rule. We need to be a bit more cunning in coming up with our rule for introducing universal quantification. 

Our cunning thought will be inspired by considering:
$$\forall x Fx \ttherefore\ \forall y Fy$$
This argument should \emph{obviously} be valid. After all, alphabetical variation in choice of variables ought to be a matter of taste, and of no logical consequence. But how might our proof system reflect this? Suppose we begin a proof thus:
\begin{proof}
	\hypo{x}{\forall x Fx} 
	\have{a}{Fa} \Ae{x}
\end{proof}
We have proved `$Fa$'. And, of course, nothing stops us from using the same justification to prove `$Fb$', `$Fc$', $…$, `$Fj_2$', $…$, `$Fr_{79002}, …$, and so on until we run out of space, time, or patience. But reflecting on this, we see that this is a way to prove $F\meta{c}$, for any name \meta{c}. And if we can do it for \emph{any} thing, we should surely be able to say that `$F$' is true of \emph{everything}. This therefore justifies us in inferring `$\forall y Fy$', thus:
\begin{proof}
	\hypo{x}{\forall x Fx}
	\have{a}{Fa} \Ae{x}
	\have{y}{\forall y Fy} \Ai{a}
\end{proof}
The crucial thought here is that `$a$' was just some \emph{arbitrary} name. There was nothing special about it – we might have chosen any other name – and still the proof would be fine. And this crucial thought motivates the universal introduction rule ($\forall$I):
\factoidbox{
\begin{proof}
	\have[m]{a}{\meta{A}\subs{\meta{c}}{\meta{x}}}
	\have[\ ]{}{\vdots}
	\have[\ ]{c}{\forall \meta{x}\meta{A}} \Ai{a}
\end{proof}
	\meta{c} must not occur in any undischarged assumption, or elsewhere in \meta{A}}

A crucial aspect of this rule, though, is bound up in the accompanying constraint. In English, a name like `Sylvester' can play two roles: it can be introduced as a name for a specific thing (`let me dub thee Sylvester'!), or as an \emph{arbitrary name}, introduced by this sort of stipulation: \emph{let `Sylvester' name some arbitrarily chosen man}. The name doesn't tell us, as it subsequently appears, whether it was introduced in one way or the other. But if it was introduced as an arbitrary name, then any conclusions we draw about this Sylvester aren't really dependent on the particular arbitrarily chosen referent – they all depend rather on the stipulation used in introducing the name, and so (specifically) they will all be consequences of the only fact we know for sure about this Sylvester, that he is male. If all men are mortal, then an arbitrarily chosen man, whom we temporarily call `Sylvester', is mortal. If Sylvester is mortal, then there is a date he will die. But since he was selected arbitrarily, without reference to any further particulars of his life, then for any man, there exists a date he will die. And that is appropriate reasoning from a universal generalisation, to another generalisation, via claims about a specific but arbitrarily chosen person.\footnote{The details about how this sort of arbitrary reference works are interesting. A controversial but nevertheless attractive view of how it might work is  Wylie Breckenridge and Ofra Magidor (2012) `Arbitrary Reference', \emph{Philosophical Studies} \textbf{158}, pp.\ 377–400.}

We don't have stipulations to introduce a name as an arbitrary name in \FOL. But we do have a way of ensuring that the name has no prior associations other than those linked to a prior universal generalisation, if we insist that,  when the name is about to be eliminated from the proof, no assumption about what that name denotes is being relied on. That way, we can know that however it was introduced to the proof, it was not done in a way that involved making specific assumptions about whatever the name arbitrarily picks out. The simplest way for this to happen is that the name was introduced by an application of $\forall$E, as an arbitrary name in the standard sense. But there are other ways too.  

 This constraint ensures that we are always reasoning at a sufficiently general level. To see the constraint in action, consider this terrible argument:
	\begin{quote}
		Everyone loves Kylie Minogue; therefore everyone loves themselves.
	\end{quote}
We might symbolise this obviously invalid inference pattern as:
$$\forall x Lxk \ttherefore \forall x Lxx$$
Now, suppose we tried to offer a proof that vindicates this argument:
\begin{proof}
	\hypo{x}{\forall x Lxk}
	\have{a}{Lkk} \Ae{x}
	\have{y}{\forall x Lxx} \by{naughtily attempting to invoke $\forall$I}{a}
\end{proof}\noindent
This is not allowed, because `$k$' occurred already in an undischarged assumption, namely, on line 1. The crucial point is that, if we have made any assumptions about the object we are working with (including assumptions embedded in \meta{A} itself), then we are not reasoning generally enough to license  the use of $\forall$I.

Although the name may not occur in any \emph{undischarged} assumption, it may occur as a discharged assumption. That is, it may occur in a subproof that we have already closed. For example:
\begin{proof}
	\open
		\hypo{f1}{Gd}
		\have{f2}{Gd}\by{R}{f1}
	\close
	\have{ff}{Gd \eif Gd}\ci{f1-f2}
	\have{zz}{\forall z(Gz \eif Gz)}\Ai{ff}
\end{proof}
This tells us that `$\forall z (Gz \eif Gz)$' is a \emph{theorem}. And that is as it should be. 

Here is another proof featuring an application of $\forall$I after discharging an assumption about some name `$a$':
\begin{proof}
	\open
	\hypo{a}{Fa \eand \enot Fa}
	\have{b}{Fa}\ae{a}
	\have{c}{\enot Fa}\ae{a}
	\close
	\have{d}{\enot(Fa \eand \enot Fa)}\ni{a-b,a-c}
	\have{e}{\forall x \enot(Fx \eand \enot Fx)}\Ai{d}
\end{proof} Here we were able to derive that something could not be true of $a$, no matter what $a$ is. We cannot make a coherent assumption that $a$ is both $F$ and isn't $F$, so it doesn't really matter what `$a$' denotes. So the open sentence `$\enot(F \eand \enot Fx)$' could not be true of anything at all. That is why we are entitled to discharge that assumption, and then any subsequent use of `$a$' in the proof must be depending not on particular facts about this $a$, but about anything at all, including whatever it is that `$a$' happens to pick out. 

You might wish to recall the proof of `$\exists x (Px \eor \enot Px)$' from page \pageref{exexmid}. Note that, by the second-last line, we had already discharged any assumption which relied on the specific name chosen (in that case, `$d$'). The existential introduction rule has no constraints on it, so that it was not necessary to discharge any assumptions using the name before applying that rule. But we see now that, since those assumptions were in fact discharged, we could have applied universal introduction at that second last line, to yield a proof of `$\forall x (Px \eor \enot Px)$'.

We can also use our universal rules together to show some things about how quantifier order doesn't matter, when the strings of quantifiers are of the same type. For example $$\forall x \forall y \forall z Syxz \ttherefore \forall z \forall y \forall x Syxz$$ can be proved as follows: \begin{proof}
	\hypo{a}{\forall x \forall y \forall z Syxz}
	\have{b}{\forall y \forall z Syaz}\Ae{a}
	\have{c}{\forall z Sbaz}\Ae{b}
	\have{d}{Sbac}\Ae{c}
	\have{e}{\forall x Sbxc}\Ai{d}
	\have{f}{\forall y \forall x Syxc}\Ai{e}
	\have{g}{\forall z \forall y \forall x Syxz}\Ai{f}
\end{proof} Here we successively eliminate the quantifiers in favour of arbitrarily chosen names, and then reintroduce the quantifiers (though in a different order). The only undischarged assumption throughout the proof is the first line, with no names at all, so all of the uses of universal introduction are acceptable.


\section{Existential Elimination}\label{exelim}
Suppose we know that \emph{something} is F. The problem is that simply knowing this does not tell us which thing is F. So it would seem that from `$\exists x Fx$' we cannot immediately conclude `$Fa$', `$Fe_{23}$', or any other substitution instance of the sentence. What can we do?

Suppose we know that something is F, and that everything which is F is G. In (almost) natural English, we might reason thus:
	\begin{quote}
		Since something is F, there is some particular thing which is an F. We do not know anything about it, other than that it's an F, but for convenience, let's call it `obbie'. So: obbie is F. Since everything which is F is G, it follows that obbie is G. But since obbie is G, it follows that something is G. And nothing depended on which object, exactly, obbie was. So, something is G.
	\end{quote}
We might try to capture this reasoning pattern in a proof as follows:
\begin{proof}
	\hypo{es}{\exists x Fx}
	\open\hypo{ast}{\forall x(Fx \eif Gx)}
	\open
		\hypo{s}{Fo}
		\have{st}{Fo \eif Go}\Ae{ast}
		\have{t}{Go} \ce{st, s}
		\have{et1}{\exists x Gx}\Ei{t}
	\close
	\have{et2}{\exists x Gx}\Ee{es,s-et1}
\end{proof}\noindent
Breaking this down: we started by writing down our assumptions. At line 3, we made an additional assumption: `$Fo$'. This was just a substitution instance of `$\exists x Fx$'. On this assumption, we established `$\exists x Gx$'. But note that we had made no \emph{special} assumptions about the object named by `$o$'; we had \emph{only} assumed that it satisfies `$Fx$'. So nothing depends upon which object it is. And line 1 told us that \emph{something} satisfies `$Fx$'. So our reasoning pattern was perfectly general. We can discharge the specific assumption `$Fo$', and simply infer `$\exists x Gx$' on its own.

Putting this together, we obtain the existential elimination rule ($\exists$E):
\factoidbox{
\begin{proof}
	\have[m]{a}{\exists \meta{x}\meta{A}}
	\open	
		\hypo[i]{b}{\meta{A}\subs{\meta{c}}{\meta{x}}}
		\have[\ ]{}{\vdots}
		\have[j]{c}{\meta{B}}
	\close
	\have[\ ]{}{\vdots}
	\have[\ ]{d}{\meta{B}} \Ee{a,b-c}
\end{proof}
\meta{c} must not occur in any assumption undischarged before line $i$\\
\meta{c} must not occur in $\exists \meta{x}\meta{A}$\\
\meta{c} must not occur in \meta{B}}
As with universal introduction, the constraints are extremely important. To see why, consider the following terrible argument:
	\begin{quote}
		Tim Button is a lecturer. There is someone who is not a lecturer. So Tim Button is both a lecturer and not a lecturer.
	\end{quote}
We might symbolise this obviously invalid inference pattern as follows:
$$Lb, \exists x \enot Lx \ttherefore Lb \eand \enot Lb$$
Now, suppose we tried to offer a proof that vindicates this argument:
\begin{proof}
	\hypo{f}{Lb}
	\hypo{nf}{\exists x \enot Lx}	
	\open	
		\hypo{na}{\enot Lb}
		\have{con}{Lb \eand \enot Lb}\ae{f, na}
	\close
	\have{econ1}{Lb \eand \enot Lb}\by{naughtily attempting to invoke $\exists$E }{nf, na-con}
\end{proof}
The last line of the proof is not allowed. The name that we used in our substitution instance for `$\exists x \enot Lx$' on line 3, namely `$b$', occurs in line 4. And the following proof would be no better:
\begin{proof}
	\hypo{f}{Lb}
	\hypo{nf}{\exists x \enot Lx}	
	\open	
		\hypo{na}{\enot Lb}
		\have{con}{Lb \eand \enot Lb}\ae{f, na}
		\have{con1}{\exists x (Lx \eand \enot Lx)}\Ei{con}		
	\close
	\have{econ1}{\exists x (Lx \eand \enot Lx)}\by{naughtily attempting to invoke $\exists$E }{nf, na-con1}
\end{proof}
The last line of the proof would still not be allowed. For the name that we used in our substitution instance for `$\exists x \enot Lx$', namely `$b$', occurs in an undischarged assumption, namely line 1. 

The moral of the story is this. \factoidbox{If you want to squeeze information out of an existentially quantified claim $\exists\meta{x}\meta{A}$, choose a \emph{new} name, never before used in the proof, to substitute for the variable in \meta{A}.} That way, you can guarantee that you meet all the constraints on the rule for $\exists$E.

An argument that makes use of both patterns of arbitrary reasoning is this example due to Breckenridge and Magidor: `from the premise that there is someone who loves everyone to the conclusion that everyone is such that someone loves them'. Here is a proof in our system, letting `$Lxy$' symbolise `\gap{1} loves \gap{2}', letting `$h$' symbolise the arbitrarily chosen name `Hiccup' and letting `$a$' symbolise the arbitrarily chosen name `Astrid': \begin{proof}
	\hypo{a}{\exists x \forall y Lxy}
	\open
		\hypo{b}{\forall y Lhy}
		\have{f}{Lha}\Ae{b}
		\have{e}{\exists x Lxa}\Ei{f}
		\have{d}{\forall y \exists x Lxy}\Ai{e}
	\close
	\have{c}{\forall y \exists x Lxy}\Ee{a,b-d}
\end{proof} At line 3, both our arbitrary names are in play – $h$ was newly introduced to the proof in line 2 as the arbitrary person Hiccup who witnesses the truth of `$\exists x \forall y Lxy$', and `$a$' at line 3 as an arbitrary person Astrid beloved by Hiccup. We can apply $\exists$I without restriction at line 4, which takes the name `$h$' out of the picture – we no longer rely on the specific instance chosen, since we are back at generalities about someone who loves everyone, being such that they also love the arbitrarily chosen someone Astrid. So we can safely apply $\forall$I at line 5, since the name `$a$' appears in no assumption nor in `$\forall y\exists x Lxy$'. But now we have at line 5 a claim that doesn't involve the arbitrary name `$h$' either, which was newly chosen to not be in any undischarged assumption or in $\exists x \forall y Lxy$. So we can safely say that the name Hiccup was just arbitrary, and nothing in the proof of `$\forall y\exists x Lxy$' depended on it, so we can discharge the specific assumption about $h$ that was used in the course of that proof and nevertheless retain our entitled ment to `$\forall y\exists x Lxy$'. 

\section{The Barber}\label{barber}

Let's try something a little more complicated, the so-called `Barber paradox': \begin{quote}
	 in a certain remote Sicilian village, approached by a long ascent up a precipitous mountain road, the barber shaves all and only those villagers who do not shave themselves. Who shaves the barber? If he himself does, then he does not (since he shaves only those who do not shave themselves); if he does not, then he indeed does (since he shaves all those who do not shave themselves).  The unacceptable supposition is that there is such a barber – one who shaves himself if and only if he does not. The story may have sounded acceptable: it turned our minds, agreeably enough, to the mountains of inland Sicily. However, once we see what the consequences are, we realize that the story cannot be true: there cannot be such a barber, or such a village. The story is unacceptable.\footnote{R M Sainsbury (2009) \emph{Paradoxes} 3rd ed., Cambridge University Press, pages 1–2.}
\end{quote} This uses some of our tricky quantifer rules, disjunction elimination (proof by cases) and negation introduction (\emph{reductio}), so it is really a showcase of many things we've learned so far. 

Let's first try to symbolise the argument. \begin{ekey}
	\item[\text{Domain}] residents of a certain remote Sicilian village
	\item[B]\gap{1} is a barber
	\item[S]\gap{1} shaves \gap{2}
\end{ekey}
The argument then revolves around the claim that there is a barber who shaves everyone who doesn't shave themselves. Semi-formally paraphrased: someone x exists such that x is a barber and for all people y: y does not shave themselves iff x shaves y. That is: $$\exists x (Bx \eand \forall y (\enot Syy \eiff Sxy)).$$ The argument takes the form of a \emph{reductio}, so we will begin the proof by assuming this claim for the sake of argument and see what happens: \begin{proof}
	\open
	\hypo{a}{\exists x (Bx \eand \forall y (\enot Syy \eiff Sxy))}
	\open
	\hypo{b}{(Ba \eand \forall y (\enot Syy \eiff Say))}
	\have{c}{\forall y (\enot Syy \eiff Sxy)}\ae{b}
	\have{d}{(\enot Saa \eiff Saa)}\Ae{c}
	\open
		\hypo{i}{\enot(P \eand \enot P)}
		\open 
		\hypo{e}{\enot Saa}
		\have{f}{Saa}\be{d,e}
		\have{g}{\enot Saa}\by{R}{e}
		\close
	\have{h}{Saa}\ne{e-f,e-g}
	\have{k}{\enot Saa}\be{d,h}
		\close
	\have{l}{(P \eand \enot P)}\ne{i-h,i-k}
	\close
	\have{m}{(P \eand \enot P)}\Ee{a,b-l}
	\have{n}{P}\ae{m}
	\have{o}{\enot P}\ae{m}
	\close
	\have{p}{\enot \exists x (Bx \eand \forall y (\enot Syy \eiff Sxy))}\ni{a-n,a-o}
\end{proof}
One trick to this proof is to be sure to instantiate the universally quantified claim at line 3 by using the same name `$a$' as was already used in line 2. This is because, intuitively, the problem case for this supposed barber arises when you think about whether they shaves themselves or not. But themselves trickiest part of this proof occurs at lines 5–11. By line 4, we've already derived a contradictory biconditional. But if we just use it to derive `$Saa$' and `$\enot Saa$', the contradictory claims we obtain would end up involving the name `$a$'. That would mean we couldn't apply the $\exists$E rule, since the final line of the subproof would contain the chosen name, so we couldn't get our logical falsehood out of the subproof beginning on line 2, and hence could perform the desired \emph{reductio} on line 1 via {\enot}I. So our trick is to suppose the negation of an \emph{unrelated} logical falsehood on line 5, derive the logical falsehood from line 4 in the range of that assumption, and hence use {\enot}E to derive the logical falsehood `$P \eand \enot P$' on line 11. This doesn't contain the name `$a$', and hence can be extracted from the subproof to show that line 1 by itself suffices to derive a logical falsehood, and that shows the supposition that there is such a barber is a logical falsehood.


\section{Justification of these Quantifier Rules}

Above, I offered informal arguments for each of our quantifier rules that seem to exemplify the pattern of argument in the rule, and to be intuitively valid. But we can also offer justifications for our rules in terms of interpretations of the sentences involved, and the principles governing truth of quantified sentences introduced in §\ref{fol.truth.quant}. 

For example, consider any interpretation which makes $\forall\meta{x}\meta{A}$ true. In any such interpretation, there will be a nonempty domain, and every name will denote some member of this domain. $\forall\meta{x}\meta{A}$ is true just in case for any name we like, it will denote something of which $\meta{A}\subs{\meta{c}}{\meta{x}}$ is true. So in any such interpretation, for each name in the language \meta{c}, $\meta{A}\subs{\meta{c}}{\meta{x}}$ will also be true. So the proof rule of $\forall$E corresponds to a valid argument form.

For $\exists$E, the case is only a little more involved. Suppose $\exists\meta{x}\meta{A}$ is true in an interpretation. Then there is some interpretation, otherwise just like the original one, in which some new name $\meta{c}$ is assigned to some object in the domain, and where $\meta{A}\subs{\meta{c}}{\meta{x}}$ is true. Suppose that, in fact, every interpretation which makes $\meta{A}\subs{\meta{c}}{\meta{x}}$ true also makes  $\meta{B}$ true, where the new name \meta{c} does not appear in \meta{B}. Could \meta{B} be false in our original interpretation? No – for everything that appears in \meta{B} is already interpreted in the original interpretation, with the same interpretation as in the interpretation which makes it true. So it must be true in our original interpretation too. So $\meta{C}_{1},…,\meta{C}_{n},\exists\meta{x}\meta{A}\entails \meta{B}$ (when the name \meta{c} makes no appearance in any sentence in this argument), and the proof rule of $\exists$E corresponds to a valid argument form.

You may also offer arguments from intepretations to the effect that our other quantifier proof rules correspond to valid arguments in \FOL:\begin{itemize}
	\item $\meta{A}\subs{\meta{c}}{\meta{x}} \entails \exists \meta{x}\meta{A}$ – if \meta{c} is used in the proof, it must have an interpretation as something in the domain, and so something in the domain satisfies \meta{A};
	\item If this entailment holds: $$\meta{C}_{1},…,\meta{C}_{n} \entails \meta{A}\subs{\meta{c}}{\meta{x}},$$ where the name \meta{c} occurs nowhere among $\meta{C}_{i}$ or elsewhere in \meta{A}, then this entailment also holds: $$\meta{C}_{1},…,\meta{C}_{n} \entails \forall\meta{x}\meta{A}.$$ For we could have substituted any other name for \meta{c} and the original entailment would still have succeeded, since it could not have depended on the specific name chosen. So it doesn't matter what the interpretation of \meta{c} happens to be, and if that doesn't matter, it must be because everything is \meta{A}.
\end{itemize} 


So we are again comforted: our proof rules can never lead us from true assumptions to false claims, if correctly applied.

\section{Proof-Theoretic Concepts in \FOL}

The notion of provability of \meta{A} from undischarged assumptions $\meta{C}_{1},…,\meta{C}_{n}$ was introduced for \TFL\ in §\ref{s:ProofTheoreticConcepts}. The rules for \FOL\ are different, but the earlier definitions go through unchanged, once we remember that the notion of proof in \FOL\ means: \emph{provable using the rules for \TFL\ and those for \FOL}. 

So in what follows I will make use of the single turnstile `$\proves$' to mean that there is a correctly formed proof using only the rules of \TFL\ and \FOL. (And once we introduce the identity rules in §\ref{ch.identity}, I will tacitly assume that proofs can make use of those rules too in justifying a claim using `$\proves$'.) 

Likewise, the notions of theoremhood, provable equivalence, and joint contrariness all carry over their definitions, because they are defined in terms of the single turnstile.

\keyideas{
	\item We augment our natural deduction proof system for \TFL\ by adding rules governing quantifiers to go partway towards a natural deduction system for \FOL.
	\item The rules for  $\forall$E and $\exists$I are straightforward and can be applied regardless of which names we deploy.
	\item But the other quantifier rules $\forall$I and $\exists$E  contain some important restrictions on which names we can use. These restrictions are motivated by considerations about arbitrary reference which inform us when we can introduce `dummy names' in the course of our proofs and what we can do with them.
	\item Our proof rules match the interpretation of \FOL\ we have given – they will not permit us to say that some claim is provable from some assumptions when that claim isn't entailed by those assumptions. 
	\item The notation `$\proves$' for provability carries over from its earlier use in \TFL\ unchanged, once we understand that a proof can now use the new rules for the new logical connectives in \FOL.
}


\practiceproblems
\problempart
The following three `proofs' are \emph{incorrect}. Explain why they are incorrect. If the argument `proved' is invalid, provide an interpretation which shows that the assumptions involved do not entail the conclusion:
\begin{enumerate}
	\item 
	\begin{proof}
		\hypo{Rxx}{\forall x Rxx}
		\have{Raa}{Raa}\Ae{Rxx}
		\have{Ray}{\forall y Ray}\Ai{Raa}
		\have{Rxy}{\forall x \forall y Rxy}\Ai{Ray}
	\end{proof}
\item	\begin{proof}
		\hypo{AE}{\forall x \exists y Rxy}
		\have{E}{\exists y Ray}\Ae{AE}
		\open
			\hypo{ass}{Raa}
			\have{Ex}{\exists x Rxx}\Ei{ass}
		\close
		\have{con}{\exists x Rxx}\Ee{E, ass-Ex}
	\end{proof}
\item \begin{proof}
	\open\hypo{a}{\exists y \enot(Ty \eor \enot Ty)}
	\open
	\hypo{b}{\enot(Td \eor \enot Td)}
	\open
	\hypo{c}{Td}
	\have{d}{Td \eor \enot Td}\oi{c}
	\have{e}{\enot(Td \eor \enot Td)}\by{R}{b}
	\close
	\have{f}{\enot Td}\ne{c-d,c-e}
	\have{g}{(Td \eor \enot Td)}\oi{f}
	\have{h}{\enot(Td \eor \enot Td)}\by{R}{b}
	\have{c2}{((Td \eor \enot Td)\eand \enot(Td \eor \enot Td))}\ai{g,h}
	\close
	\have{c1}{((Td \eor \enot Td)\eand \enot(Td \eor \enot Td))}\Ee{a,b-c2}
	\close
	\have{con}{\enot\exists y \enot(Ty \eor \enot Ty)}\ni{a-c1}
\end{proof}
\end{enumerate}
\newpage
\problempart 
\label{pr.justifyFOLproof}
The following three proofs are missing their commentaries (rule and line numbers). Add them, to turn them into bona fide proofs. 
\begin{multicols}{2}\begin{proof}
\hypo{p1}{\forall x\exists y(Rxy \eor Ryx)}
\open\hypo{p2}{\forall x \enot Rmx}
\have{3}{\exists y(Rmy \eor Rym)}%\Ae{p1}
	\open
		\hypo{a1}{Rma \eor Ram}
\open
\hypo{ram}{Ram}
\close
\open
\hypo{rma}{Rma}
\open
\hypo{nram}{\enot Ram}
\have{a2}{\enot Rma}%\Ae{p2}
\close
\have{r}{Ram}%\ne{nram-rma,nram-a2}
\close
\have{a3}{Ram}%\oe{a1,ram-ram,rma-r}
		\have{a4}{\exists x Rxm}%\Ei{a3}
	\close
\have{n}{\exists x Rxm}%\Ee{3,a1-a4}
\end{proof}
\begin{proof}
\hypo{1}{\forall x(\exists yLxy \eif \forall zLzx)}
\open\hypo{2}{Lab}
\have{3}{\exists y Lay \eif \forall zLza}{}
\have{4}{\exists y Lay} {}
\have{5}{\forall z Lza} {}
\have{6}{Lca}{}
\have{7}{\exists y Lcy \eif \forall zLzc}{}
\have{8}{\exists y Lcy}{}
\have{9}{\forall z Lzc}{}
\have{10}{Lcc}{}
\have{11}{\forall x Lxx}{}
\end{proof}
\begin{proof}
\hypo{a}{\forall x(Jx \eif Kx)}
\open\hypo{b}{\exists x\forall y Lxy}
\open	\hypo{c}{\forall x Jx}
\open
	\hypo{2}{\forall y Lay}
	\have{3}{Laa}{}
	\have{d}{Ja}{}
	\have{e}{Ja \eif Ka}{}
	\have{f}{Ka}{}
	\have{4}{Ka \eand Laa}{}
	\have{5}{\exists x(Kx \eand Lxx)}{}
\close
\have{j}{\exists x(Kx \eand Lxx)}{}
\end{proof}
\end{multicols}


\problempart
\label{pr.BarbaraEtc.proof1}
In §\ref{s:MoreMonadic} problem part A, we considered fifteen syllogistic figures of Aristotelian logic. Provide proofs for each of the argument forms. NB: You will find it \emph{much} easier if you symbolise (for example) `No F is G' as `$\forall x (Fx \eif \enot Gx)$'.


\problempart
\label{pr.BarbaraEtc.proof2}
Aristotle and his successors identified other syllogistic forms which depended upon `existential import'. Symbolise each of the following argument forms in \FOL\ and offer proofs.
\begin{itemize}
	\item \textbf{Barbari.} Something is H. All G are F. All H are G. So: Some H is F
	\item \textbf{Celaront.} Something is H. No G are F. All H are G. So: Some H is not F
	\item \textbf{Cesaro.} Something is H. No F are G. All H are G. So: Some H is not F.
	\item \textbf{Camestros.} Something is H. All F are G. No H are G. So: Some H is not F.
	\item \textbf{Felapton.} Something is G. No G are F. All G are H. So: Some H is not F.
	\item \textbf{Darapti.} Something is G. All G are F. All G are H. So: Some H is F.
	\item \textbf{Calemos.} Something is H. All F are G. No G are H. So: Some H is not F.
	\item \textbf{Fesapo.} Something is G. No F is G. All G are H. So: Some H is not F.
	\item \textbf{Bamalip.} Something is F. All F are G. All G are H. So: Some H are F.
\end{itemize}

\problempart
\label{pr.someFOLproofs}
Provide a proof of each claim.
\begin{earg}
\item $\proves \forall x Fx \eor \enot \forall x Fx$
\item $\proves\forall z (Pz \eor \enot Pz)$
\item $\forall x(Ax\eif Bx), \exists x Ax \proves \exists x Bx$
\item $\forall x(Mx \eiff Nx), Ma\eand\exists x Rxa\proves \exists x Nx$
\item $\forall x \forall y Gxy\proves\exists x Gxx$
\item $\proves\forall x Rxx\eif \exists x \exists y Rxy$
\item $\proves\forall y \exists x (Qy \eif Qx)$
\item $Na \eif \forall x(Mx \eiff Ma), Ma, \enot Mb\proves \enot Na$
\item $\forall x \forall y (Gxy \eif Gyx) \proves \forall x\forall y (Gxy \eiff Gyx)$
\item $\forall x(\enot Mx \eor Ljx), \forall x(Bx\eif Ljx), \forall x(Mx\eor Bx)\proves \forall xLjx$
\end{earg}

\solutions
\problempart
\label{pr.likes}
Write a symbolisation key for the following argument, symbolise it, and prove it:
\begin{quote}
There is someone who likes everyone who likes everyone that she likes. Therefore, there is someone who likes herself.
\end{quote}

\solutions
\problempart
\label{pr.FOLequivornot}
For each of the following pairs of sentences: If they are provably equivalent, give proofs to show this. If they are not, construct an interpretation to show that they are not logically equivalent.
\begin{earg}
\item $\forall x Px \eif Qc, \forall x (Px \eif Qc)$
\item $\forall x\forall y \forall z Bxyz, \forall x Bxxx$
\item $\forall x\forall y Dxy, \forall y\forall x Dxy$
\item $\exists x\forall y Dxy, \forall y\exists x Dxy$
\item $\forall x (Rca \eiff Rxa), Rca \eiff \forall x Rxa$
\end{earg}

\solutions
\problempart
\label{pr.FOLvalidornot}
For each of the following arguments: If it is valid in \FOL, give a proof. If it is invalid, construct an interpretation to show that it is invalid.
\begin{earg}
\item $\exists y\forall x Rxy \ttherefore \forall x\exists y Rxy$
\item $\exists x(Px \eand \enot Qx) \ttherefore \forall x(Px \eif \enot Qx)$
\item $\forall x(Sx \eif Ta), Sd \ttherefore Ta$
\item $\forall x(Ax\eif Bx), \forall x(Bx \eif Cx) \ttherefore \forall x(Ax \eif Cx)$
\item $\exists x(Dx \eor Ex), \forall x(Dx \eif Fx) \ttherefore \exists x(Dx \eand Fx)$
\item $\forall x\forall y(Rxy \eor Ryx) \ttherefore Rjj$
\item $\exists x\exists y(Rxy \eor Ryx) \ttherefore Rjj$
\item $\forall x Px \eif \forall x Qx, \exists x \enot Px \ttherefore \exists x \enot Qx$
\end{earg}


\chapter{Derived Rules for \textnormal{\FOL}}\label{s:CQ}

In this section, we shall add some additional rules to the basic rules of the previous section. These govern the interaction of quantifiers and negation. But they are no substantive addition to our basic rules: for each of the proposed additions, it can be shown that their role in any proof can be wholly emulated by some suitable applications of our basic rules from §\ref{s:BasicFOL}. (The point here is as in §\ref{s:Derived}.)
 
\section{Conversion of Quantifiers}\label{cq.rules}

In §\ref{s:FOLBuildingBlocks}, we noted that $\enot\exists x\meta{A}$ is logically equivalent to $\forall x \enot\meta{A}$. We shall add some rules to our proof system that govern this. In particular, we add two rules, one for each direction of the equivalence:
	\factoidbox{\begin{minipage}{0.35\textwidth}
		\begin{proof}
		\have[m]{a}{\forall \meta{x} \enot\meta{A}}
		\have[\ ]{con}{\enot \exists \meta{x} \meta{A}}\by{CQ$_{\forall/\enot\exists}$}{a}
	\end{proof}
	\end{minipage}\qquad\begin{minipage}{0.35\textwidth}
		\begin{proof}
		\have[m]{a}{ \enot \exists \meta{x} \meta{A}}
		\have[\ ]{con}{\forall  \meta{x} \enot \meta{A}}\by{CQ$_{\enot\exists/\forall}$}{a}
	\end{proof}
	\end{minipage}
	}

 Here is a schematic proof corresponding to our first conversion of quantifiers rule, CQ$_{\forall/\enot\exists}$:
\begin{proof}
	\hypo{An}{\forall \meta{x} \enot \meta{A}}
	\open
		\hypo{E}{\exists \meta{x} \meta{A}}
		\open
			\hypo{c}{\meta{A}\subs{\meta{c}}{\meta{x}}}%\by{for $\exists$E}{}
			\open
			\hypo{nb}{\enot(\meta{B}\eand\enot\meta{B})}
			\have{nc}{\enot \meta{A}\subs{\meta{c}}{\meta{x}}}\Ae{An}
		\close
		\have{b}{\meta{B}\eand\enot\meta{B}}\ni{nb-nc,nb-c}
		\close
		\have{red2}{\meta{B}\eand\enot\meta{B}}\Ee{E,c-b}
		\close
		\have{cc}{\enot\exists\meta{x}\meta{A}}\ni{E-red2}
\end{proof}
A couple of things to note about this proof. \begin{enumerate}
\item I was hasty at line 9 – officially I ought to have applied $\eand$E to line 8, obtaining the contradictory conjuncts in the subproof, and then applied $\enot$I to the assumption opening that subproof. (But then the proof would have gone over the page.)
\item Note that we had to introduce the new name \meta{c} at line 3. Once we did so, there was no obstacle to applying $\forall$E on that newly introduced name in line 5. But if we had done things the other way around, applying $\forall$E first to some new name \meta{c}, we would have had to open the subproof with yet another new name \meta{d}. 
	\item The sentence $\meta{B}$ cannot contain the name \meta{c} if the application of $\exists$E at line 8 is to be correct. We introduce this arbitrary logical falsehood precisely so we can show that the contradictoriness of our initial assumptions does not depend on the particular choice of name. The alternative would have been to show that the assumption $\meta{A}\subs{\meta{c}}{\meta{x}}$ leads to logical falsehood, and then applied $\enot$I – but that would have left the name \meta{c} outside the scope of a subproof and would not have allowed us to apply $\exists$E.
\end{enumerate}

A similar schematic proof could be offered for the second conversion rule, CQ$_{\enot\exists/\forall}$.

Equally, we might add rules corresponding to the equivalence of $\exists\meta{x}\enot\meta{A}$ and $\enot\forall\meta{x}\meta{A}$:
\factoidbox{\begin{minipage}{0.35\textwidth}
	\begin{proof}
		\have[m]{a}{\exists \meta{x}\enot \meta{A}}
		\have[\ ]{con}{\enot \forall \meta{x} \meta{A}}\by{CQ$_{\exists/\enot\forall}$}{a}
	\end{proof}
\end{minipage}\qquad\begin{minipage}{0.35\textwidth}
	\begin{proof}
		\have[m]{a}{\enot \forall \meta{x} \meta{A}}
		\have[\ ]{con}{\exists \meta{x} \enot \meta{A}}\by{CQ$_{\enot\forall/\exists}$}{a}
	\end{proof}
\end{minipage}
	}

Here is a schematic basic proof showing that the third conversion of quantifiers rule just introduced, CQ$_{\exists/\enot\forall}$, can be emulated just using the standard quantifier rules in combination with the other rules of our system, in which some of the same issues arise as in the earlier schematic proof: 
\begin{proof}
	\hypo{nEna}{\exists \meta{x}  \enot \meta{A}} 
	\open
		\hypo{Aa}{\forall \meta{x} \meta{A}}
		\open
			\hypo{nac}{\enot \meta{A}\subs{\meta{c}}{\meta{x}}}
			\open
				\hypo{nb}{\enot(\meta{B}\eand\enot\meta{B})}
				\have{a}{\meta{A}\subs{\meta{c}}{\meta{x}}}\Ae{Aa}
				\have{r}{\enot\meta{A}\subs{\meta{c}}{\meta{x}}}\by{R}{nac}
			\close
			\have{con}{\meta{B}\eand\enot\meta{B}}\ne{nb-a,nb-r}
		\close
		\have{con1}{\meta{B}\eand\enot\meta{B}}\Ee{nEna, nac-con}
	\close
	\have{dada}{\enot \forall \meta{x} \meta{A}}\ni{Aa-con1}
\end{proof}
A similar schematic proof can be offered for the final CQ rule.

\section{Alternative Proof Systems for \FOL}

We saw in §\ref{s:alternates} that it is possible to formulate alternative proof systems that can nevertheless establish the same arguments are provable in \TFL. The same is true for \FOL. The idea is to get rid of the rules for one quantifier, retaining the rules governing the other quantifier, but then to take the conversion of quantifier rules as basic. 

So, for example, we could consider the system which has $\exists$I and $\exists$E, and also has CQ$_{\exists/\enot\forall}$ and CQ$_{\enot\forall/\exists}$. With these rules, we can emulate $\forall$E and $\forall$I. A schematic proof showing how to emulate $\forall$E using our other basic rules is this:
\begin{proof}
	\have{a}{\forall \meta{x}\meta{A}}
	\open
	\hypo{b}{\enot\meta{A}\subs{\meta{c}}{\meta{x}}}
	\have{c}{\exists \meta{x}\enot\meta{A}}\Ei{b}
	\have{d}{\enot\forall \meta{x}\meta{A}}\by{CQ$_{\exists/\enot\forall}$}{c}
	\have{e}{\forall \meta{x}\meta{A}}\by{R}{a}
	\close
	\have{f}{\meta{A}\subs{\meta{c}}{\meta{x}}}\ne{b-e,b-d}
\end{proof}

A schematic proof emulating $\forall$I using our other basic rules is trickier. Here it is: 
\begin{proof}
	\hypo{aaa}{\Gamma}
	\have[\ ]{}{\vdots}
	\open
	\hypo[m]{a}{\enot\forall \meta{x}\meta{A}}
	\have{b}{\exists \meta{x}\enot\meta{A}}\by{CQ$_{\enot\forall/\exists}$}{a}
	\open
	\hypo{c}{\enot\meta{A}\subs{\meta{c}}{\meta{x}}}
	\open
	\hypo{d}{\enot(\meta{B}\eand\enot\meta{B})}
	\have[\ ]{}{\vdots}
	\have[n]{e}{\meta{A}\subs{\meta{c}}{\meta{x}}}\by{Original proof from}{aaa}
	\have{r}{\enot\meta{A}\subs{\meta{c}}{\meta{x}}}\by{R}{c}
	\close
	\have{f}{\meta{B}\eand\enot\meta{B}}\ne{d-e,d-r}
	\close
	\have{g}{\meta{B}\eand\enot\meta{B}}\Ee{b,c-f}
	\close
	\have{h}{\forall \meta{x}\meta{A}}\ne{a-g}
\end{proof}
To understand this schematic proof, what we need to remember is that, in order for the original $\forall$I rule to apply, we must already have a proof of $\meta{A}\subs{\meta{c}}{\meta{x}}$ which relies on assumptions $\Gamma$ that do not mention \meta{c} at all. The trick is to make use of that proof \emph{inside} an assumption about an existential witness. We don't try to perform that proof to derive $\meta{A}\subs{\meta{c}}{\meta{x}}$ and then attempt to manipulate $\enot\forall \meta{xA}$ to generate a logical falsehood. Rather, we first assume $\enot\forall \meta{xA}$, apply quantifier conversion to obtain $\exists \meta{x}\enot\meta{A}$, assume that \meta{c} witnesses that existential claim so that $\enot\meta{A}\subs{\meta{c}}{\meta{x}}$, and then use our original proof to derive $\meta{A}\subs{\meta{c}}{\meta{x}}$ at line $n$. To avoid problems with the name appearing at the bottom of the existential witness subproof, we perform the same trick of assuming the falsehood of an arbitrary logical falsehood (so long as \meta{B} doesn't include \meta{c}), and then we manage to derive from $\Gamma$ what we had hoped to: that $\forall \meta{xA}$.

\keyideas{
	\item The derived rules for \FOL\ concern the interaction of quantifiers with negation.
	\item Do not make use of these derived rules unless you are explicitly told you may do so.
}

\practiceproblems

\problempart
Show that the following are jointly contrary:
\begin{earg}
\item $Sa\eif Tm, Tm \eif Sa, Tm \eand \enot Sa$
\item $\enot\exists x Rxa, \forall x \forall y Ryx$
\item $\enot\exists x \exists y Lxy, Laa$
\item $\forall x(Px \eif Qx), \forall z(Pz \eif Rz), \forall y Py, \enot Qa \eand \enot Rb$
\end{earg}

\problempart
Show that each pair of sentences is provably equivalent:
\begin{earg}
\item $\forall x (Ax\eif \enot Bx), \enot\exists x(Ax \eand Bx)$
\item $\forall x (\enot Ax\eif Bd), \forall x Ax \eor Bd$
\end{earg}

\problempart
In §\ref{s:MoreMonadic}, I considered what happens when we move quantifiers `across' various connectives. Show that each pair of sentences is provably equivalent:
\begin{earg}
\item $\forall x (Fx \eand Ga), \forall x Fx \eand Ga$
\item $\exists x (Fx \eor Ga), \exists x Fx \eor Ga$
\item $\forall x(Ga \eif Fx), Ga \eif \forall x Fx$
\item $\forall x(Fx \eif Ga), \exists x Fx \eif Ga$
\item $\exists x(Ga \eif Fx), Ga \eif \exists x Fx$
\item $\exists x(Fx \eif Ga), \forall x Fx \eif Ga$
\end{earg}
NB: the variable `$x$' does not occur in `$Ga$'.

When all the quantifiers occur at the beginning of a sentence, that sentence is said to be in \emph{prenex normal form}. These equivalences are sometimes called \emph{prenexing rules}, since they give us a means for putting any sentence into prenex normal form.



\problempart
Offer proofs which justify the addition of the other CQ rules as derived rules.



\chapter{Rules for Identity}\label{ch.identity}

\section{Identity Introduction}\label{idint}
In §\ref{s:Interpretations}, I mentioned the philosophically contentious thesis of the \emph{identity of indiscernibles}. This is the claim that objects which are indiscernible in every way – which means, for us, that exactly the same predicates are true of both objects – are, in fact, identical to each other. I also mentioned that in \FOL, this thesis is not true. There are interpretations in which, for each property denoted by any predicate of the language, two distinct objects either both have that property, or both lack that property. They may well differ on some property `in reality', but there is nothing in \FOL\ which guarantees that that property is assigned as the interpretation of any predicate of the language. 
It follows that, no matter how many sentences of \FOL\ about those two objects I assume, those sentences will not entail that these distinct objects are identical (luckily for us). Unless, of course, you tell me that the two objects are, in fact, identical. But then the argument will hardly be very illuminating.

The consequence of this, for our proof system, is that there are no sentences that do not already contain the identity predicate that could justify the conclusion `$a=b$'. This means that the identity introduction rule will not justify `$a=b$', or any other identity claim containing two different names.

However, every object is identical to itself. No premises, then, are required in order to conclude that something is identical to itself. So this will be the identity introduction rule:
\factoidbox{
\begin{proof}
	\have[\ \,\,\,]{x}{\meta{c}=\meta{c}} \idi{}
\end{proof}}
Notice that this rule does not require referring to any prior lines of the proof, nor does it rely on any assumptions. For any name \meta{c}, you can write $\meta{c}=\meta{c}$ at any point, with only the {=}I rule as justification. 

Recall that a relation is reflexive iff it holds between anything in the domain and itself (§§\ref{lli} and\ref{binary}). Let's see this rule in action, in a proof that identity is reflexive:
\begin{proof}
	\have{a}{a=a}\idi{}
	\have{b}{\forall x x=x}\Ai{a}
\end{proof} This seems like magic! But note that the first line is not an assumption (there is no horizontal line), and hence not an undischarged assumption. So the constant `$a$' appears in no undischarged assumption or anywhere in the proof other than in `$x=x$'$\subs{a}{x}$, so the conclusion `$\forall x x=x$' follows by legitimate application of the $\forall$I rule.


\section{Identity Elimination} % (fold)
\label{idelim}


Our elimination rule is more fun. If you have established `$a=b$', then anything that is true of the object named by `$a$' must also be true of the object named by `$b$', since it just is the object named by `$a$'. So for any sentence with `$a$' in it, given the prior claim that `$a=b$', you can replace some or all of the occurrences of `$a$' with `$b$' and produce an equivalent sentence. For example, from `$Raa$' and `$a = b$', you are justified in inferring `$Rab$', `$Rba$' or `$Rbb$'. More generally:
\factoidbox{\begin{proof}
	\have[m]{e}{\meta{a}=\meta{b}}
	\have[n]{a}{\meta{A}\subs{\meta{a}}{\meta{x}}}
	\have[\ ]{ea1}{\meta{A}\subs{\meta{b}}{\meta{x}}} \ide{e,a}
\end{proof}}This uses our standard notion of substitution – it basically says that if you have some sentence which arises from substituting $\meta{a}$ for some variable in a formula, then you are entitled to another substitution instance of the same formula using $\meta{b}$ instead. 
 Lines $m$ and $n$ can occur in either order, and do not need to be adjacent, but we always cite the statement of identity first. 

Note that nothing in the rule forbids the constant $\meta{b}$ from occurring in \meta{A}. So this is a perfectly good instance of the rule: \begin{proof}
	\hypo{ab}{a=b}
	\have{a}{Rab}
	\have{b}{Rbb}\ide{ab,a}
\end{proof} Here, `$Rab$' is `$Rxb$'$\subs{a}{x}$, and the conclusion `$Rbb$' is `$Rxb$'$\subs{b}{x}$, which conforms to the rule. This formulation allows us, in effect, to replace some-but-not-all occurrences of a name in a sentence by a co-referring name.
This rule is sometimes called \emph{Leibniz's Law} – though recall §\ref{lli}, where we used that name for a claim about the interpretation of `$=$'.


To see the rules in action, we shall prove some quick results. Recall that a relation is symmetric iff whenever it holds between x and y in one direction, it holds also between y and x in the other direction (§\ref{graph}). This condition can be expressed as a sentence of \FOL: 


 So first, we shall prove that identity is symmetric, a result we already noted on semantic grounds in §\ref{lli}:
\begin{proof}
	\open
		\hypo{ab}{a = b}
		\have{aa}{a = a}\idi{}
		\have{ba}{b = a}\ide{ab, aa}
	\close
	\have{abba}{a = b \eif b =a}\ci{ab-ba}
	\have{ayya}{\forall y (a = y \eif y = a)}\Ai{abba}
	\have{xyyx}{\forall x \forall y (x = y \eif y = x)}\Ai{ayya}
\end{proof}
Line 2 is just `$x=a$'$\subs{a}{x}$, as well as being of the right form for $=$I, and line 3 is just $x=a$'$\subs{b}{x}$, so the move from 2 to 3 is in conformity with the $=$E rule given the opening assumption.

Having noted the symmetry of identity, note that we can use this to establish the following schematic proof that allows us to use $\meta{a}=\meta{b}$ to also move from a claim about $\meta{b}$ to a claim about $\meta{a}$, not just \emph{vice versa} as in our $=$E rule.:
\begin{proof}
	\have[m]{j}{\meta{a}=\meta{b}}
	\have{i}{\meta{a}=\meta{a}}\idi{}
	\have{e}{\meta{b}=\meta{a}}\ide{j,i}
	\have[n]{a}{\meta{A}\subs{\meta{b}}{\meta{x}}}
	\have[\ ]{ea1}{\meta{A}\subs{\meta{a}}{\meta{x}}} \ide{e,a}
\end{proof} This schematic proof actually justifies this derived rule, to save time:\label{id.es} \factoidbox{
	\begin{proof}
		\have[m]{e}{\meta{a}=\meta{b}}
	\have[n]{a}{\meta{A}\subs{\meta{b}}{\meta{x}}}
	\have[\ ]{ea1}{\meta{A}\subs{\meta{a}}{\meta{x}}} \by{=ES}{e,a}
	\end{proof}
}


A relation is  \define{transitive} iff whenever it holds between x and y and between y and z, it also holds between x and z. (In the directed graph representation of the relation introduced in §\ref{graph}, if there is a path along arrows going from node a to node b via a third node, there is also a direct arrow from a to b.) Second, we shall prove that identity is \emph{transitive}:
\begin{proof}
	\open
		\hypo{abc}{a = b \eand b = c}
		\have{bc}{b = c}\ae{abc}
		\have{ab}{a = b}\ae{abc}
		\have{ac}{a = c}\ide{bc,ab}
	\close
	\have{con}{(a = b \eand b =c) \eif a = c}\ci{abc-ac}
	\have{conz}{\forall z((a = b \eand b = z) \eif a = z)}\Ai{con}
	\have{cony}{\forall y\forall z((a = y \eand y = z) \eif a = z)}\Ai{conz}
	\have{conx}{\forall x \forall y \forall z((x = y \eand y = z) \eif x = z)}\Ai{cony}
\end{proof}
We obtain line 4 by replacing `$b$' in line 3 with `$c$'; this is justified given line 2, `$b=c$'. We could alternatively have used the derived rule =ES to replace `$b$' in line 2 with `$a$', justified by line 3, `$a=b$'. 

A relation that is reflexive, symmetric, and transitive is known as an \define{equivalence relation}. So we've proved that identity is an equivalence relation. 

\subsection{Definite Descriptions in Proofs}

I want to close this section by giving an example of how proofs involving definite descriptions work. Remember from §\ref{subsec.defdesc} that we symbolised a definite description `the F is G', following Russell, as $$\exists x (\meta{F}x  \eand \forall y(\meta{F}y \eiff x=y) \eand \meta{G}x).$$ (I omit interior parentheses to aid legibility.) 

Using this kind of approach, let us consider this argument: \begin{quote}
	The F is G; the H isn't G; so the F isn't H.
\end{quote} The argument, symbolised, looks like this: 
\begin{align*}
	\exists x (Fx  \eand \forall y(Fy \eiff x=y)\eand Gx),& \exists x (Hx \eand \forall y(Hy \eiff x=y) \eand \enot Gx) \\ &\proves \exists x (Fx \eand \forall y(Fy \eiff x=y)\eand \enot Hx).
\end{align*} Here's the proof: \begin{proof}
	\hypo{a}{\exists x (Fx  \eand \forall y(Fy \eiff x=y)\eand Gx)}
	\hypo{b}{\exists x (Hx \eand \forall y(Hy \eiff x=y) \eand \enot Gx)}
	\open
	\hypo{c}{(Fa  \eand \forall y(Fy \eiff a=y)\eand Ga)}
	\open
	\hypo{d}{(Hb \eand \forall y(Hy \eiff b=y) \eand \enot Gb)}
	\open
	\hypo{e}{Ha}
	\have{f}{\forall y(Hy \eiff b=y)}\ae{d}
	\have{g}{(Ha \eiff b=a)}\Ae{f}
	\have{h}{b=a}\be{g,e}
	\have{i}{Ga}\ae{c}
	\have{j}{\enot Gb}\ae{d}
	\have{k}{\enot Ga}\ide{h,j}
	\close
	\have{l}{\enot Ha}\ni{e-i,e-k}
	\close
	\have{m}{\enot Ha}\Ee{b,d-l}
	\have{n}{(Fa \eand \forall y (Fy \eiff a=y))}\ae{c}
	\have{o}{(Fa \eand \forall y (Fy \eiff a=y) \eand \enot Ha)}\ai{n,m}
	\have{p}{\exists x (Fx \eand \forall y (Fy \eiff x=y) \eand \enot Hx)}\Ei{o}
	\close
	\have{q}{\exists x (Fx \eand \forall y (Fy \eiff x=y) \eand \enot Hx)}\Ee{a,c-p}

\end{proof}

\keyideas{
	\item Adding rules for identity completes our presentation of the natural deduction system for \FOL.
	\item The identity elimination rule is basically Leibniz' Law. The idenity introduction rule expresses how trivial identity is when the same name flanks the identity symbol.
	\item We can prove all the properties we traditionally ascribe to identity – reflexivity, symmetry, and transitivity – so identity is an equivalence relation.
}
\newpage
\practiceproblems
\problempart
\label{pr.identity}
Provide a proof of each of the following.
\begin{earg}
\item $Pa \eor Qb, Qb \eif b=c, \enot Pa \proves Qc$
\item $m=n \eor n=o, An \proves Am \eor Ao$
\item $\forall x\ x=m, Rma\proves \exists x Rxx$
\item $\forall x\forall y(Rxy \eif x=y)\proves Rab \eif Rba$
\item $\enot \exists x\enot x = m \proves \forall x\forall y (Px \eif Py)$
\item $\exists x Jx, \exists x \enot Jx\proves \exists x \exists y\ \enot x = y$
\item $\forall x(x=n \eiff Mx), \forall x(Ox \eor \enot Mx)\proves On$
\item $\exists x Dx, \forall x(x=p \eiff Dx)\proves Dp$
\item $\exists x\bigl( (Kx \eand \forall y(Ky \eif x=y)) \eand Bx\bigr), Kd\proves Bd$
\item $\proves Pa \eif \forall x(Px \eor \enot x = a)$
\end{earg}

\problempart
Show that the following are provably equivalent:
\begin{itemize}
\item $\exists x \bigl(\bigl(Fx \eand \forall y (Fy \eif x = y) \bigr) \eand x = n\bigr)$
\item $Fn \eand \forall y (Fy \eif n= y)$
\end{itemize}
And hence that both have a decent claim to symbolise the English sentence `Nick is the F'.



\problempart
In §\ref{sec.identity}, I claimed that the following are logically equivalent symbolisations of the English sentence `there is exactly one F':
\begin{itemize}
\item $\exists x Fx \eand \forall x \forall y \bigl( (Fx \eand Fy) \eif x = y\bigr)$
\item $\exists x \bigl(Fx \eand \forall y (Fy \eif x = y) \bigr)$
\item $\exists x \forall y (Fy \eiff x = y)$
\end{itemize}
Show that they are all provably equivalent. (\emph{Hint}: to show that three claims are provably equivalent, it suffices to show that the first proves the second, the second proves the third and the third proves the first; think about why.)



\problempart
Symbolise the following argument
	\begin{quote}
		There is exactly one F. There is exactly one G. Nothing is both F and G. So: there are exactly two things that are either F or G.
	\end{quote}
And offer a proof of it.
%\begin{itemize}
%\item  $\exists x \bigl(Fx \eand \forall y (Fy \eif x = y) \bigr), \exists x \bigl(Gx \eand \forall y ( Gy \eif x = y) \bigr), \forall x (\enot Fx \eor \enot Gx) \proves \exists x \exists y \bigl(\enot x = y \eand \forall z ((Fz \eor Gz) \eif (x = y \eor x = z)) \bigr)$
%\end{itemize}

\problempart
What condition on the directed graph of a relation corresponds to that relation being an equivalence relation?

\chapter{Proof-Theoretic Concepts and Semantic Concepts}\label{sec:soundcomp}
We have used two different turnstiles in this book.  This:
$$\meta{A}_1, \meta{A}_2, …, \meta{A}_n \proves \meta{C}$$
means that there is some proof which starts with assumptions $\meta{A}_1, \meta{A}_2, …, \meta{A}_n$ and ends with $\meta{C}$ (and no undischarged assumptions other than $\meta{A}_1, \meta{A}_2, …, \meta{A}_n$). This is a \emph{proof-theoretic notion}.

By contrast, this:
$$\meta{A}_1, \meta{A}_2, …, \meta{A}_n \entails \meta{C}$$
means that there is no valuation (or interpretation) which makes all of $\meta{A}_1, \meta{A}_2, …, \meta{A}_n$ true and makes $\meta{C}$ false. This concerns assignments of truth and falsity to sentences. It is a \emph{semantic notion}.

I cannot emphasise enough that these are different notions. But I can emphasise it a bit more: \emph{They are different notions.}

Once you have internalised this point, continue reading. 

\section{Connecting Entailment and Provability}

Although our semantic and proof-theoretic notions are different, there is a deep connection between them. To explain this connection, I shall start by considering the relationship between logical truths and theorems.

To show that a sentence is a theorem, you need only perform a proof. Granted, it may be hard to produce a twenty line proof, but it is not so hard to check each line of the proof and confirm that it is legitimate; and if each line of the proof individually is legitimate, then the whole proof is legitimate. Showing that a sentence is a logical truth, though, requires reasoning about all possible interpretations. Given a choice between showing that a sentence is a theorem and showing that it is a logical truth, it would be easier to show that it is a theorem.

Contrawise, to show that a sentence is \emph{not} a theorem is hard. We would need to reason about all (possible) proofs. That is very difficult. But to show that a sentence is not a logical truth, you need only construct an interpretation in which the sentence is false. Granted, it may be hard to come up with the interpretation; but once you have done so, it is relatively straightforward to check what truth value it assigns to a sentence. Given a choice between showing that a sentence is not a theorem and showing that it is not a logical truth, it would be easier to show that it is not a logical truth.

Fortunately, \emph{a sentence is a theorem if and only if it is a logical truth}. As a result, if we provide a proof of $\meta{A}$ on no assumptions, and thus show that $\meta{A}$ is a theorem, we can legitimately infer that $\meta{A}$ is a logical truth; i.e., $\entails\meta{A}$. Similarly, if we construct an interpretation in which \meta{A} is false and thus show that it is not a logical truth, it follows that \meta{A} is not a theorem.

More generally, we have the following powerful result:\factoidbox{
$$\meta{A}_{1}, \meta{A}_{2}, …, \meta{A}_{n} \proves\meta{B} \textbf{ iff }\meta{A}_{1}, \meta{A}_{2}, …, \meta{A}_{n} \entails\meta{B}.$$}
This shows that, whilst provability and entailment are \emph{different} notions, they are extensionally equivalent, holding between just the same sentences in our languages. As such:
	\begin{itemize}
		\item An argument is \emph{valid} iff \emph{the conclusion can be proved from the premises}.
		\item Two sentences are \emph{logically equivalent} iff they are \emph{provably equivalent}.
		\item Sentences are \emph{jointly consistent} iff they are \emph{not jointly contrary}.
	\end{itemize}
For this reason, you can pick and choose when to think in terms of proofs and when to think in terms of valuations/interpretations, doing whichever is easier for a given task. The table on the next page summarises which is (usually) easier.

It is intuitive that provability and semantic entailment should agree. But – let me repeat this – do not be fooled by the similarity of the symbols `$\entails$' and `$\proves$'. These two symbols have very different meanings. And the fact that provability and semantic entailment agree is not an easy result to come by. 

We showed part of this result along the way, actually. All those little observations I made about how our proof rules were good each took the form of an argument that whenever $\meta{A}_{1}, \meta{A}_{2}, …, \meta{A}_{n} \proves\meta{B}$, then also $\meta{A}_{1}, \meta{A}_{2}, …, \meta{A}_{n} \entails\meta{B}$. This is sometimes known as the \define{soundness} of our proof system, that whenever you can prove something, there is a corresponding valid argument.

\section{Next Steps in Logic}\label{s:nextsteps}

The other half of this result is more demanding. This is the \define{completeness} of our proof system, that whenever there is a valid argument, there is a proof. It is somewhere around this point, of demonstrating the completeness half of the agreement between  provability and semantic entailment, when introductory logic becomes intermediate logic. More than one of the authors of this book have written distinct sequels in which the completeness of \TFL\ is established, along with other results of a `metalogical' flavour. The interested student is directed to these texts: \begin{itemize}
	\item Tim Button's book is \emph{Metatheory}; it covers just \TFL, and some of the results there were actually covered in this book already, notably results around expressiveness of \TFL\ in §\ref{s:expressiveness}. \emph{Metatheory} is available at \href{http://www.homepages.ucl.ac.uk/~uctytbu/Metatheory.pdf}{\nolinkurl{www.homepages.ucl.ac.uk/~uctytbu/Metatheory.pdf}} It should be accessible for self-study by students who have successfully completed this course.

\item Antony Eagle's book \emph{Elements of Deductive Logic} goes rather further than \emph{Metatheory}, including direct proofs of Compactness for \TFL, consideration of alternative derivation systems, discussion of computation and decidability, and metatheory for \FOL. It is available at \href{https://github.com/antonyeagle/edl}{\nolinkurl{github.com/antonyeagle/edl}}.
\end{itemize}

tense example from §\ref{FOL.semantics}.



\keyideas{
	\item The two turnstiles are distinct concepts. But they coincide: every provable argument corresponds to an entailment, and \emph{vice versa}.
	\item This means that we can pick and choose our methods to suit our task. If we want to show an entailment, we can sometimes most effectively proceed by providing a proof. And if we want to show something isn't provable, it can be more efficient to provide a countermodel.
	\item With the basics under your belt, there are many possible next steps in logic. It is natural to try to establish more metatheoretical results about the systems we have discussed, and some resources are indicated in the text to help you do this if you wish.
}


\begin{sidewaystable}
\begin{center}
\begin{tabular*}{\textwidth}{p{.25\textheight}p{.325\textheight}p{.325\textheight}}
 & \textbf{Yes}  & \textbf{No}\\
\\
Is \meta{A} a \textbf{logical truth}? 
& give a proof which shows $\proves\meta{A}$ 
& give an interpretation in which \meta{A} is false\\
\\
Is \meta{A} a \textbf{logical falsehood}? &
give a proof which shows $\proves\enot\meta{A}$ & 
give an interpretation in which \meta{A} is true\\
\\
%Is \meta{A} contingent? & 
%give two interpretations, one in which \meta{A} is true and another in which \meta{A} is false & give a proof which either shows $\proves\meta{A}$ or $\proves\enot\meta{A}$\\
%\\
Are \meta{A} and \meta{B} \textbf{equivalent}? &
give two proofs, one for $\meta{A}\proves\meta{B}$ and one for $\meta{B}\proves\meta{A}$  
& give an interpretation in which \meta{A} and \meta{B} have different truth values\\
\\
Are $\meta{A}_{1}, \meta{A}_2, …, \meta{A}_n$ \textbf{jointly consistent}? 
& give an interpretation in which all of $\meta{A}_1, \meta{A}_2, …, \meta{A}_n$ are true 
& prove a logical falsehood from assumptions $\meta{A}_1, \meta{A}_2, …, \meta{A}_n$\\
\\
Is $\meta{A}_1, \meta{A}_2, …, \meta{A}_n \ttherefore \meta{C}$ \textbf{valid}? 
& give a proof with assumptions $\meta{A}_1, \meta{A}_2, …, \meta{A}_n$ and concluding with \meta{C}
& give an interpretation in which each of $\meta{A}_1, \meta{A}_2, …, \meta{A}_n$ is true and \meta{C} is false\\
\end{tabular*}
\end{center}
\end{sidewaystable}

\chapter{Next Steps in Logic}
