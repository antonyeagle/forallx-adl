%!TEX root = forallx-adl.tex
\thispagestyle{empty}
\onecolumn



\subsubsection*{Acknowledgements}
\addtocontents{toc}{\vspace{\normalbaselineskip}}
\addcontentsline{toc}{chapter}{Acknowledgements, etc.}
P.D.\ Magnus would like to thank the people who made this project possible. Notable among these are Cristyn Magnus, who read many early drafts; Aaron Schiller, who was an early adopter and provided considerable, helpful feedback; {and} Bin Kang, Craig Erb, Nathan Carter, Wes McMichael, Selva Samuel,  Dave Krueger, Brandon Lee, and the students of Introduction to Logic, who detected various errors in previous versions of the book. \medskip

Tim Button would like to thank P.D.\ Magnus for his extraordinary act of generosity, in making \forallx\ available to everyone. Thanks also to Alfredo Manfredini Böhm, Sam Brain, Felicity Davies, Emily Dyson, Phoebe Hill, Richard Jennings, Justin Niven,  and Igor Stojanovic for noticing errata in earlier versions. \medskip

Antony Eagle would like to thank P.D.\ Magnus and Tim Button for their work from which this text derives. He thanks Atheer Al-Khalfa, Caitlin Bettess, Andrew Carter, Keith Dear, Jack Garland, Bowen Jiang, Millie Lewis, Yaoying Li, Jon Opie, Matt Nestor, Jaime von Schwarzburg, and Mike Walmer for comments. % updated October 2018

\subsubsection*{About the Authors}


P.D.\ Magnus is a professor of philosophy in Albany, New York. His primary research is in the philosophy of science. \href{https://www.fecundity.com/job/}{\nolinkurl{www.fecundity.com/job/}}
\medskip

Tim Button is a University Lecturer, and Fellow of St John's College, at the University of Cambridge. His first book, \emph{The Limits of Realism}, was published by Oxford University Press in 2013. \href{http://nottub.com}{\nolinkurl{nottub.com}}
\medskip

Antony Eagle teaches philosophy at the University of Adelaide. His research interests include metaphysics, philosophy of probability, philosophy of physics, and philosophy of logic and language. \href{https://antonyeagle.org}{\nolinkurl{antonyeagle.org}}



\vfill

In the Introduction to his \emph{Symbolic Logic}, Charles Lutwidge Dodson advised: \begin{quote}
	When you come to any passage you don't understand, \emph{read it again}: if you \emph{still} don't understand it, \emph{read it again}: if you fail, even after \emph{three} readings, very likely your brain is getting a little tired. In that case, put the book away, and take to other occupations, and next day, when you come to it fresh, you will very likely find that it is \emph{quite} easy.
\end{quote}



The same might be said for this volume, although readers are forgiven if they take a break for snacks after \emph{two} readings.
}



